\vspace{1em}
\addcontentsline{toc}{section}{()  \textbf{*lin \STEDTU{⪤} *liŋ/k} SKIN}
\markright{() [\#3611]  \textbf{*lin} \STEDTU{⪤} \textbf{*liŋ/k} SKIN}
{\large
\parindent=-1em
() \hspace{\stretch{1}} \textbf{*lin} \STEDTU{⪤} \textbf{*li\begin{tabular}[c]{c}ŋ\\k\\\end{tabular}}\hspace{\stretch{1}}\textbf{SKIN} \textit{\tiny[\#3611]}}

This root is poorly attested so far, with no more than one putative reflex in any given subgroup.



{\footnotesize
1. Kamarupan\\
\fascicletablebegin
Miji &ñiw-\textbf{lji} &lip &\mbox{IMS-Miji} &\hspace*{1ex}{\tiny 2052,\textasciitilde}\\
\end{longtable}
2.1.1. Western Himalayish\\
\fascicletablebegin
Kanauri &ṭa\textbf{löṅ} &skin of cattle &\mbox{JAM-Ety} &\hspace*{1ex}{\tiny m,\textasciitilde}\\
\end{longtable}
2.3.1. Kham-Magar-Chepang-Sunwar\\
\fascicletablebegin
Magar &bur\,\textbf{lin} &lip &\mbox{JAM-Ety} &\hspace*{1ex}{\tiny 452,\textasciitilde}\\
\end{longtable}
2.3.2. Kiranti\\
\fascicletablebegin
Bahing &kheʔ\,be\,\textbf{liŋ}\,ma &skin &\mbox{BM-Bah} &\hspace*{1ex}{\tiny 586,792,\textasciitilde,s}\\
\end{longtable}
}

\vspace{1em}
\addcontentsline{toc}{section}{()  \textbf{*lak} SKIN}
\markright{() [\#602]  \textbf{*lak} SKIN}
{\large
\parindent=-1em
() \hspace{\stretch{1}} \textbf{*lak}\hspace{\stretch{1}}\textbf{SKIN} \textit{\tiny[\#602]}}

This root has so far only been unearthed in the poorly known Luish branch of Tibeto-Burman.



{\footnotesize
4.3. Luish\\
\fascicletablebegin
Ganan &ʃă\,\textbf{le¹} &skin &\mbox{GHL-PPB}:L.161 &\raisebox{-0.5ex}{\footnotemark}\footnotetext{These Ganan and Kadu forms superficially resemble Loloish reflexes of \textbf{\textit{\tiny[\#596]}} \textbf{*N/s‑k‑rəy} SKIN, but they have been assigned to the present etymon because of their resemblance to other Luish forms.}
{\tiny 34,\textasciitilde}\\
 &ʃă\,\textbf{re¹} &skin &\mbox{GHL-PPB}:L.161 &\hspace*{1ex}{\tiny 34,\textasciitilde}\\
Kadu &sa\textbf{lê} &skin &\mbox{JAM-Ety} &\hspace*{1ex}{\tiny 34,\textasciitilde}\\
Kadu (Kantu) &sʻa\,\textbf{le³} &skin &\mbox{GHL-PPB}:L.161 &\hspace*{1ex}{\tiny 34,\textasciitilde}\\
Lui &\textbf{la}\,ho &skin &\mbox{JAM-Ety} &\hspace*{1ex}{\tiny \textasciitilde,589}\\
 &\textbf{lök}\,rök &skin &\mbox{JAM-Ety} &\hspace*{1ex}{\tiny \textasciitilde,782}\\
Sak (Bawtala) &ă\,\textbf{laᴜʔ²} &skin &\mbox{GHL-PPB}:L.161 &\hspace*{1ex}{\tiny p,\textasciitilde}\\
Sak (Dodem) &\textbf{lɑk²} &skin &\mbox{GHL-PPB}:L.161 &\hspace*{1ex}\\
\end{longtable}
}

\vspace{1em}
\addcontentsline{toc}{section}{(166)  \textbf{*N/s-k-rəy} SKIN}
\markright{(166) [\#596]  \textbf{*N/s-k-rəy} SKIN}
{\large
\parindent=-1em
(166) \hspace{\stretch{1}} \textbf{*\begin{tabular}[c]{c}N\\s\\\end{tabular}-k-rəy}\hspace{\stretch{1}}\textbf{SKIN} \textit{\tiny[\#596]}}

The nasal prefix is well-attested in Loloish and Qiangic. It appears overtly in many Northern Loloish languages, and indirectly in e.g.\ the Lahu voiced initial.

See \textit{HPTB} PLB \textbf{*m-k-rəy¹}, p. 189.



{\footnotesize
1.1. North Assam\\
\fascicletablebegin
Idu &tia³⁵\textbf{ndze⁵⁵} &lip &\mbox{SHK-Idu}:3.9 &\raisebox{-0.5ex}{\footnotemark}\footnotetext{1st morpheme \textbf{tia³⁵} occurs also in ‘tooth’ \textbf{tia³⁵pɹa³⁵}}
{\tiny m,\textasciitilde}\\
\end{longtable}
1.2. Kuki-Chin\\
\fascicletablebegin
Tiddim Chin &\textbf{ˊciː} &skin &\mbox{EJAH-TC} &\hspace*{1ex}\\
Tiddim &\textbf{ˊciː} &skin &\mbox{JAM-Ety} &\hspace*{1ex}\\
\end{longtable}
1.3. Naga\\
\fascicletablebegin
Angami (Khonoma) &\textbf{zhü} &skin &\mbox{GEM-CNL} &\hspace*{1ex}\\
Angami (Kohima) &u\,\textbf{zhü} &skin &\mbox{GEM-CNL} &\hspace*{1ex}{\tiny p,\textasciitilde}\\
Chokri &(u) \textbf{yi³¹} &skin &\mbox{VN-ChkQ}:8.2 &\hspace*{1ex}{\tiny p,\textasciitilde}\\
Khezha &ʼè\,\textbf{ži} &skin &\mbox{SY-KhözhaQ}:8.2 &\hspace*{1ex}{\tiny p,\textasciitilde}\\
Khoirao &ta\,\textbf{ghi} &skin &\mbox{GEM-CNL} &\raisebox{-0.5ex}{\footnotemark}\footnotetext{\textbf{gh} is a voiced velar fricative.}
{\tiny p,\textasciitilde}\\
Liangmei &pa\,\textbf{gih} &skin &\mbox{GEM-CNL} &\hspace*{1ex}{\tiny m,\textasciitilde}\\
Maram &tu\,\textbf{gi} &skin &\mbox{GEM-CNL} &\hspace*{1ex}{\tiny p,\textasciitilde}\\
Rongmei &ka\,\textbf{gi} &skin &\mbox{GEM-CNL} &\hspace*{1ex}{\tiny p,\textasciitilde}\\
 &məi\,\textbf{g̯i} &skin &\mbox{JAM-Rong} &\hspace*{1ex}{\tiny 39,\textasciitilde}\\
Simi &a\,\textbf{yi}\,ko &skin &\mbox{Qbp-Sumi}:8.2 &\hspace*{1ex}{\tiny p,\textasciitilde,586/ 589}\\
Tangkhul &ā\textbf{hui} &skin &\mbox{JAM-Ety} &\hspace*{1ex}{\tiny p,\textasciitilde}\\
 &¹ə\textbf{³hui} &skin &\mbox{AW-TBT}:170 &\raisebox{-0.5ex}{\footnotemark}\footnotetext{The development posited here is *kr > *hr > h.}
{\tiny p,\textasciitilde}\\
\end{longtable}
1.6. Mru\\
\fascicletablebegin
Mru &ʃɛ̆ \textbf{ʀeʔ²} &skin &\mbox{GHL-PPB}:S.16 &\hspace*{1ex}{\tiny 34,\textasciitilde}\\
\end{longtable}
2.1.2. Bodic\\
\fascicletablebegin
Tibetan (Written) &\textbf{skyi} &scales; outer skin; skin &\mbox{GHL-PPB}:S.16,X.41 &\hspace*{1ex}\\
 &\textbf{skyi}-mo &skin &\mbox{ZLS-Tib}:49 &\hspace*{1ex}{\tiny \textasciitilde,m}\\
\end{longtable}
2.1.4. Tamangic\\
\fascicletablebegin
{}*Tamang &*\textbf{ɖi³ / (ɖwi)} &skin &\mbox{MM-K78}:42 &\raisebox{-0.5ex}{\footnotemark}\footnotetext{The Chantyal (Tamangic) form \textbf{chala} is a borrowng from Indo-Aryan. See also \textbf{kəpal‑ye chala}, lit. “head-skin”.}
\\
 &*\textbf{ᴬɖi} &skin*** &\mbox{MM-Thesis}:408 &\hspace*{1ex}\\
Gurung (Ghachok) &mā\,\textbf{diq} &lip &\mbox{JAM-Ety} &\hspace*{1ex}{\tiny m,\textasciitilde}\\
 &ma\,\textbf{diq} &lip &\mbox{SIL-Gur}:2.A.25 &\hspace*{1ex}{\tiny m,\textasciitilde}\\
Manang (Gyaru) &\textbf{ɖiy³} &skin &\mbox{YN-Man}:049 &\hspace*{1ex}\\
Manang (Ngawal) &\textbf{³ʈi} &skin &\mbox{MM-K78}:42 &\raisebox{-0.5ex}{\footnotemark}\footnotetext{We are assuming a Tamangic development of *kr > retroflex stop.}
\\
Manang (Prakaa) &\textbf{²ʈi} &skin &\mbox{HM-Prak}:0038 &\hspace*{1ex}\\
 &\textbf{³ʈi} &skin &\mbox{MM-Thesis}:408 &\hspace*{1ex}\\
Tamang (Risiangku) &\textbf{³ʈi} &skin; skin, hide; leather &\mbox{MM-K78}:42; \mbox{MM-TamRisQ}:8.2, 8.2.6; \mbox{MM-Thesis}:408 &\hspace*{1ex}\\
Tamang (Sahu) &\textbf{Tih} &skin &\mbox{JAM-Ety} &\hspace*{1ex}\\
 &\textbf{³ʈi} &skin &\mbox{MM-K78}:42 &\hspace*{1ex}\\
 &\textbf{³ʈi} = \textbf{³ʈi} &skin &\mbox{MM-Thesis}:408 &\hspace*{1ex}\\
Tamang (Taglung) &\textbf{³ʈi} &skin &\mbox{MM-K78}:42 &\hspace*{1ex}\\
 &\textbf{ʈi} &skin &\mbox{MM-Thesis}:408 &\hspace*{1ex}\\
Thakali (Marpha) &\textbf{³ʈi} &skin &\mbox{MM-K78}:42 &\hspace*{1ex}\\
 &\textbf{³ʈiʱ} &skin &\mbox{MM-Thesis}:408 &\hspace*{1ex}\\
Thakali (Syang) &\textbf{²¹ɖi} = \textbf{²¹ɖiʱ} &skin &\mbox{MM-Thesis}:408 &\hspace*{1ex}\\
 &\textbf{³ʈi} &skin &\mbox{MM-K78}:42 &\hspace*{1ex}\\
 &\textbf{ᴸʈi} = \textbf{Xʈiʱ} &skin &\mbox{MM-Thesis}:408 &\hspace*{1ex}\\
Thakali (Tukche) &\textbf{Tih} &skin &\mbox{JAM-Ety} &\hspace*{1ex}\\
 &\textbf{³ʈi} &skin &\mbox{MM-K78}:42 &\hspace*{1ex}\\
 &\textbf{³ʈi} = ³ʈi = ʈih &skin (n.) &\mbox{MM-Thesis}:408 &\hspace*{1ex}\\
 &\textbf{ʈih} &skin &\mbox{SIL-Thak}:1.28 &\hspace*{1ex}\\
 &\textbf{ʈih} 'pli-lɔ &skin &\mbox{SIL-Thak}:3.B.31 &\hspace*{1ex}{\tiny \textasciitilde,792,595}\\
\end{longtable}
2.2. Newar\\
\fascicletablebegin
Newar (Dolakhali) &chyɔu\,\textbf{ri} &skin / bark &\mbox{CG-Dolak} &\hspace*{1ex}{\tiny 585,\textasciitilde}\\
\end{longtable}
2.3.1. Kham-Magar-Chepang-Sunwar\\
\fascicletablebegin
Sunwar &sũy\,\textbf{gi} &lip &\mbox{JAM-Ety} &\hspace*{1ex}{\tiny 477,\textasciitilde}\\
\end{longtable}
3.1. Tangut\\
\fascicletablebegin
Tangut [Xixia] &\textbf{ndži} (?) &skin &\mbox{NT-SGK}:297-055 &\hspace*{1ex}\\
 &n\,\textbf{dzɪ¹} &skin &\mbox{MVS-Grin} &\hspace*{1ex}{\tiny p,\textasciitilde}\\
 &n\,\textbf{dɪ̯wạ¹} &skin &\mbox{MVS-Grin} &\hspace*{1ex}{\tiny p,\textasciitilde}\\
 &ʔu \textbf{ndzɪ} &scalp &\mbox{DQ-Xixia}:2.8 &\hspace*{1ex}{\tiny 385,\textasciitilde}\\
\end{longtable}
3.2. Qiangic\\
\fascicletablebegin
Ersu &\textbf{ndʐo⁵⁵}pi⁵⁵ &skin &\mbox{ZMYYC}:266.18 &\hspace*{1ex}{\tiny \textasciitilde,590}\\
Guiqiong &\textbf{ɕi⁵³} &skin &\mbox{SHK-GuiqQ}; \mbox{ZMYYC}:266.17 &\hspace*{1ex}\\
Lyuzu &\textbf{n-gəʴ³⁵} &skin &\mbox{TBL}:0265.18 &\hspace*{1ex}\\
 &\textbf{ngaʴ³³}pi⁵³ &skin &\mbox{TBL}:0120.18 &\hspace*{1ex}{\tiny \textasciitilde,590}\\
Muya &\textbf{rə³³}mbɐ⁵³ &skin &\mbox{TBL}:0120.15 &\hspace*{1ex}{\tiny \textasciitilde,m}\\
Muya [Minyak] &\textbf{ʐɯ³⁵}mbɐ⁵³ &skin &\mbox{SHK-MuyaQ}:8.2; \mbox{ZMYYC}:266.15 &\hspace*{1ex}{\tiny \textasciitilde,592}\\
Namuyi &əʴ⁵⁵\textbf{ʂɿ³³} &skin &\mbox{SHK-NamuQ}:8.2; \mbox{ZMYYC}:266.19 &\hspace*{1ex}{\tiny m,\textasciitilde}\\
Pumi (Jiulong) &\textbf{ʐɿ³⁵} &skin &\mbox{TBL}:0265.10 &\hspace*{1ex}\\
Pumi (Lanping) &\textbf{ʐɤ¹³} &skin &\mbox{TBL}:0265.09 &\hspace*{1ex}\\
Pumi (Qinghua) &\textbf{ʐɤ¹³} &skin &\mbox{JZ-Pumi}; \mbox{ZMYYC}:266.11 &\hspace*{1ex}\\
Pumi (Taoba) &\textbf{rə⁵³} &skin &\mbox{JZ-Pumi}; \mbox{ZMYYC}:266.10 &\hspace*{1ex}\\
Qiang (Mawo) &nə \textbf{ɹə} pi &skin &\mbox{ZMYYC}:266.8; \mbox{JZ-Qiang} &\hspace*{1ex}{\tiny m,\textasciitilde,590}\\
 &qə pa\,tʂ \textbf{ɹa}\,bi &scalp &\mbox{SHK-MawoQ}:2.8 &\hspace*{1ex}{\tiny 1224,m,m,\textasciitilde,590}\\
 &\textbf{ɹa}\,pi &skin &\mbox{SHK-MawoQ}:8.2 &\hspace*{1ex}{\tiny \textasciitilde,590}\\
 &\textbf{ɹə}\,pi &skin &\mbox{JS-Mawo} &\hspace*{1ex}{\tiny \textasciitilde,590}\\
 &\textbf{ɹɛ} piɛ &skin &\mbox{TBL}:0265.08 &\hspace*{1ex}{\tiny \textasciitilde,590}\\
 &\textbf{ɹɛ} qɛ &skin (cattle) &\mbox{TBL}:1144.08 &\hspace*{1ex}{\tiny \textasciitilde,m}\\
Qiang (Taoping) &\textbf{tʃhɿ³¹}pa³³ &skin &\mbox{ZMYYC}:266.9 &\hspace*{1ex}{\tiny \textasciitilde,s}\\
 &\textbf{tʃʰɿ³¹} pa³³ &skin &\mbox{JZ-Qiang} &\hspace*{1ex}{\tiny \textasciitilde,s}\\
Qiang (Yadu) &\textbf{ʐe} pi &skin (animal) &\mbox{DQ-QiangN}:331 &\hspace*{1ex}{\tiny \textasciitilde,590}\\
Queyu (Yajiang) [Zhaba] &\textbf{ri³¹} &skin &\mbox{SHK-ZhabQ}:8.2; \mbox{ZMYYC}:266.16 &\hspace*{1ex}\\
Queyu (Xinlong) &ri¹³\textbf{riɛ³³} &skin &\mbox{TBL}:0265.13 &\hspace*{1ex}{\tiny m,\textasciitilde}\\
Shixing &hĩ⁵³qua³³\textbf{tʂɿ³³} &skin &\mbox{TBL}:0120.17 &\hspace*{1ex}{\tiny m,m,\textasciitilde}\\
 &\textbf{re⁵⁵} &skin &\mbox{TBL}:0265.17 &\hspace*{1ex}\\
 &\textbf{ɣɐ³⁵} &skin &\mbox{ZMYYC}:266.20 &\hspace*{1ex}\\
 &ȵi⁵⁵ ɕɑ̃³³ \textbf{dʑi³⁵} &lip &\mbox{SHK-ShixQ} &\raisebox{-0.5ex}{\footnotemark}\footnotetext{1st two morphemes = ‘mouth’}
{\tiny 467,2097,\textasciitilde}\\
 &βɛ³³ \textbf{rɿ⁵⁵} &hide / leather &\mbox{SHK-ShixQ} &\hspace*{1ex}{\tiny m,\textasciitilde}\\
Ersu &\textbf{ndʐo⁵⁵} pi⁵⁵ &skin &\mbox{SHK-ErsCQ} &\hspace*{1ex}{\tiny \textasciitilde,590}\\
 &sɿ⁵⁵ mphɑ⁵⁵ \textbf{ndʐo⁵⁵} pi⁵⁵ &lip &\mbox{SHK-ErsCQ} &\raisebox{-0.5ex}{\footnotemark}\footnotetext{\textbf{sɿ⁵⁵ mphɑ⁵⁵} ‘mouth’ + \textbf{ndʐo⁵⁵ pi⁵⁵} ‘skin’.}
{\tiny 686,2098,\textasciitilde,590}\\
\end{longtable}
3.3. rGyalrongic\\
\fascicletablebegin
Ergong (Northern) &\textbf{dʒə³³} dʒua³³ &skin &\mbox{SHK-ErgNQ}:8.2 &\hspace*{1ex}{\tiny \textasciitilde,448}\\
Ergong (Danba) &\textbf{dʑi} dʑa &skin &\mbox{SHK-ErgDQ}:8.2 &\hspace*{1ex}{\tiny \textasciitilde,448}\\
Daofu &\textbf{dʑi} dʑa &skin &\mbox{TBL}:0120.12 &\hspace*{1ex}{\tiny \textasciitilde,m}\\
Ergong (Danba) &\textbf{dʑi} dʑa &skin &\mbox{ZMYYC}:266.14 &\hspace*{1ex}{\tiny \textasciitilde,m}\\
Daofu &\textbf{ɣɿə} &skin &\mbox{TBL}:0265.12 &\hspace*{1ex}\\
Ergong (Northern) &ʁo⁵³ \textbf{dʑi³³} dʑua³³ &scalp &\mbox{SHK-ErgNQ}:2.8 &\hspace*{1ex}{\tiny 386,\textasciitilde,448}\\
rGyalrong (Eastern) &mɲɐ \textbf{ndʐi} r̥kʰo &eyelid &\mbox{SHK-rGEQ}:3.4.1 &\hspace*{1ex}{\tiny 681,\textasciitilde,589}\\
rGyalrong &ta ko n\,\textbf{dɕi} &scalp &\mbox{DQ-Jiarong}:2.8 &\hspace*{1ex}{\tiny p,386,p,\textasciitilde}\\
rGyalrong (Eastern) &ta\,ko n\,\textbf{dʐi} &scalp &\mbox{SHK-rGEQ}:2.8 &\hspace*{1ex}{\tiny p,386,p,\textasciitilde}\\
rGyalrong (Maerkang) &tə \textbf{ndʐi} &skin &\mbox{TBL}:0265.11 &\hspace*{1ex}{\tiny p,\textasciitilde}\\
rGyalrong &tə \textbf{ndʐi} &skin &\mbox{ZMYYC}:266.12 &\hspace*{1ex}{\tiny p,\textasciitilde}\\
 &tə n\,\textbf{dʑi} &hide; leather; skin &\mbox{DQ-Jiarong}:8.2,8.2.6 &\hspace*{1ex}{\tiny p,p,\textasciitilde}\\
Caodeng &tə́-\textbf{ndʒi} &skin &\mbox{JS-Caodeng} &\hspace*{1ex}{\tiny p,\textasciitilde}\\
rGyalrong (Northern) &tə\,ku n\,\textbf{dʐi} &scalp &\mbox{SHK-rGNQ}:2.8 &\hspace*{1ex}{\tiny p,386,p,\textasciitilde}\\
rGyalrong (NW) &tə\,ku n\,\textbf{dʐə} &scalp &\mbox{SHK-rGNWQ}:2.8 &\hspace*{1ex}{\tiny p,386,p,\textasciitilde}\\
rGyalrong (Northern) &tə\,mtɕʰi \textbf{ndʐi} &lip &\mbox{SHK-rGNQ}:3.9 &\hspace*{1ex}{\tiny p,442,\textasciitilde}\\
rGyalrong (NW) &tə\,mȵaŋ \textbf{ndʒə} &eyelid &\mbox{SHK-rGNWQ}:3.4.1 &\hspace*{1ex}{\tiny p,681,\textasciitilde}\\
 &tə\,n\,\textbf{dʒə} &hide  /  leather (dried animal skin) &\mbox{SHK-rGNWQ}:8.2.6 &\hspace*{1ex}{\tiny p,p,\textasciitilde}\\
rGyalrong (Eastern) &tə\,n\,\textbf{dʐi} &hide  /  leather (dried animal skin) &\mbox{SHK-rGEQ}:8.2.6 &\hspace*{1ex}{\tiny p,p,\textasciitilde}\\
rGyalrong (Northern) &tə\,n\,\textbf{dʐi} &hide; leather (dried animal skin); skin &\mbox{SHK-rGNQ}:8.2,8.2.6 &\hspace*{1ex}{\tiny p,p,\textasciitilde}\\
rGyalrong (Eastern) &tə\,ʃno \textbf{ndʐi} &lip &\mbox{SHK-rGEQ}:3.9 &\hspace*{1ex}{\tiny p,441,\textasciitilde}\\
rGyalrong (NW) &tə\,ɕe n\,\textbf{dʒə} &skin &\mbox{SHK-rGNWQ}:8.2 &\hspace*{1ex}{\tiny p,m,p,\textasciitilde}\\
rGyalrong (Eastern) &wu\,ʃan\,\textbf{dʐi} &skin &\mbox{SHK-rGEQ}:8.2 &\hspace*{1ex}{\tiny 591,m,\textasciitilde}\\
rGyalrong (NW) &ʁa\,ju n\,\textbf{dʒə} &scales (of fish) &\mbox{SHK-rGNWQ}:8.2.5 &\hspace*{1ex}{\tiny m,m,p,\textasciitilde}\\
\end{longtable}
6.1. Burmish\\
\fascicletablebegin
Achang (Lianghe) &nut⁵⁵ \textbf{ɯ⁵⁵} &lip &\mbox{JZ-Achang} &\hspace*{1ex}{\tiny 471,\textasciitilde}\\
 &ɑ³¹ \textbf{ɯ⁵⁵} &skin &\mbox{JZ-Achang} &\hspace*{1ex}{\tiny p,\textasciitilde}\\
Achang (Longchuan) &a³¹ \textbf{ʐɿ⁵⁵} &skin &\mbox{JZ-Achang}; \mbox{TBL}:0265.28; \mbox{ZMYYC}:266.41 &\hspace*{1ex}{\tiny p,\textasciitilde}\\
Achang (Luxi) &a³¹ \textbf{ɯ³¹} &skin &\mbox{JZ-Achang} &\hspace*{1ex}{\tiny p,\textasciitilde}\\
Achang (Xiandao) &a³¹ \textbf{ʐʅ⁵⁵} &skin; skin (animal) &\mbox{DQ-Xiandao}:147,315 &\hspace*{1ex}{\tiny p,\textasciitilde}\\
 &a³¹\textbf{ʐɿ⁵⁵} &skin &\mbox{TBL}:0265.29 &\hspace*{1ex}{\tiny p,\textasciitilde}\\
 &n̥ut⁵⁵ \textbf{ʐʅ⁵⁵} &lip &\mbox{DQ-Xiandao}:108 &\hspace*{1ex}{\tiny 471,\textasciitilde}\\
Arakanese &\textbf{-krì} &scale &\mbox{JO-PB} &\hspace*{1ex}\\
Bola &ŋə̆³¹ \textbf{kjɛ̱ʔ⁵⁵} &scales (fish) &\mbox{DQ-Bola}:409 &\raisebox{-0.5ex}{\footnotemark}\footnotetext{First syllable means ‘fish’.}
{\tiny 1455,\textasciitilde}\\
Burmese (Modern) &a\,\textbf{re³} &skin &\mbox{GHL-PPB}:S.16 &\hspace*{1ex}{\tiny p,\textasciitilde}\\
Burmese (Rangoon) &ɑ⁵³\textbf{je²²} &skin &\mbox{TBL}:0265.27 &\hspace*{1ex}{\tiny p,\textasciitilde}\\
 &ɑ⁵³tθɑ⁵⁵ɑ⁵³\textbf{je²²} &skin &\mbox{TBL}:0120.27 &\hspace*{1ex}{\tiny p,m,p,\textasciitilde}\\
Burmese (Standard Spoken) &\textbf{-ceì} &scale &\mbox{JO-PB} &\hspace*{1ex}\\
Burmese (Spoken) &ă\,\textbf{ye³} &skin &\mbox{GHL-PPB}:S.16 &\hspace*{1ex}{\tiny p,\textasciitilde}\\
Burmese (Spoken Rangoon) &ɑ tθɑ⁵⁵ɑ je²² &skin &\mbox{ZMYYC}:266.40 &\hspace*{1ex}{\tiny p,m,p,\textasciitilde}\\
Burmese (Written) &(pûn-)liŋ-\textbf{ʔəre}-prâ &foreskin &\mbox{JAM-Ety} &\raisebox{-0.5ex}{\footnotemark}\footnotetext{First syllable means ‘hide (v.)’, i.e.\ “hidden skin”. Second syllable is a borrowing from Sanskrit \textbf{lingam} (private parts, penis).}
{\tiny m,b,\textasciitilde,792}\\
 &\textbf{-kre:} &scale &\mbox{JO-PB} &\hspace*{1ex}\\
 &a\,\textbf{re} &skin &\mbox{GEM-CNL} &\hspace*{1ex}{\tiny p,\textasciitilde}\\
 &a\,\textbf{riy} &skin &\mbox{GHL-PPB}:S.16 &\hspace*{1ex}{\tiny p,\textasciitilde}\\
 &ə-\textbf{krê} &scales of fish &\mbox{PKB-WBRD} &\hspace*{1ex}{\tiny p,\textasciitilde}\\
 &ə-\textbf{re} &skin ( cf. ə-krê ) &\mbox{PKB-WBRD} &\hspace*{1ex}{\tiny p,\textasciitilde}\\
 &ɑ¹\textbf{re²} &skin &\mbox{TBL}:0265.26 &\hspace*{1ex}{\tiny p,\textasciitilde}\\
 &ɑ¹thɑ³ɑ¹\textbf{re²} &skin &\mbox{ZMYYC}:266.39 &\hspace*{1ex}{\tiny p,m,p,\textasciitilde}\\
 &ɑ¹tθɑ³ɑ¹\textbf{re²} &skin &\mbox{TBL}:0120.26 &\hspace*{1ex}{\tiny p,m,p,\textasciitilde}\\
 &ʔû-khôŋ-\textbf{re} &scalp &\mbox{JAM-Ety} &\hspace*{1ex}{\tiny 385,387,\textasciitilde}\\
 &ʔə-\textbf{krê} &scales (of fish) &\mbox{JAM-Ety} &\hspace*{1ex}{\tiny p,\textasciitilde}\\
 &ʔə-paw-sâ-\textbf{re} &epidermis, cuticle, scarfskin &\mbox{JAM-Ety} &\hspace*{1ex}{\tiny p,m,m,\textasciitilde}\\
 &ʔə-\textbf{re} &skin &\mbox{JAM-Ety} &\hspace*{1ex}{\tiny p,\textasciitilde}\\
 &ʔə-\textbf{re}-prâ &skin &\mbox{JAM-Ety} &\hspace*{1ex}{\tiny p,\textasciitilde,792}\\
Danu &ă\,\textbf{ye⁴} &skin &\mbox{GHL-PPB}:S.16 &\hspace*{1ex}{\tiny p,\textasciitilde}\\
Hpun (Metjo) &ă\,\textbf{ʀih¹} &skin &\mbox{GHL-PPB}:S.16 &\hspace*{1ex}{\tiny p,\textasciitilde}\\
Hpun (Northern) &ă\,\textbf{ʀìʼ}, ă\,\textbf{ʀíʼ} &skin &\mbox{EJAH-Hpun} &\hspace*{1ex}{\tiny p,\textasciitilde,p,\textasciitilde}\\
Intha &\textbf{-cì} &scale &\mbox{JO-PB} &\hspace*{1ex}\\
Lashi (Lachhe') &\textbf{yeɪ⁴} &skin &\mbox{GHL-PPB}:S.16 &\hspace*{1ex}\\
Maru [Langsu] &ŋə̆³¹ \textbf{kjɛ̱ʔ⁵⁵} &scales (of fish) &\mbox{DQ-Langsu}:8.2.5 &\hspace*{1ex}{\tiny 1455,\textasciitilde}\\
Taung-Yo &ă\,\textbf{ye⁴} &skin &\mbox{GHL-PPB}:S.16 &\hspace*{1ex}{\tiny p,\textasciitilde}\\
Tavoyan &\textbf{-ceì} &scale &\mbox{JO-PB} &\hspace*{1ex}\\
Atsi [Zaiwa] &ʃɔ⁴\textbf{we²} &skin &\mbox{GHL-PPB}:S.16 &\hspace*{1ex}{\tiny 34,\textasciitilde}\\
\end{longtable}
6.2. Loloish\\
\fascicletablebegin
{}*Loloish &*\textbf{re¹} &skin &\mbox{DB-PLolo}:134 &\hspace*{1ex}\\
Ahi &i³³ \textbf{tɕi²²} &skin &\mbox{LMZ-AhiQ}:8.2 &\hspace*{1ex}{\tiny m,\textasciitilde}\\
 &ne̱³³ \textbf{tɕi²²} &eyelid &\mbox{LMZ-AhiQ}:3.4.1 &\hspace*{1ex}{\tiny 681,\textasciitilde}\\
 &ni²¹ \textbf{tɕi²²} &lip &\mbox{LMZ-AhiQ}:3.9 &\hspace*{1ex}{\tiny 467,\textasciitilde}\\
 &ni²¹ \textbf{tɕi²²} i³³ tʻu³³⁀ɛ³³ tʻi²¹ ʈo⁵⁵ &lower lip &\mbox{LMZ-AhiQ}:3.9.2 &\hspace*{1ex}{\tiny 467,\textasciitilde,m,280,m}\\
 &ni²¹ \textbf{tɕi²²} o⁵⁵ kʻɑ³³ tʻi²¹ ʈo⁵⁵ &upper lip &\mbox{LMZ-AhiQ}:3.9.1 &\hspace*{1ex}{\tiny 467,\textasciitilde,m,m,m}\\
 &o⁵⁵ \textbf{tɕi²²} &scalp &\mbox{LMZ-AhiQ}:2.8 &\raisebox{-0.5ex}{\footnotemark}\footnotetext{Lit. “head-skin”.}
{\tiny 385,\textasciitilde}\\
Lolo (Ni) &\textbf{tchè} &skin &\mbox{GHL-PPB}:S.16 &\hspace*{1ex}\\
Ahi &\textbf{tɕi²} &skin &\mbox{CK-CS}:48. &\hspace*{1ex}\\
 &xo²¹\textbf{tɕi²²} &skin &\mbox{CK-YiQ}:8.2 &\hspace*{1ex}{\tiny m,\textasciitilde}\\
Hani (Dazhai) &sa³¹ \textbf{gɯ⁵⁵} &skin &\mbox{JZ-Hani}; \mbox{ZMYYC}:266.31 &\hspace*{1ex}{\tiny 34,\textasciitilde}\\
Hani (Lüchun) &sa³¹\textbf{gɯ⁵⁵} &skin &\mbox{TBL}:0265.41 &\hspace*{1ex}{\tiny 34,\textasciitilde}\\
Hani (Caiyuan) &ɔ³¹ \textbf{tɕi⁵⁵} &skin &\mbox{JZ-Hani} &\hspace*{1ex}{\tiny p,\textasciitilde}\\
 &ɔ³¹\textbf{tsi⁵⁵} &skin &\mbox{ZMYYC}:266.30 &\hspace*{1ex}{\tiny p,\textasciitilde}\\
Hani (Shuikui) &ʃɔ³¹ \textbf{tsʰɿ⁵⁵} &skin &\mbox{JZ-Hani} &\hspace*{1ex}{\tiny 586,\textasciitilde}\\
 &ʃɔ³¹\textbf{tshɯ⁵⁵} &skin &\mbox{ZMYYC}:266.32 &\hspace*{1ex}{\tiny 34,\textasciitilde}\\
Hani (Mojiang) &ʃɔ³¹\textbf{kɯ⁵⁵} &skin &\mbox{TBL}:0265.42 &\hspace*{1ex}{\tiny 34,\textasciitilde}\\
 &ʃɔ³¹\textbf{tshl⁵⁵} &skin &\mbox{TBL}:0120.42 &\hspace*{1ex}{\tiny 34,\textasciitilde}\\
Lahu (Bakeo) &aw\STEDTU{˯}\,\textbf{geu\STEDTU{˯}}\,k'uˉ &skin &\mbox{DB-Lahu}:134 &\hspace*{1ex}{\tiny p,\textasciitilde,586}\\
Lahu (Banlan) &aw\STEDTU{˯}\,\textbf{gi\STEDTU{˯}}\,kuˉ &skin &\mbox{DB-Lahu}:134 &\hspace*{1ex}{\tiny p,\textasciitilde,586}\\
Lahu (Nyi) &aw\STEDTU{˯}\,\textbf{gui\STEDTU{˯}}\,k'uˉ &skin &\mbox{DB-Lahu}:134 &\hspace*{1ex}{\tiny p,\textasciitilde,586}\\
{}*Common Lahu &*\textbf{gui\STEDTU{˯}} &skin &\mbox{DB-PLolo}:134 &\hspace*{1ex}\\
Lahu (Lancang) &ɔ³¹\textbf{gɯ³¹} &skin &\mbox{TBL}:0265.43 &\hspace*{1ex}{\tiny p,\textasciitilde}\\
Lahu (Black) &aw\STEDTU{˯} \textbf{gui\STEDTU{˯}} &skin &\mbox{GHL-PPB}:S.16 &\hspace*{1ex}{\tiny p,\textasciitilde}\\
 &mə̂-\textbf{gɨ̀} &lip &\mbox{JAM-Ety} &\hspace*{1ex}{\tiny 467,\textasciitilde}\\
 &mɛ̂ʔ-\textbf{gɨ̀} &eyelid &\mbox{JAM-Ety} &\hspace*{1ex}{\tiny 681,\textasciitilde}\\
 &mɯ⁵³ \textbf{gɯ³¹} &lip &\mbox{JZ-Lahu} &\hspace*{1ex}{\tiny 467,\textasciitilde}\\
 &ɔ³¹ \textbf{gɯ³¹} &skin; hide &\mbox{JZ-Lahu}; \mbox{ZMYYC}:266.33 &\hspace*{1ex}{\tiny p,\textasciitilde}\\
 &ɔ̀-\textbf{gɨ̀(}qú) &skin &\mbox{JAM-Ety} &\hspace*{1ex}{\tiny p,\textasciitilde,586}\\
 &ɔ̀-\textbf{gɨ̀}mə̂ &foreskin &\mbox{JAM-Ety} &\raisebox{-0.5ex}{\footnotemark}\footnotetext{Last syllable means “tip”.}
{\tiny p,\textasciitilde,m}\\
Lahu (Yellow) &mɯ⁵⁵ \textbf{gɯ³¹} &lip &\mbox{JZ-Lahu} &\hspace*{1ex}{\tiny 467,\textasciitilde}\\
 &ɔ³¹ \textbf{gɯ³¹} &skin; hide &\mbox{JZ-Lahu} &\hspace*{1ex}{\tiny p,\textasciitilde}\\
Lalo &\textbf{gɯ⁵⁵} &skin &\mbox{CK-CS}:48.; \mbox{CK-YiQ}:8.2 &\hspace*{1ex}\\
Lipho &\textbf{dzi³³} &skin &\mbox{CK-CS}:48. &\hspace*{1ex}\\
 &\textbf{dzi³³}tʂho³³ &skin &\mbox{CK-YiQ}:8.2 &\hspace*{1ex}{\tiny \textasciitilde,790}\\
Lisu (Northern) &hwa²¹\textbf{dʑi⁴⁴} &fur; leather; skin of an animal &\mbox{DB-Lisu} &\hspace*{1ex}{\tiny m,\textasciitilde}\\
 &hwa²¹kɔ³⁵\textbf{dʑi⁴⁴} &skin of an animal &\mbox{DB-Lisu} &\hspace*{1ex}{\tiny m,m,\textasciitilde}\\
Lisu &hwa⁵-\textbf{ji⁴} &skin &\mbox{JAM-Ety} &\hspace*{1ex}{\tiny 589,\textasciitilde}\\
Lisu (Central) &hwa⁵-\textbf{ji⁴} &skin &\mbox{JF-HLL} &\hspace*{1ex}{\tiny 589,\textasciitilde}\\
Lisu (Northern) &hɔ²¹bi⁴⁴\textbf{dʑi⁴⁴} &skin of flying squirrel &\mbox{DB-Lisu} &\hspace*{1ex}{\tiny m,m,\textasciitilde}\\
Lisu &\textbf{ji⁴} &skin &\mbox{DB-PLolo}:134 &\hspace*{1ex}\\
Lisu (Theng-yüeh) &\textbf{ji⁴} &skin &\mbox{GHL-PPB}:S.16 &\hspace*{1ex}\\
Lisu &\textbf{ji⁴} &skin &\mbox{JAM-Ety} &\hspace*{1ex}\\
Lisu (Central) &\textbf{ji⁴} &skin; hide (skin of animal) &\mbox{JF-HLL} &\hspace*{1ex}\\
Lisu (Northern) &ji⁵⁵kɔ⁵⁵\textbf{dʑi³³} &peel; skin &\mbox{DB-Lisu} &\hspace*{1ex}{\tiny m,m,\textasciitilde}\\
 &ji⁵⁵kɔ⁵⁵\textbf{tɕhɤ³⁵} &skin; scab; shell; scale; tortoise shell &\mbox{DB-Lisu} &\hspace*{1ex}{\tiny m,m,\textasciitilde}\\
Lisu &kaw³-\textbf{ji⁴} &skin &\mbox{JAM-Ety} &\hspace*{1ex}{\tiny 589,\textasciitilde}\\
Lisu (Central) &kaw³-\textbf{ji⁴} &skin &\mbox{JF-HLL} &\hspace*{1ex}{\tiny m,\textasciitilde}\\
Lisu &kaw³\textbf{ji⁴} &lip &\mbox{DB-PLolo}:95 &\hspace*{1ex}{\tiny 586,\textasciitilde}\\
Lisu (Nujiang) &ko³⁵ \textbf{dʒi³³} &skin &\mbox{JZ-Lisu} &\hspace*{1ex}{\tiny 589,\textasciitilde}\\
Lisu &ko³⁵\textbf{dʒi³³} &skin &\mbox{ZMYYC}:266.27 &\hspace*{1ex}{\tiny 589,\textasciitilde}\\
 &ko³⁵\textbf{dʑi³³} &skin &\mbox{TBL}:0120.40 &\hspace*{1ex}{\tiny 589,\textasciitilde}\\
Lisu (Northern) &kɔ³⁵\textbf{dʑi⁴⁴} &skin &\mbox{DB-Lisu} &\hspace*{1ex}{\tiny m,\textasciitilde}\\
 &la²¹ma³³\textbf{dʑi⁴⁴} &tiger skin &\mbox{DB-Lisu} &\hspace*{1ex}{\tiny 2389,m,\textasciitilde}\\
 &la³³\textbf{dʑi⁴⁴} &skin of river deer &\mbox{DB-Lisu} &\hspace*{1ex}{\tiny 2389,\textasciitilde}\\
Lisu &mrgh⁵-lrghe² kaw³\textbf{ji⁴} &lip &\mbox{JAM-Ety} &\hspace*{1ex}{\tiny 467,456,586,\textasciitilde}\\
 &myá³ kaw³-\textbf{ji⁴} &eyelid &\mbox{JAM-Ety} &\hspace*{1ex}{\tiny 681,m,\textasciitilde}\\
Lisu (Nujiang) &mɯ³¹ lɯ³⁵ ku³⁵ \textbf{dʒi³³} &lip &\mbox{JZ-Lisu} &\hspace*{1ex}{\tiny 467,456,586,\textasciitilde}\\
Lisu (Northern) &tʃhɿʔ²¹na⁵⁵\textbf{dʑi⁴⁴} &skin of wether &\mbox{DB-Lisu} &\hspace*{1ex}{\tiny m,m,\textasciitilde}\\
Lisu (Putao) &yi¹\textbf{dje⁴} &skin &\mbox{GHL-PPB}:S.16 &\hspace*{1ex}{\tiny m,\textasciitilde}\\
Lisu (Northern) &ɣɔ⁴⁴ge³³\textbf{dʑi³³} &skin of lesser panda &\mbox{DB-Lisu} &\hspace*{1ex}{\tiny m,m,\textasciitilde}\\
Lolopho &\textbf{dʑi³³} &skin &\mbox{DQ-Lolopho}:8.2 &\hspace*{1ex}\\
Nasu &\textbf{ndʑi²¹} &skin &\mbox{CK-CS}:48. &\hspace*{1ex}\\
 &n\,\textbf{dʑʰi²¹} &hide; leather; skin &\mbox{CK-YiQ}:8.2,8.2.6 &\hspace*{1ex}{\tiny p,\textasciitilde}\\
Neisu &\textbf{ndʑi²¹} &skin &\mbox{CK-CS}:48. &\hspace*{1ex}\\
Nesu &\textbf{dzɿ⁵⁵} &skin &\mbox{CK-YiQ}:8.2 &\hspace*{1ex}\\
 &\textbf{dʑi²¹} &skin &\mbox{CK-CS}:48. &\hspace*{1ex}\\
Noesu &\textbf{ndʑi²¹} &skin &\mbox{CK-YiQ}:8.2 &\hspace*{1ex}\\
Nosu &\textbf{ndʑi³³} &skin &\mbox{CK-CS}:48. &\hspace*{1ex}\\
 &\textbf{ndʑɿ⁴⁴}sɯ³³ &skin &\mbox{CK-YiQ}:8.2 &\hspace*{1ex}{\tiny \textasciitilde,790}\\
Nusu (Bijiang) &khu³¹\textbf{ɹi³⁵} &skin &\mbox{ZMYYC}:266.45 &\hspace*{1ex}{\tiny 589,\textasciitilde}\\
Nusu (Central) &khu̱⁵³\textbf{ɣɹi³³} &skin &\mbox{TBL}:0265.34 &\hspace*{1ex}{\tiny 586,\textasciitilde}\\
Nusu (Southern) &kʰu³¹ \textbf{ɹi³⁵} &skin &\mbox{JZ-Nusu} &\hspace*{1ex}{\tiny 586,\textasciitilde}\\
Nusu (Central) &kʰu³¹ \textbf{ɹi³⁵} &skin &\mbox{JZ-Nusu} &\hspace*{1ex}{\tiny 586,\textasciitilde}\\
 &kʰu̱³¹ \textbf{ɣʴi³³} &skin (animal) &\mbox{DQ-NusuB}:315. &\hspace*{1ex}{\tiny 586,\textasciitilde}\\
Nusu (Central/Zhizhiluo) &kʰu̱³¹ \textbf{ɣʴi³⁵} &skin; skin (animal) &\mbox{DQ-NusuA}:147.,315. &\hspace*{1ex}{\tiny 586,\textasciitilde}\\
Nusu (Central) &kʰu̱⁵³ \textbf{ɣʴi³³} &skin &\mbox{DQ-NusuB}:147. &\hspace*{1ex}{\tiny 586,\textasciitilde}\\
Nusu (Northern) &kʰu⁵⁵ \textbf{ɹɿ³¹} &skin &\mbox{JZ-Nusu} &\hspace*{1ex}{\tiny 586,\textasciitilde}\\
Sani [Nyi] &qɤ⁵⁵ \textbf{tsɥ³³} &skin &\mbox{MXL-SaniQ}:332.3 &\hspace*{1ex}{\tiny m,\textasciitilde}\\
Yi (Sani) &qɤ⁵⁵\textbf{tsz̊³³} &skin &\mbox{TBL}:0120.39 &\hspace*{1ex}{\tiny 589,\textasciitilde}\\
 &\textbf{tsz̊³³} &skin &\mbox{TBL}:0265.39 &\hspace*{1ex}\\
Sani [Nyi] &\textbf{tsɥ³³} &skin &\mbox{MXL-SaniQ}:348.5 &\hspace*{1ex}\\
 &\textbf{tʂi³³} &skin &\mbox{CK-CS}:48.; \mbox{CK-YiQ}:8.2 &\hspace*{1ex}\\
 &ɪ²¹ pɪ³³ \textbf{qɤ⁵⁵} tʂʅ³³ &skin of belly &\mbox{WAH-Sani}:313.4 &\hspace*{1ex}{\tiny m,792,\textasciitilde,m}\\
Yi (Dafang) &mi¹³ \textbf{ndʑi²¹} &lip &\mbox{JZ-Yi} &\hspace*{1ex}{\tiny 467,\textasciitilde}\\
 &\textbf{ndʑi²¹} &skin &\mbox{ZMYYC}:266.22; \mbox{JZ-Yi} &\hspace*{1ex}\\
Yi (Mile) &xo²¹\textbf{tɕi³³} &skin &\mbox{ZMYYC}:266.25 &\hspace*{1ex}{\tiny 589,\textasciitilde}\\
Yi (Mojiang) &\textbf{dʑi⁵⁵}phi⁵⁵ &skin &\mbox{ZMYYC}:266.26 &\hspace*{1ex}{\tiny \textasciitilde,590}\\
Yi (Nanhua) &\textbf{dʑi³³} &skin &\mbox{TBL}:0265.37 &\hspace*{1ex}\\
 &xo²¹\textbf{dʑi³³} &skin &\mbox{ZMYYC}:266.24 &\hspace*{1ex}{\tiny 589,\textasciitilde}\\
Yi (Nanjian) &\textbf{gɯ⁵⁵} tʂu̱²¹ &skin &\mbox{JZ-Yi} &\hspace*{1ex}{\tiny \textasciitilde,m}\\
 &\textbf{gɯ⁵⁵}tʂu̪²¹ &skin &\mbox{ZMYYC}:266.23 &\hspace*{1ex}{\tiny \textasciitilde,790}\\
Yi (Weishan) &\textbf{gɯ⁵⁵} &skin &\mbox{TBL}:0265.36 &\hspace*{1ex}\\
 &xɑ²¹\textbf{gɯ⁵⁵} &skin &\mbox{TBL}:0120.36 &\hspace*{1ex}{\tiny 34,\textasciitilde}\\
Yi (Wuding) &\textbf{ntɕhi¹¹} &skin &\mbox{TBL}:0265.38 &\hspace*{1ex}\\
 &\textbf{ȵtɕhi¹¹} &skin &\mbox{TBL}:0120.38 &\hspace*{1ex}\\
Yi (Xide) &bu⁵⁵-\textbf{tɕɿ⁵⁵} &lip &\mbox{CSL-YIzd} &\hspace*{1ex}{\tiny 453,\textasciitilde}\\
 &mi²¹-\textbf{ndʑɿ³³} &lip &\mbox{CSL-YIzd} &\hspace*{1ex}{\tiny 467,\textasciitilde}\\
 &\textbf{ndʑɿ³³} &skin &\mbox{TBL}:0265.35 &\hspace*{1ex}\\
 &\textbf{ndʑɿ³⁴}ʂɯ³³ &skin &\mbox{TBL}:0120.35 &\hspace*{1ex}{\tiny \textasciitilde,790}\\
 &\textbf{ndʑɿ⁴⁴}ʂɯ³³ &skin &\mbox{ZMYYC}:266.21 &\hspace*{1ex}{\tiny \textasciitilde,790}\\
 &n\,\textbf{dʑɿ³⁴}-ʂɯ³³ &skin &\mbox{CSL-YIzd} &\hspace*{1ex}{\tiny p,\textasciitilde,790}\\
 &ȵɔ³³-\textbf{ndʑɿ³³} &eyelid &\mbox{CSL-YIzd} &\hspace*{1ex}{\tiny 681,\textasciitilde}\\
 &\textbf{ʐʅ²¹}-mi⁵⁵ &dirt on skin &\mbox{CSL-YIzd} &\hspace*{1ex}{\tiny \textasciitilde,3568}\\
\end{longtable}
6.3. Naxi\\
\fascicletablebegin
Naxi (Western) &\textbf{ɣɯ³³} &skin &\mbox{JZ-Naxi} &\hspace*{1ex}\\
Naxi &\textbf{ɣɯ³³} &skin &\mbox{TBL}:0265.45 &\hspace*{1ex}\\
Naxi (Lijiang) &\textbf{ɣɯ³³}phi³¹ &skin &\mbox{ZMYYC}:266.28 &\hspace*{1ex}{\tiny \textasciitilde,590}\\
Naxi (Eastern) &\textbf{ɣɯ¹³} &skin &\mbox{JZ-Naxi} &\hspace*{1ex}\\
Naxi (Yongning) &\textbf{ɣɯ¹³} &skin &\mbox{ZMYYC}:266.29 &\hspace*{1ex}\\
\end{longtable}
6.4. Jinuo\\
\fascicletablebegin
Jinuo (Baya/Banai) &ŋɔ⁴⁴ \textbf{kji⁴⁴} &scales &\mbox{DQ-JinA}:433 &\hspace*{1ex}{\tiny 1455,\textasciitilde}\\
Jinuo (Baka) &ŋ̊ɔ⁴⁴ \textbf{kji⁴⁴} &scales &\mbox{DQ-JinB}:433 &\hspace*{1ex}{\tiny 1455,\textasciitilde}\\
Jinuo (Youle) &ɑ⁴⁴ \textbf{ki⁴⁴} &scales &\mbox{JZ-Jinuo} &\hspace*{1ex}{\tiny p,\textasciitilde}\\
\end{longtable}
8. Bai\\
\fascicletablebegin
Bai (Bijiang) &\textbf{tɕui³³} qɑ̱⁴⁴ &skin &\mbox{JZ-Bai} &\hspace*{1ex}{\tiny \textasciitilde,s}\\
 &\textbf{tɕui³³}qɑ⁵⁵ &skin &\mbox{ZMYYC}:266.37 &\hspace*{1ex}{\tiny \textasciitilde,m}\\
\end{longtable}
X. Non-TB\\
\fascicletablebegin
Shili &tə-\textbf{ndʒə} &skin &\mbox{JS-Shili} &\hspace*{1ex}{\tiny p,\textasciitilde}\\
\end{longtable}
}

\vspace{1em}
\addcontentsline{toc}{section}{(167)  \textbf{*s/r-kok \STEDTU{⪤} *(r-)kwak} SKIN, BARK, RIND}
\markright{(167) [\#586]  \textbf{*s/r-kok} \STEDTU{⪤} \textbf{*(r-)kwak} SKIN, BARK, RIND}
{\large
\parindent=-1em
(167) \hspace{\stretch{1}} \textbf{*\begin{tabular}[c]{c}s\\r\\\end{tabular}-kok} \STEDTU{⪤} \textbf{*(r-)kwak}\hspace{\stretch{1}}\textbf{SKIN, BARK, RIND} \textit{\tiny[\#586]}}

See \textit{STC} \#342 \textbf{*kok} and n.\ 229. The allofam with medial labial is reconstructed on the basis of the Chourasya, Thulung, and rGyalrong forms. The \textbf{r‑} prefix is reconstructed based on the rGyalrong forms. This reflexes of this etymon vary from initial \textbf{k} to \textbf{h}.

See \textit{HPTB} \textbf{*kok}, pp. 378, 514; PLB \textbf{*ʔ-gukᴸ}, p. 378.



{\footnotesize
0. Sino-Tibetan\\
\fascicletablebegin
{}*Sino-Tibetan &*\textbf{khwak} &skin / leather &\mbox{WSC-SH}:134 &\hspace*{1ex}\\
{}*Tibeto-Burman &*\textbf{(r-)kwâk} &skin &\mbox{AW-TBT}:170 &\hspace*{1ex}\\
 &*\textbf{(r-)kwɑ̂k} &leather &\mbox{ACST}:774i &\hspace*{1ex}\\
 &*\textbf{kok} &bark; rind; skin; skin, bark, rind &\mbox{ACST}:1226a; \mbox{AW-TBT}:170; \mbox{STC}:342 &\hspace*{1ex}\\
 &*\textbf{kwâk} &skin / leather &\mbox{WSC-SH}:134 &\hspace*{1ex}\\
\end{longtable}
1.1. North Assam\\
\fascicletablebegin
Apatani &pà-pu pa-\textbf{xu} &egg shell &\mbox{JS-Tani} &\hspace*{1ex}{\tiny m,1654,m,\textasciitilde}\\
 &pa-\textbf{xu} &egg shell &\mbox{JS-Tani} &\hspace*{1ex}{\tiny m,\textasciitilde}\\
Miri, Hill &\textbf{huk}\,tu &leather coat armour &\mbox{IMS-HMLG} &\hspace*{1ex}{\tiny \textasciitilde,m}\\
\end{longtable}
1.2. Kuki-Chin\\
\fascicletablebegin
Moyon &mìk\,\textbf{kòʔ} &eyelid &\mbox{DK-Moyon}:3.4.1 &\hspace*{1ex}{\tiny 682,\textasciitilde}\\
Tiddim &\textbf{hok¹/hoʔ³} &skin (v.) &\mbox{PB-TCV} &\hspace*{1ex}\\
 &\textbf{hoːk¹} &skin (v.) &\mbox{PB-TCV} &\hspace*{1ex}\\
\end{longtable}
1.3. Naga\\
\fascicletablebegin
Tangkhul &\textbf{khək} &skin &\mbox{Bhat-TNV}:96 &\hspace*{1ex}\\
\end{longtable}
1.5. Mikir\\
\fascicletablebegin
Mikir &\textbf{òk}-rèng &hide / leather &\mbox{KHG-Mikir}:34 &\hspace*{1ex}{\tiny \textasciitilde,781}\\
\end{longtable}
1.7. Bodo-Garo = Barish\\
\fascicletablebegin
Kokborok &\textbf{kʰoʔ} &peel &\mbox{PT-Kok} &\hspace*{1ex}\\
\end{longtable}
2. Himalayish\\
\fascicletablebegin
Chourasya &\textbf{kwak} &leather &\mbox{JAM-Ety} &\hspace*{1ex}\\
 &\textbf{kwak}-te &skin &\mbox{WSC-SH}:134 &\hspace*{1ex}{\tiny \textasciitilde,s}\\
 &\textbf{kwak}-te/\textbf{kok}-te &skin &\mbox{STC}:74n229 &\hspace*{1ex}{\tiny \textasciitilde,m,\textasciitilde,m}\\
\end{longtable}
2.1.2. Bodic\\
\fascicletablebegin
Tibetan (Batang) &\textbf{xhaʔ⁵⁵} baʔ⁵⁵ &skin &\mbox{DQ-Batang}:8.2 &\hspace*{1ex}{\tiny \textasciitilde,588}\\
 &\textbf{xhaʔ⁵⁵} tsha⁵³ &blemish on skin &\mbox{DQ-Batang}:8.2.3 &\hspace*{1ex}{\tiny \textasciitilde,m}\\
Tibetan (Jirel) &\textbf{kok}tenq &skin &\mbox{JAM-Ety} &\hspace*{1ex}{\tiny \textasciitilde,785}\\
Tibetan (Written) &\textbf{kog}-pa &skin &\mbox{AW-TBT}:170 &\hspace*{1ex}{\tiny \textasciitilde,s}\\
 &phyi-\textbf{kog} &bark &\mbox{STC}:342 &\raisebox{-0.5ex}{\footnotemark}\footnotetext{First syllable means ‘outside’.}
{\tiny m,\textasciitilde}\\
 &\textbf{skog}-pa &shell / rind &\mbox{AW-TBT}:170 &\hspace*{1ex}{\tiny \textasciitilde,s}\\
 &\textbf{skog}-pa \STEDTU{⪤} \textbf{kog}-pa &skin,bark,rind &\mbox{STC}:342 &\hspace*{1ex}{\tiny \textasciitilde,s,\textasciitilde,s}\\
\end{longtable}
2.1.4. Tamangic\\
\fascicletablebegin
Kharmile &sa\,\textbf{ko} &skin &\mbox{AW-TBT}:170 &\hspace*{1ex}{\tiny 34,\textasciitilde}\\
\end{longtable}
2.3. Mahakiranti\\
\fascicletablebegin
{}*Kiranti &*\textbf{kok} &skin &\mbox{BM-PK7}:157 &\hspace*{1ex}\\
\end{longtable}
2.3.2. Kiranti\\
\fascicletablebegin
Bahing &\textbf{kheʔ}\,be\,liŋ\,ma &skin &\mbox{BM-Bah} &\hspace*{1ex}{\tiny \textasciitilde,792,3611,s}\\
 &\textbf{kok}-te &skin &\mbox{JAM-Ety}; \mbox{STC}:342 &\hspace*{1ex}{\tiny \textasciitilde,448}\\
 &\textbf{kok}\,si &skin, bark &\mbox{BM-Bah} &\hspace*{1ex}{\tiny \textasciitilde,m}\\
 &\textbf{kok}\,te &skin &\mbox{BM-PK7}:157 &\hspace*{1ex}{\tiny \textasciitilde,s}\\
 &\textbf{kɔk}\,tɛ &skin &\mbox{AW-TBT}:170 &\hspace*{1ex}{\tiny \textasciitilde,s}\\
 &siŋ-\textbf{kok}-te &bark &\mbox{JAM-Ety} &\raisebox{-0.5ex}{\footnotemark}\footnotetext{\textbf{siŋ} means ‘tree’}
{\tiny m,\textasciitilde,s}\\
Bantawa &(u-\textbf{)hok}\,wa &skin &\mbox{AW-TBT}:170 &\hspace*{1ex}{\tiny p,\textasciitilde,s}\\
 &(u-\textbf{)hoʔ}\,wa &skin &\mbox{AW-TBT}:170 &\hspace*{1ex}{\tiny p,\textasciitilde,s}\\
 &\textbf{hok}-wa &skin &\mbox{WW-Bant}:29 &\hspace*{1ex}{\tiny \textasciitilde,s}\\
 &\textbf{houʔ}-wa &skin &\mbox{AW-TBT}:170 &\hspace*{1ex}{\tiny \textasciitilde,s}\\
 &\textbf{hoʔ}-wa &skin; bark &\mbox{JAM-Ety}; \mbox{BM-PK7}:157; \mbox{NKR-Bant} &\hspace*{1ex}{\tiny \textasciitilde,s}\\
 &sa-\textbf{hok}\,wa &leather &\mbox{WW-Bant}:62 &\raisebox{-0.5ex}{\footnotemark}\footnotetext{\textbf{sa} means ‘animal’.}
{\tiny 34,\textasciitilde,s}\\
Chamling &\textbf{hu}\,lep\,pa &skin &\mbox{WW-Cham}:15 &\hspace*{1ex}{\tiny \textasciitilde,562,s}\\
Hayu &\textbf{kuk}\,di-waq\,mi &skin, bark, husk, peel &\mbox{BM-Hay}:84.198 &\hspace*{1ex}{\tiny \textasciitilde,m,m,m}\\
 &\textbf{kuk}\,tsho &skin &\mbox{BM-PK7}:157; \mbox{JAM-Ety} &\hspace*{1ex}{\tiny \textasciitilde,585}\\
Kulung &so\,\textbf{go} &skin &\mbox{AW-TBT}:170 &\hspace*{1ex}{\tiny 790,\textasciitilde}\\
 &so\,\textbf{ko}\,warə &skin &\mbox{RPHH-Kul} &\hspace*{1ex}{\tiny 790,\textasciitilde,m}\\
Limbu &\textbf{hok}-rik &skin &\mbox{JAM-Ety} &\hspace*{1ex}{\tiny \textasciitilde,782}\\
 &\textbf{hoː}\,rik &skin, bark; skin &\mbox{BM-Lim}; \mbox{BM-PK7}:157 &\hspace*{1ex}{\tiny \textasciitilde,782}\\
 &\textbf{hɔk} &skin (of a nut, inside the shell) &\mbox{BM-PK7}:157 &\hspace*{1ex}\\
 &\textbf{[hoː}\,rik\,pa] &skin &\mbox{AW-TBT}:170 &\hspace*{1ex}{\tiny \textasciitilde,782,s}\\
 &\textbf{[hoː}\,riʔ\,pa] &skin &\mbox{AW-TBT}:170 &\hspace*{1ex}{\tiny \textasciitilde,782,s}\\
 &\textbf{[hu}\,rik] &skin &\mbox{AW-TBT}:170 &\hspace*{1ex}{\tiny \textasciitilde,782}\\
 &\textbf{[hu}\,riʔ] &skin &\mbox{AW-TBT}:170 &\hspace*{1ex}{\tiny \textasciitilde,782}\\
Thulung &\textbf{kok}å\,te &skin &\mbox{JAM-Ety} &\hspace*{1ex}{\tiny \textasciitilde,p,448}\\
 &\textbf{kok}\,sa &skin on heated milk, crust on cooked food &\mbox{NJA-Thulung} &\hspace*{1ex}{\tiny \textasciitilde,m}\\
 &\textbf{kok}\,te &skin; skin, bark, peel &\mbox{BM-PK7}:157; \mbox{NJA-Thulung} &\hspace*{1ex}{\tiny \textasciitilde,s}\\
 &\textbf{kwok}-si/\textbf{kok}-si &skin &\mbox{STC}:74n229 &\hspace*{1ex}{\tiny \textasciitilde,m,\textasciitilde,m}\\
 &\textbf{kɔk}\,tɛ &skin &\mbox{AW-TBT}:170 &\hspace*{1ex}{\tiny \textasciitilde,s}\\
 &si\,\textbf{kok}\,te &lip &\mbox{NJA-Thulung} &\hspace*{1ex}{\tiny 686,\textasciitilde,448}\\
 &sī\,\textbf{ko}\,kaʔ\,te &lip &\mbox{JAM-Ety} &\hspace*{1ex}{\tiny 686,\textasciitilde,m,448}\\
\end{longtable}
3.2. Qiangic\\
\fascicletablebegin
Namuyi &mpʰ\,sɿ⁵\,⁵ɦəʴ³³\,\textbf{qu⁵⁵} &lip &\mbox{SHK-NamuQ}:3.9 &\hspace*{1ex}{\tiny 2098,686,m,\textasciitilde}\\
Qiang (Yadu) &zde\,\textbf{ku̥} &lip &\mbox{DQ-QiangN}:111 &\hspace*{1ex}{\tiny 442,\textasciitilde}\\
Shixing &\textbf{ɣɐ³⁵} &skin &\mbox{SHK-ShixQ} &\hspace*{1ex}\\
\end{longtable}
3.3. rGyalrongic\\
\fascicletablebegin
rGyalrong (Northern) &tə\,mbu \textbf{r̥qʰu} &foreskin &\mbox{SHK-rGNQ}:10.3.3 &\hspace*{1ex}{\tiny p,3425,\textasciitilde}\\
 &tə\,rme ə \textbf{r̥qʰu} &skin &\mbox{SHK-rGNQ}:8.2 &\hspace*{1ex}{\tiny m,m,m,\textasciitilde}\\
rGyalrong (Eastern) &tʃi\,wjo wu\,\textbf{r̥kʰo} &scales (of fish) &\mbox{SHK-rGEQ}:8.2.5 &\raisebox{-0.5ex}{\footnotemark}\footnotetext{\textbf{tʃiwjo} means ‘fish’.}
{\tiny m,m,589,\textasciitilde}\\
rGyalrong &wer\,\textbf{khwak} &skin (its-); its skin &\mbox{STC}:74n229; \mbox{WSC-SH}:134 &\hspace*{1ex}{\tiny m,\textasciitilde}\\
\end{longtable}
4.1. Jingpho\\
\fascicletablebegin
Jingpho &mjiʔ³¹ \textbf{ko̱³³} mun³³ &eyebrow &\mbox{JZ-Jingpo} &\raisebox{-0.5ex}{\footnotemark}\footnotetext{The first syllable of this form means ‘eye’ (<~PTB \textbf{*s‑mik \STEDTU{⪤} *s‑myak}, and the third syllable means ‘hair’ (<~PTB \textbf{*mul \STEDTU{⪤} *mil}).}
{\tiny 681,\textasciitilde,363}\\
 &myìʔ \textbf{kaw} &brow ridge; supraorbital ridge &\mbox{JAM-Ety} &\hspace*{1ex}{\tiny 682,\textasciitilde}\\
 &myìʔ \textbf{kaw}-mun &eyebrow &\mbox{JAM-Ety} &\hspace*{1ex}{\tiny 682,\textasciitilde,363}\\
\end{longtable}
6. Lolo-Burmese\\
\fascicletablebegin
{}*Lolo-Burmese &*\textbf{ʔkuk/ʔguk} &skin &\mbox{JAM-TSR}:71 &\hspace*{1ex}\\
\end{longtable}
6.1. Burmish\\
\fascicletablebegin
Achang (Lianghe) &ɑ³¹ \textbf{khok⁵⁵} &scales &\mbox{JZ-Achang} &\hspace*{1ex}{\tiny p,\textasciitilde}\\
Bola &nɔ̱t⁵⁵ \textbf{ka̱uʔ} &lip &\mbox{DQ-Bola}:108 &\hspace*{1ex}{\tiny 471,\textasciitilde}\\
 &ʃă³⁵ \textbf{ka̱uʔ⁵⁵} &skin; skin (human or animal) &\mbox{DQ-Bola}:147,315 &\hspace*{1ex}{\tiny 34,\textasciitilde}\\
Bola (Luxi) &ʃă³⁵\textbf{ka̱uʔ⁵⁵} &skin &\mbox{TBL}:0265.32 &\hspace*{1ex}{\tiny 34,\textasciitilde}\\
Burmese (Written) &\textbf{-khauk} &skin &\mbox{STC}:74n229 &\hspace*{1ex}\\
 &a\,\textbf{khauk} &skin,bark,rind &\mbox{STC}:342 &\hspace*{1ex}{\tiny p,\textasciitilde}\\
 &ə\,\textbf{khauk} &skin &\mbox{AW-TBT}:170 &\hspace*{1ex}{\tiny p,\textasciitilde}\\
Langsu (Luxi) &ʃɔ̆³⁵\textbf{ka̱uk⁵⁵} &skin &\mbox{TBL}:0265.31 &\hspace*{1ex}{\tiny 34,\textasciitilde}\\
Lashi &mjɔʔ³¹ \textbf{ku̱k⁵⁵} &eyelid &\mbox{DQ-Lashi}:3.4.1 &\hspace*{1ex}{\tiny 681,\textasciitilde}\\
 &nua̱t⁵⁵ \textbf{ku̱k⁵⁵} &lip &\mbox{DQ-Lashi}:3.9 &\hspace*{1ex}{\tiny 471,\textasciitilde}\\
 &ʃǒ⁵⁵ \textbf{ku̱k⁵⁵} &skin &\mbox{DQ-Lashi}:8.2 &\hspace*{1ex}{\tiny 34,\textasciitilde}\\
 &ʃə̌⁵⁵ \textbf{ku̱k⁵⁵} &hide  /  leather &\mbox{DQ-Lashi}:8.2.6 &\hspace*{1ex}{\tiny 34,\textasciitilde}\\
Leqi (Luxi) &sə̆⁵⁵\textbf{kuk⁵⁵} &skin &\mbox{TBL}:0265.33 &\hspace*{1ex}{\tiny 34,\textasciitilde}\\
 &ʃŏ⁵⁵\textbf{ku̱k⁵⁵} &skin &\mbox{TBL}:0120.33 &\hspace*{1ex}{\tiny 34,\textasciitilde}\\
Maru [Langsu] &na̱t⁵⁵ \textbf{ka̱uk⁵⁵} &lip &\mbox{DQ-Langsu}:3.9 &\hspace*{1ex}{\tiny 471,\textasciitilde}\\
 &ɔ³¹ na̱t⁵⁵ \textbf{ka̱uk⁵⁵} &lower lip &\mbox{DQ-Langsu}:3.9.2 &\hspace*{1ex}{\tiny p,471,\textasciitilde}\\
 &ʃɔ̆³⁵ \textbf{ka̱uk⁵⁵} &hide  /  leather &\mbox{DQ-Langsu}:8.2.6 &\hspace*{1ex}{\tiny 34,\textasciitilde}\\
Atsi [Zaiwa] &\textbf{ku̱ʔ⁵⁵} &skin &\mbox{TBL}:0265.30 &\hspace*{1ex}\\
 &nu̱t⁵⁵ \textbf{ku̱ʔ⁵⁵} &lip &\mbox{JZ-Zaiwa} &\hspace*{1ex}{\tiny 471,\textasciitilde}\\
 &ʃŏ²¹ \textbf{ku̱ʔ⁵⁵} &skin &\mbox{JZ-Zaiwa}; \mbox{TBL}:0265.30; \mbox{ZMYYC}:266.42 &\hspace*{1ex}{\tiny 34,\textasciitilde}\\
\end{longtable}
6.2. Loloish\\
\fascicletablebegin
Akha &a-\textbf{kʻo} H-HS &peel / shell &\mbox{JAM-TSR}:71(a) &\hspace*{1ex}{\tiny p,\textasciitilde}\\
 &aˇ-\textbf{k'oˆ} &skin of fruit &\mbox{JAM-Ety} &\hspace*{1ex}{\tiny p,\textasciitilde}\\
 &baq-\textbf{xoq} &skin &\mbox{JAM-TSR}:71(a) &\hspace*{1ex}{\tiny 792,\textasciitilde}\\
 &baˆ \textbf{k'oˆ} &skin,bark &\mbox{PL-AETD} &\hspace*{1ex}{\tiny m,\textasciitilde}\\
Bisu &aŋ\,\textbf{khɔ} &skin &\mbox{PB-Bisu}:15 &\hspace*{1ex}{\tiny p,\textasciitilde}\\
 &ʔaŋ \textbf{khɔ} &skin &\mbox{DB-PLolo} &\hspace*{1ex}{\tiny p,\textasciitilde}\\
Gazhuo &ji³¹ tsɿ³³ sa³¹ \textbf{khɯ⁵⁵} &scalp &\mbox{DQ-Gazhuo}:2.8 &\raisebox{-0.5ex}{\footnotemark}\footnotetext{\textbf{ji³¹ tsɿ³³} means ‘head’.}
{\tiny m,m,34,\textasciitilde}\\
 &sa²¹ \textbf{khɯ⁵⁵} &skin &\mbox{DQ-Gazhuo}:8.2 &\hspace*{1ex}{\tiny 34,\textasciitilde}\\
 &ŋa²¹ \textbf{khɯ⁵⁵} &scales (of fish) &\mbox{DQ-Gazhuo}:8.2.5 &\hspace*{1ex}{\tiny 1455,\textasciitilde}\\
Hani (Dazhai) &(mja̱³³) mja̱³³ \textbf{xo̱³³} &eyelid &\mbox{JZ-Hani} &\hspace*{1ex}{\tiny 681,681,\textasciitilde}\\
Hani (Caiyuan) &me³¹ \textbf{kʰv̩̄³³} &lip &\mbox{JZ-Hani} &\hspace*{1ex}{\tiny 467,\textasciitilde}\\
Hani (Gelanghe) &ɕa³¹ \textbf{xo³³} &skin &\mbox{JZ-Hani} &\hspace*{1ex}{\tiny 34,\textasciitilde}\\
Hani (Shuikui) &\textbf{ʃɔ³¹} tsʰɿ⁵⁵ &skin &\mbox{JZ-Hani} &\hspace*{1ex}{\tiny \textasciitilde,596}\\
Lahu (Bakeo) &aw\STEDTU{˯}\,geu\STEDTU{˯}\,\textbf{k'uˉ} &skin &\mbox{DB-Lahu}:134 &\hspace*{1ex}{\tiny p,596,\textasciitilde}\\
Lahu (Banlan) &aw\STEDTU{˯}\,gi\STEDTU{˯}\,\textbf{kuˉ} &skin &\mbox{DB-Lahu}:134 &\hspace*{1ex}{\tiny p,596,\textasciitilde}\\
Lahu (Nyi) &aw\STEDTU{˯}\,gui\STEDTU{˯}\,\textbf{k'uˉ} &skin &\mbox{DB-Lahu}:134 &\hspace*{1ex}{\tiny p,596,\textasciitilde}\\
Lahu &ɔ̀-\textbf{qú} &bark &\mbox{STC}:342 &\hspace*{1ex}{\tiny p,\textasciitilde}\\
Lahu (Black) &làʔ-šɛ̄-\textbf{qú} &fingernail &\mbox{JAM-Ety} &\hspace*{1ex}{\tiny 695,511,\textasciitilde}\\
 &mɛʔ⁵⁴ \textbf{qv̩³⁵} mv̩³³ &eyebrow &\mbox{JZ-Lahu} &\hspace*{1ex}{\tiny 681,\textasciitilde,363}\\
 &mɛ̂ʔ-\textbf{qú}-mu &eyebrow &\mbox{JAM-Ety} &\hspace*{1ex}{\tiny 681,\textasciitilde,363}\\
 &ɔ̀-gɨ̀(\textbf{qú)} &skin &\mbox{JAM-Ety} &\hspace*{1ex}{\tiny p,596,\textasciitilde}\\
Lisu &\textbf{kaw³}ji⁴ &lip &\mbox{DB-PLolo}:95 &\hspace*{1ex}{\tiny \textasciitilde,596}\\
 &mrgh⁵-lrghe² \textbf{kaw³}ji⁴ &lip &\mbox{JAM-Ety} &\hspace*{1ex}{\tiny 467,456,\textasciitilde,596}\\
Lisu (Nujiang) &mɯ³¹ lɯ³⁵ \textbf{ku³⁵} dʒi³³ &lip &\mbox{JZ-Lisu} &\hspace*{1ex}{\tiny 467,456,\textasciitilde,596}\\
Lolopho &me̱⁴⁴ \textbf{kɯ⁵⁵} phæ³³ &eyebrow &\mbox{DQ-Lolopho}:3.4.3 &\raisebox{-0.5ex}{\footnotemark}\footnotetext{See the similar Newar form \textbf{mi\,khā phu\,si}.}
{\tiny 681,\textasciitilde,m}\\
Mpi &ʔɑ²\textbf{khoʔ⁴} &skin &\mbox{DB-PLolo} &\hspace*{1ex}{\tiny p,\textasciitilde}\\
Nasu &na̱²¹ \textbf{ko̱²¹} &eyelid &\mbox{CK-YiQ}:3.4.1 &\hspace*{1ex}{\tiny 681,\textasciitilde}\\
Nesu &ȵɪ̱³³ \textbf{kɤ̱³³} &eyelid &\mbox{CK-YiQ}:3.4.1 &\hspace*{1ex}{\tiny 681,\textasciitilde}\\
Nusu (Central) &\textbf{khu̱⁵³}ɣɹi³³ &skin &\mbox{TBL}:0265.34 &\hspace*{1ex}{\tiny \textasciitilde,596}\\
Nusu (Southern) &\textbf{kʰu³¹} ɹi³⁵ &skin &\mbox{JZ-Nusu} &\hspace*{1ex}{\tiny \textasciitilde,596}\\
Nusu (Central) &\textbf{kʰu³¹} ɹi³⁵ &skin &\mbox{JZ-Nusu} &\hspace*{1ex}{\tiny \textasciitilde,596}\\
Nusu (Central/Zhizhiluo) &\textbf{kʰu̱³¹} sɿ³¹ &scales &\mbox{DQ-NusuA}:409. &\hspace*{1ex}{\tiny \textasciitilde,1454}\\
Nusu (Central) &\textbf{kʰu̱³¹} ɣʴi³³ &skin (animal) &\mbox{DQ-NusuB}:315. &\hspace*{1ex}{\tiny \textasciitilde,596}\\
Nusu (Central/Zhizhiluo) &\textbf{kʰu̱³¹} ɣʴi³⁵ &skin; skin (animal) &\mbox{DQ-NusuA}:147.,315. &\hspace*{1ex}{\tiny \textasciitilde,596}\\
Nusu (Central) &\textbf{kʰu̱⁵³} ɣʴi³³ &skin &\mbox{DQ-NusuB}:147. &\hspace*{1ex}{\tiny \textasciitilde,596}\\
Nusu (Northern) &\textbf{kʰu⁵⁵} ɹɿ³¹ &skin &\mbox{JZ-Nusu} &\hspace*{1ex}{\tiny \textasciitilde,596}\\
Sani [Nyi] &ne³³ \textbf{qɤ⁵⁵} &eyelid &\mbox{WAH-Sani}:95.1; \mbox{YHJC-Sani} &\hspace*{1ex}{\tiny 681,\textasciitilde}\\
PNL &\textbf{ʔkuk}/ʔguk &skin &\mbox{CK-CS}:48. &\hspace*{1ex}\\
Sangkong &maŋ³¹ \textbf{qho̱³³} &lip &\mbox{LYS-Sangkon} &\hspace*{1ex}{\tiny 467,\textasciitilde}\\
Ugong &\textbf{kó̵̩ʔ̩} &skin &\mbox{DB-Ugong}:8.2 &\hspace*{1ex}\\
Yi (Xide) &bu̱³³-\textbf{ku̱³³} &skin; scales (fish) &\mbox{CSL-YIzd} &\hspace*{1ex}{\tiny 592,\textasciitilde}\\
 &hɯ³³-bu̱³³-\textbf{ku̱³³} &scales &\mbox{CSL-YIzd} &\hspace*{1ex}{\tiny m,592,\textasciitilde}\\
\end{longtable}
6.4. Jinuo\\
\fascicletablebegin
Jinuo (Baya/Banai) &a⁴⁴ \textbf{kʰo³¹} &skin; skin (animal) &\mbox{DQ-JinA}:150,331 &\hspace*{1ex}{\tiny p,\textasciitilde}\\
Jinuo (Baka) &a⁴⁴ \textbf{kʰu³¹} &skin (animal) &\mbox{DQ-JinB}:331 &\hspace*{1ex}{\tiny p,\textasciitilde}\\
Jinuo (Buyuan) &mi⁴⁴ \textbf{kʰu⁴⁴} &lip &\mbox{JZ-Jinuo} &\hspace*{1ex}{\tiny 467,\textasciitilde}\\
Jinuo (Baya/Banai) &mø⁴⁴ \textbf{kʰo³¹} &lip &\mbox{DQ-JinA}:111 &\hspace*{1ex}{\tiny 467,\textasciitilde}\\
Jinuo (Youle) &m̥ø⁴⁴ \textbf{kʰo⁴²} &lip &\mbox{JZ-Jinuo} &\hspace*{1ex}{\tiny 467,\textasciitilde}\\
Jinuo (Baka) &m̥ø⁴⁴ \textbf{kʰu³¹} &lip &\mbox{DQ-JinB}:111 &\hspace*{1ex}{\tiny 467,\textasciitilde}\\
Jinuo (Youle) &ɑ⁴⁴ \textbf{kʰo⁴²} &skin &\mbox{JZ-Jinuo} &\hspace*{1ex}{\tiny p,\textasciitilde}\\
Jinuo (Buyuan) &ɑ⁴⁴ \textbf{kʰu⁴⁴} &skin &\mbox{JZ-Jinuo} &\hspace*{1ex}{\tiny p,\textasciitilde}\\
\end{longtable}
7. Karenic\\
\fascicletablebegin
Bwe &\textbf{-kó} &outer covering, skin, shell &\mbox{EJAH-BKD} &\hspace*{1ex}\\
 &-phe-\textbf{kó} &skin, outer covering &\mbox{EJAH-BKD} &\hspace*{1ex}{\tiny 590,\textasciitilde}\\
 &-phe-\textbf{kú} &skin, outer covering &\mbox{EJAH-BKD} &\hspace*{1ex}{\tiny 590,\textasciitilde}\\
Pa-O &thɔ́ʔ \textbf{khɔ̀k} &skin of pig, pig hide &\mbox{DBS-PaO} &\hspace*{1ex}{\tiny m,\textasciitilde}\\
Karen (Sgaw/Yue) &mɛ̱ʔ³¹ \textbf{kʰo³¹} θu⁵⁵ &eyebrow &\mbox{DQ-KarenA}:103 &\hspace*{1ex}{\tiny 682,\textasciitilde,370}\\
\end{longtable}
9. Sinitic\\
\fascicletablebegin
Chinese &\textbf{kɛk} &skin, hide, take away &\mbox{STC}:74n229 &\hspace*{1ex}\\
Chinese (Mandarin) &\textbf{ko} &skin, hide &\mbox{GSR}:931a-b &\hspace*{1ex}\\
Chinese (Middle) &\textbf{khwâk} &leather &\mbox{WSC-SH}:134 &\hspace*{1ex}\\
Chinese (Old) &\textbf{k'wâk} &leather &\mbox{JAM-TIL}:33 &\hspace*{1ex}\\
 &\textbf{khwak} &skin / leather &\mbox{WSC-SH}:134 &\hspace*{1ex}\\
 &\textbf{kɛk} &hide, skin &\mbox{JAM-TIL}:33 &\hspace*{1ex}\\
Chinese (Old/Mid) &\textbf{kɛk}/kɛk &skin, hide &\mbox{GSR}:931a-b &\hspace*{1ex}\\
 &\textbf{kʻwɑ̂k} &leather &\mbox{ACST}:774i &\hspace*{1ex}\\
\end{longtable}
}

{\large \textbf{Chinese comparandum}}

\TC{革} \textit{GSR} 931a-b OC \textbf{*kɛk} ‘leather’.

\vspace{1em}
\addcontentsline{toc}{section}{(168)  \textbf{*gul \STEDTU{⪤} *gil} SKIN}
\markright{(168) [\#593]  \textbf{*gul} \STEDTU{⪤} \textbf{*gil} SKIN}
{\large
\parindent=-1em
(168) \hspace{\stretch{1}} \textbf{*gul} \STEDTU{⪤} \textbf{*gil}\hspace{\stretch{1}}\textbf{SKIN} \textit{\tiny[\#593]}}



{\footnotesize
1.2. Kuki-Chin\\
\fascicletablebegin
Lushai [Mizo] &\textbf{kóor} &skin &\mbox{AW-TBT}:170 &\hspace*{1ex}\\
\end{longtable}
1.3. Naga\\
\fascicletablebegin
{}*Northern Naga &*C\STEDTU{Ⓥ}-\textbf{kʰuar} &skin &\mbox{WTF-PNN}:549 &\hspace*{1ex}{\tiny p,\textasciitilde}\\
Chang &\textbf{khon} &skin &\mbox{GEM-CNL} &\hspace*{1ex}\\
 &\textbf{khowun} &leather &\mbox{GEM-CNL} &\hspace*{1ex}\\
 &\textbf{kòn} &skin &\mbox{AW-TBT}:170 &\hspace*{1ex}\\
 &\textbf{kʰo an} &leather &\mbox{WTF-PNN}:549 &\hspace*{1ex}\\
 &\textbf{kʰo wun} &leather &\mbox{WTF-PNN}:549 &\hspace*{1ex}\\
 &\textbf{kʰon} &skin, bark, leather &\mbox{WTF-PNN}:549 &\hspace*{1ex}\\
 &\textbf{kʰon} sak &suffer &\mbox{WTF-PNN}:561 &\hspace*{1ex}\\
Konyak &mei \textbf{šo} &skin, leather &\mbox{WTF-PNN}:549 &\hspace*{1ex}{\tiny m,\textasciitilde}\\
 &\textbf{sho}, mei\,\textbf{sho} &skin &\mbox{GEM-CNL} &\hspace*{1ex}{\tiny \textasciitilde,m,\textasciitilde}\\
 &\textbf{sho}\,shao &hat, helmet &\mbox{GEM-CNL} &\hspace*{1ex}{\tiny \textasciitilde,386}\\
 &\textbf{šo} &skin &\mbox{WTF-PNN}:549 &\hspace*{1ex}\\
 &\textbf{šo} šau &hat &\mbox{WTF-PNN}:502 &\raisebox{-0.5ex}{\footnotemark}\footnotetext{According to French, the first syllable means ‘skin’ (lit. “head-skin”). Cf.\ PNN \textbf{*kh} > Konyak, Phom \textbf{š}.}
{\tiny \textasciitilde,386}\\
Nocte &a \textbf{kʰuon} &skin; bark &\mbox{WTF-PNN}:549,549 &\hspace*{1ex}{\tiny p,\textasciitilde}\\
 &a\,\textbf{khuon} &skin &\mbox{GEM-CNL} &\hspace*{1ex}{\tiny p,\textasciitilde}\\
 &\textbf{kʰo-wan} &skin &\mbox{WTF-PNN}:549 &\hspace*{1ex}\\
 &\textbf{¹khoʌn} &skin &\mbox{AW-TBT}:170 &\hspace*{1ex}\\
Phom &pʰai \textbf{šu} &leather &\mbox{WTF-PNN}:549 &\hspace*{1ex}{\tiny m,\textasciitilde}\\
 &\textbf{shu} &skin &\mbox{GEM-CNL} &\hspace*{1ex}\\
 &\textbf{šu} &skin &\mbox{WTF-PNN}:549 &\hspace*{1ex}\\
Tangkhul &\textbf{kor} &skin &\mbox{Bhat-TNV}:96 &\hspace*{1ex}\\
Wancho &\textbf{kan} &skin &\mbox{GEM-CNL}; \mbox{WTF-PNN}:549 &\hspace*{1ex}\\
 &\textbf{kuon} &skin &\mbox{WTF-PNN}:549 &\hspace*{1ex}\\
\end{longtable}
1.7. Bodo-Garo = Barish\\
\fascicletablebegin
{}*Barish (East) &*\textbf{khoon} &skin &\mbox{AW-TBT}:170 &\hspace*{1ex}\\
Atong &\textbf{kor} &skin &\mbox{JAM-Ety} &\hspace*{1ex}\\
Bodo &bi\,\textbf{gúr} &skin, bark &\mbox{Bhat-Boro} &\hspace*{1ex}{\tiny m,\textasciitilde}\\
 &bi\,\textbf{gur} &skin &\mbox{JAM-Ety} &\hspace*{1ex}{\tiny p,\textasciitilde}\\
 &bi\,\textbf{gúrʔ} &skin &\mbox{AW-TBT}:170 &\hspace*{1ex}{\tiny p,\textasciitilde}\\
 &\textbf{gu}\,slay &slough (snake), change color of skin &\mbox{JAM-GSTC}:069 &\raisebox{-0.5ex}{\footnotemark}\footnotetext{The second syllable of this form means ‘change, exchange’ <~PTB \textbf{*s‑lay} (see \textit{STC} \#283, \textit{HPTB} p.\ 208, 216, 217).}
{\tiny \textasciitilde,2137}\\
 &mu\,sú\,\textbf{gur} &eyebrow &\mbox{JAM-Ety} &\hspace*{1ex}{\tiny m,1218,\textasciitilde}\\
 &biʔ-\textbf{gur} &skin &\mbox{JAM-Ety} &\hspace*{1ex}{\tiny p,\textasciitilde}\\
Dimasa &bu\,\textbf{gur} &skin &\mbox{GEM-CNL} &\hspace*{1ex}{\tiny m,\textasciitilde}\\
Garo &anʔ\,\textbf{gil} &skin &\mbox{AW-TBT}:170 &\hspace*{1ex}{\tiny p,\textasciitilde}\\
 &bi-\textbf{gir} &skin &\mbox{JAM-Ety} &\hspace*{1ex}{\tiny p,\textasciitilde}\\
 &bi\,\textbf{gil} &skin &\mbox{AW-TBT}:170 &\hspace*{1ex}{\tiny p,\textasciitilde}\\
 &mik-\textbf{gir} &eye / eyelid &\mbox{JAM-Ety} &\hspace*{1ex}{\tiny 682,\textasciitilde}\\
Garo (Bangladesh) &mik-\textbf{gil} &eyelid &\mbox{RB-GB} &\hspace*{1ex}{\tiny 682,\textasciitilde}\\
Khiamngan &\textbf{¹kᴜn} &skin &\mbox{AW-TBT}:170 &\hspace*{1ex}\\
Kokborok &bə-\textbf{ku} &skin &\mbox{PT-Kok} &\hspace*{1ex}{\tiny p,\textasciitilde}\\
Lalung &\textbf{kur} &skin &\mbox{MB-Lal}:22 &\hspace*{1ex}\\
 &\textbf{kur}\,le\,da &skin &\mbox{MB-Lal}:89 &\hspace*{1ex}{\tiny \textasciitilde,m,m}\\
 &mo\,\textbf{kun} &eyelid &\mbox{MB-Lal}:39 &\raisebox{-0.5ex}{\footnotemark}\footnotetext{Cf.\ also Lalung \textbf{kur} ‘skin’, under \textbf{\textit{\tiny[\#593]}} \textbf{*gul} \STEDTU{⪤} \textbf{*gil} SKIN, pointing to a possible alternation between Lalung final \textbf{‑n} and \textbf{‑r}.}
{\tiny 682,\textasciitilde}\\
Meche &biŋ\,\textbf{gur} &skin &\mbox{AW-TBT}:170 &\hspace*{1ex}{\tiny p,\textasciitilde}\\
 &bi\,\textbf{gur} &skin &\mbox{AW-TBT}:170 &\hspace*{1ex}{\tiny p,\textasciitilde}\\
\end{longtable}
2.1.1. Western Himalayish\\
\fascicletablebegin
Kanauri &chaṅ \textbf{khŭl} &womb (child-skin) &\mbox{JAM-Ety} &\hspace*{1ex}{\tiny m,\textasciitilde}\\
 &\textbf{khŭl(h)} &skin (of sheep, goats, birds); skin of sheep, goats, birds &\mbox{BAI1911}; \mbox{JAM-Ety} &\hspace*{1ex}\\
Pattani [Manchati] &\textbf{khàl} &hide  /  leather (dried animal s &\mbox{STP-ManQ}:8.2.6 &\hspace*{1ex}\\
 &\textbf{kʰəl} &skin &\mbox{DS-Patt} &\hspace*{1ex}\\
\end{longtable}
}

\vspace{1em}
\addcontentsline{toc}{section}{(169)  \textbf{*pʷul/n \STEDTU{⪤} *pʷil/n} SKIN}
\markright{(169) [\#591]  \textbf{*pʷul/n} \STEDTU{⪤} \textbf{*pʷil/n} SKIN}
{\large
\parindent=-1em
(169) \hspace{\stretch{1}} \textbf{*pʷu\begin{tabular}[c]{c}l\\n\\\end{tabular}} \STEDTU{⪤} \textbf{*pʷi\begin{tabular}[c]{c}l\\n\\\end{tabular}}\hspace{\stretch{1}}\textbf{SKIN} \textit{\tiny[\#591]}}

Like other roots reconstructed with \textbf{*pʷ‑}, the reflexes of this etymon vary between labial stops and \textbf{w‑}, \textbf{v‑}, or zero initial. See Matisoff 2000, “An Extrusional Approach to \textbf{*p‑/w‑} Variation in Sino-Tibetan.” This root also shows the widespread variation between medial \textbf{‑i‑} and \textbf{‑u‑}, and also between final \textbf{‑l} and \textbf{‑n}.

A key doublet justifying this reconstruction are the Mizo (Lushai) forms \textbf{pil} and \textbf{vun}, both cited in Marrison 1967:233.

The Sunwar forms with initial \textbf{s‑} and final \textbf{‑l} (\textbf{'ku=sul} ‘	skin’, \textbf{'sul‑cā} ‘feel (by touch)’, \textbf{krusul} ‘bark (of tree)’, \textbf{kusul 'u:‑cā} ‘skin’) might also be related, if we assume an intermediate form like \textbf{**s‑wul}. Cf.\ also French’s (1983) Proto-Northern-Naga reconstruction \textbf{*swun} FLESH. 

See \textit{HPTB} \textbf{*pun \STEDTU{⪤} *pin}, p. 418; \textbf{*pʷul \STEDTU{⪤} *pʷil}, pp. 280, 501; \textbf{*wul \STEDTU{⪤} *wun}, p. 418; \textbf{*ʔul}, p. 58.



{\footnotesize
1.1. North Assam\\
\fascicletablebegin
{}*Tani &*\textbf{pɯn} &skin &\mbox{JS-HCST}:366 &\hspace*{1ex}\\
Bengni &a-\textbf{pin} &skin &\mbox{JS-HCST}; \mbox{JS-Tani} &\hspace*{1ex}{\tiny p,\textasciitilde}\\
 &a-\textbf{pin} di &skin &\mbox{JS-Tani} &\hspace*{1ex}{\tiny p,\textasciitilde,m}\\
Bokar &a \textbf{pin} &skin &\mbox{TBL}:0265.24 &\hspace*{1ex}{\tiny p,\textasciitilde}\\
Bokar Lhoba &a \textbf{pin} &skin &\mbox{ZMYYC}:266.51 &\hspace*{1ex}{\tiny p,\textasciitilde}\\
Bokar &a-\textbf{pin} &skin &\mbox{JS-HCST}; \mbox{JS-Tani} &\hspace*{1ex}{\tiny p,\textasciitilde}\\
Bokar Lhoba &a\,\textbf{pin} &skin; hide, leather &\mbox{SLZO-MLD} &\hspace*{1ex}{\tiny p,\textasciitilde}\\
Bokar &mik-\textbf{pin} &eyelid &\mbox{JS-Tani} &\hspace*{1ex}{\tiny 682,\textasciitilde}\\
Bokar Lhoba &mik\,\textbf{pin} &eyelid &\mbox{SLZO-MLD} &\hspace*{1ex}{\tiny 682,\textasciitilde}\\
Dafla &a\,\textbf{pin} &skin &\mbox{DG-Dafla} &\hspace*{1ex}{\tiny p,\textasciitilde}\\
Gallong &a\,\textbf{pin} &skin &\mbox{KDG-IGL} &\hspace*{1ex}{\tiny p,\textasciitilde}\\
 &dum\,po-a\,\textbf{pin} &scalp &\mbox{KDG-IGL} &\hspace*{1ex}{\tiny 1425,388,p,\textasciitilde}\\
Kaman [Miju] &min⁵⁵ \textbf{uŋ³⁵} &eyelid &\mbox{SLZO-MLD} &\hspace*{1ex}{\tiny 682,\textasciitilde}\\
 &\textbf{uŋ³⁵} &skin &\mbox{SLZO-MLD} &\hspace*{1ex}\\
Geman Deng &\textbf{uŋ³⁵} &skin &\mbox{TBL}:0120.23 &\hspace*{1ex}\\
Kaman [Miju] &\textbf{uŋ³⁵} &skin &\mbox{ZMYYC}:266.48 &\hspace*{1ex}\\
Milang &a\,\textbf{pan} &skin &\mbox{AT-MPB} &\hspace*{1ex}{\tiny p,\textasciitilde}\\
Tagin &a\,\textbf{pin} &skin &\mbox{KDG-Tag} &\hspace*{1ex}{\tiny p,\textasciitilde}\\
 &a\,\textbf{pɯn} &skin &\mbox{KDG-Tag} &\hspace*{1ex}{\tiny p,\textasciitilde}\\
\end{longtable}
1.2. Kuki-Chin\\
\fascicletablebegin
Chinbok &\textbf{vun} &skin &\mbox{JAM-Ety} &\hspace*{1ex}\\
Kuki-Chin &\textbf{vun} &hide; leather (dried animal skin); skin &\mbox{Qbp-KC}:8.2,8.2.6 &\hspace*{1ex}\\
Kom Rem &ǰəŋ \textbf{vun} &foreskin ("penis-skin") &\mbox{T-KomRQ}:10.3.3 &\hspace*{1ex}{\tiny 1283,\textasciitilde}\\
 &lu \textbf{vun} &scalp &\mbox{T-KomRQ}:2.8 &\hspace*{1ex}{\tiny 1221,\textasciitilde}\\
 &mit \textbf{vun} &eyelid &\mbox{T-KomRQ}:3.4.1 &\hspace*{1ex}{\tiny 682,\textasciitilde}\\
 &sə\,\textbf{vun} &hide  /  leather (dried animal skin) &\mbox{T-KomRQ}:8.2.6 &\hspace*{1ex}{\tiny p,\textasciitilde}\\
 &\textbf{vun} &skin &\mbox{T-KomRQ}:8.2 &\hspace*{1ex}\\
Lakher [Mara] &mo-\textbf{vo} &eyelid &\mbox{JAM-Ety} &\hspace*{1ex}{\tiny 682,\textasciitilde}\\
 &sah-\textbf{vo} &skin &\mbox{JAM-Ety} &\hspace*{1ex}{\tiny 34,\textasciitilde}\\
 &\textbf{vo} &skin &\mbox{JAM-Ety} &\hspace*{1ex}\\
 &za-\textbf{vo} &foreskin ("penis-skin") &\mbox{JAM-Ety} &\hspace*{1ex}{\tiny 1283,\textasciitilde}\\
 &za-\textbf{vo}-tai &circumcision &\mbox{JAM-Ety} &\hspace*{1ex}{\tiny 1283,\textasciitilde,m}\\
Lushai [Mizo] &mit-\textbf{vun} &eyelid &\mbox{JAM-Ety} &\hspace*{1ex}{\tiny 682,\textasciitilde}\\
 &\textbf{pil} &skin &\mbox{GEM-CNL}:233 &\hspace*{1ex}\\
 &\textbf{vun} &skin &\mbox{GEM-CNL}:233; \mbox{JAM-Ety}; \mbox{PB-CLDB}:1969 &\hspace*{1ex}\\
Maring &\textbf{un}, \textbf{wun} &skin &\mbox{GEM-CNL} &\hspace*{1ex}\\
Moyon &\textbf{vin} &skin &\mbox{DK-Moyon}:8.2 &\hspace*{1ex}\\
Puiron &\textbf{mun} &skin &\mbox{GEM-CNL} &\raisebox{-0.5ex}{\footnotemark}\footnotetext{This form seems to show a rare nasal reflex of this *labial initial.}
\\
Thado &\textbf{vún} &skin &\mbox{THI1972}:47 &\hspace*{1ex}\\
Tiddim &sa-\textbf{vun} &skin &\mbox{JAM-Ety} &\hspace*{1ex}{\tiny 34,\textasciitilde}\\
 &\textbf{vun¹} &skin &\mbox{PB-CLDB}:1969 &\hspace*{1ex}\\
\end{longtable}
1.3. Naga\\
\fascicletablebegin
{}*Northern Naga &*s\,\textbf{wun} &flesh &\mbox{WTF-PNN}:489 &\hspace*{1ex}{\tiny p,\textasciitilde}\\
 &*\textbf{wur} &skin / flesh &\mbox{WTF-PNN}:549 &\hspace*{1ex}\\
Nocte &na \textbf{van} &earlobe &\mbox{WTF-PNN}:480 &\hspace*{1ex}{\tiny 811,\textasciitilde}\\
\end{longtable}
1.4. Meithei\\
\fascicletablebegin
Meithei &sə\,\textbf{un} &hide  /  leather &\mbox{CYS-Meithei}:8.2.6 &\hspace*{1ex}{\tiny 34,\textasciitilde}\\
 &\textbf{ul} &skin &\mbox{JAM-Ety} &\hspace*{1ex}\\
 &\textbf{un}, \textbf{un}\,sa &skin &\mbox{GEM-CNL} &\hspace*{1ex}{\tiny \textasciitilde,\textasciitilde,34}\\
 &\textbf{un}\,sa &skin &\mbox{CYS-Meithei}:8.2 &\hspace*{1ex}{\tiny \textasciitilde,34}\\
\end{longtable}
2.1.2. Bodic\\
\fascicletablebegin
Kaike &\textbf{phon} &skin &\mbox{JAM-Ety} &\hspace*{1ex}\\
\end{longtable}
2.1.4. Tamangic\\
\fascicletablebegin
Tamang (Sahu) &\textbf{puhr} \textbf{puhr} 'kʰraŋ-pa &skin &\mbox{SIL-Sahu}:14.31 &\hspace*{1ex}{\tiny \textasciitilde,\textasciitilde,m,s}\\
\end{longtable}
2.3.1. Kham-Magar-Chepang-Sunwar\\
\fascicletablebegin
Chepang &narʔ\,\textbf{pun} &lip &\mbox{JAM-Ety} &\raisebox{-0.5ex}{\footnotemark}\footnotetext{\textit{=mouth+skin}}
{\tiny 2099,\textasciitilde}\\
 &nərʔ\,\textbf{pun} &lip &\mbox{SIL-Chep}:2.A.25 &\raisebox{-0.5ex}{\footnotemark}\footnotetext{=mouth+skin}
{\tiny 2099,\textasciitilde}\\
 &\textbf{pun} &skin &\mbox{JAM-Ety}; \mbox{SIL-Chep}:1.28 &\hspace*{1ex}\\
Chepang (Eastern) &mik\,\textbf{pun} &eyelid &\mbox{RC-ChepQ}:3.4.1 &\hspace*{1ex}{\tiny 682,\textasciitilde}\\
 &nərʔ\,\textbf{pun} &lip &\mbox{RC-ChepQ}:3.9 &\raisebox{-0.5ex}{\footnotemark}\footnotetext{\textit{=mouth+skin}}
{\tiny 2099,\textasciitilde}\\
 &\textbf{pun} &skin &\mbox{RC-ChepQ}:8.2 &\hspace*{1ex}\\
Kham &\textbf{ol}kotā &skin &\mbox{JAM-Ety} &\hspace*{1ex}{\tiny \textasciitilde,589,m}\\
 &\textbf{ol}\,ko\,ta &skin &\mbox{DNW-KhamQ}:1.28 &\hspace*{1ex}{\tiny \textasciitilde,589,m}\\
Sunwar &'ku\textbf{sul} &skin &\mbox{JAM-Ety} &\hspace*{1ex}{\tiny 589,\textasciitilde}\\
 &\textbf{'sul}-cā &feel (by touch) &\mbox{AH-CSDPN}:02b1.51 &\hspace*{1ex}{\tiny \textasciitilde,m}\\
 &kru\,\textbf{sul} &bark (of tree) &\mbox{AH-CSDPN}:01.027 &\hspace*{1ex}{\tiny m,\textasciitilde}\\
 &ku\,\textbf{sul} 'u:-cā &skin &\mbox{AH-CSDPN}:03b.31 &\hspace*{1ex}{\tiny 589,\textasciitilde,m,m}\\
\end{longtable}
3.2. Qiangic\\
\fascicletablebegin
Qiang (Mawo) &\textbf{ʴuɛ} piɛ &skin &\mbox{TBL}:0120.08 &\hspace*{1ex}{\tiny \textasciitilde,590}\\
\end{longtable}
3.3. rGyalrongic\\
\fascicletablebegin
rGyalrong (Eastern) &\textbf{wu} scçon &blemish on skin &\mbox{SHK-rGEQ}:8.2.3 &\hspace*{1ex}{\tiny \textasciitilde,m}\\
 &\textbf{wu}\,ʃan\,dʐi &skin &\mbox{SHK-rGEQ}:8.2 &\hspace*{1ex}{\tiny \textasciitilde,m,596}\\
\end{longtable}
4.1. Jingpho\\
\fascicletablebegin
Jingpho &nèʔ-\textbf{ūm} &foreskin / prepuce &\mbox{JAM-TJLB}:126 &\raisebox{-0.5ex}{\footnotemark}\footnotetext{This form seems to show assimilation of the final nasal to the preceding rounded vowel. The first syllable means ‘penis’.}
{\tiny 545,\textasciitilde}\\
\end{longtable}
4.2. Nungic\\
\fascicletablebegin
Dulong &\textbf{pɯn⁵⁵} &skin &\mbox{TBL}:0265.20 &\hspace*{1ex}\\
Trung [Dulong] &šia⁴² \textbf{pun⁴⁴} &skin &\mbox{JAM-Ety} &\hspace*{1ex}{\tiny 34,\textasciitilde}\\
 &ɑŋ³¹\textbf{pɯ̆n⁵⁵} &skin &\mbox{ZMYYC}:266.46 &\hspace*{1ex}{\tiny p,\textasciitilde}\\
Trung [Dulong] (Dulonghe) &ɑŋ³¹ \textbf{pɯ̆n⁵⁵} &skin &\mbox{JZ-Dulong} &\hspace*{1ex}{\tiny p,\textasciitilde}\\
 &ɕɑ⁵⁵ \textbf{pɯn⁵⁵} &leather, hide &\mbox{JZ-Dulong} &\hspace*{1ex}{\tiny 34,\textasciitilde}\\
Trung [Dulong] (Nujiang) &ɑŋ³¹ \textbf{pin⁵³} &skin &\mbox{JZ-Dulong} &\hspace*{1ex}{\tiny p,\textasciitilde}\\
\end{longtable}
}

\vspace{1em}
\addcontentsline{toc}{section}{(170)  \textbf{*s-graw} SKIN / OUTER COVERING}
\markright{(170) [\#585]  \textbf{*s-graw} SKIN / OUTER COVERING}
{\large
\parindent=-1em
(170) \hspace{\stretch{1}} \textbf{*s-graw}\hspace{\stretch{1}}\textbf{SKIN / OUTER COVERING} \textit{\tiny[\#585]}}

See \textit{STC} \#121. (This root was inadvertently omitted from \textit{HPTB}.)



{\footnotesize
0. Sino-Tibetan\\
\fascicletablebegin
{}*Tibeto-Burman &*\textbf{s-graw} &bark of willow, skin &\mbox{STC}:121 &\hspace*{1ex}\\
\end{longtable}
1.1. North Assam\\
\fascicletablebegin
Darang [Taraon] &wa:\,ci\,\textbf{kru} &skin (dry, over sores) &\mbox{NEFA-Taraon} &\hspace*{1ex}{\tiny m,m,\textasciitilde}\\
\end{longtable}
1.3. Naga\\
\fascicletablebegin
Tangsa (Moshang) &a \textbf{kru} &skin &\mbox{WTF-PNN}:549; \mbox{GEM-CNL} &\hspace*{1ex}{\tiny p,\textasciitilde}\\
\end{longtable}
2.1.2. Bodic\\
\fascicletablebegin
Tibetan (Amdo:Zeku) &\textbf{dʑo} &feather, quill &\mbox{JS-Amdo}:776 &\hspace*{1ex}\\
 &ŋo \textbf{dʑo} &hair (on the head) &\mbox{JS-Amdo}:338 &\hspace*{1ex}{\tiny 386,\textasciitilde}\\
 &ɕær \textbf{dʑo} &feather &\mbox{JS-Amdo}:71 &\hspace*{1ex}{\tiny m,\textasciitilde}\\
Tibetan (Written) &bya.\textbf{sgro} &feather &\mbox{JS-Tib}:71 &\hspace*{1ex}{\tiny 1605,\textasciitilde}\\
 &mgo.\textbf{sgro} &hair (on the head) &\mbox{JS-Tib}:338 &\raisebox{-0.5ex}{\footnotemark}\footnotetext{Lit. ‘head-covering’ ? cf.\ WT \textbf{sgro‑ba} ‘bark of a species of willow’. cf.\ also WT \textbf{sgro} ‘feather’.}
{\tiny 386,\textasciitilde}\\
 &\textbf{sgro} &feather, quill; feather &\mbox{JS-Tib}:776; \mbox{WSC-SH}:78 &\hspace*{1ex}\\
 &\textbf{sgro}-ba &bark of willow,skin &\mbox{STC}:121 &\hspace*{1ex}{\tiny \textasciitilde,s}\\
\end{longtable}
2.1.3. Lepcha\\
\fascicletablebegin
Lepcha &\textbf{kryu} &skin &\mbox{JAM-Ety} &\raisebox{-0.5ex}{\footnotemark}\footnotetext{Lepcha medial \textbf{‑y‑} frequently descends from prefixal \textbf{s‑} (\textbf{**s‑kru}). See Benedict 1943, “Secondary infixation in Lepcha.”}
\\
\end{longtable}
2.1.4. Tamangic\\
\fascicletablebegin
Gurung (Ghachok) &\textbf{Tuh}bi &skin &\mbox{JAM-Ety} &\hspace*{1ex}{\tiny \textasciitilde,590}\\
 &\textbf{³ʈu}\,bi &skin &\mbox{MM-K78}:42 &\hspace*{1ex}{\tiny \textasciitilde,590}\\
 &\textbf{ʈuh}\,bi &skin &\mbox{SIL-Gur}:1.28 &\hspace*{1ex}{\tiny \textasciitilde,590}\\
 &\textbf{ʈu}\,hbi plaː\,baq &skin &\mbox{SIL-Gur}:3.B.31 &\hspace*{1ex}{\tiny \textasciitilde,590,792,m}\\
\end{longtable}
2.2. Newar\\
\fascicletablebegin
Newar (Dolakhali) &\textbf{chyɔu}\,ri &skin / bark &\mbox{CG-Dolak} &\hspace*{1ex}{\tiny \textasciitilde,596}\\
\end{longtable}
2.3.2. Kiranti\\
\fascicletablebegin
Hayu &kuk\,\textbf{tsho} &skin &\mbox{BM-PK7}:157; \mbox{JAM-Ety} &\hspace*{1ex}{\tiny 586,\textasciitilde}\\
\end{longtable}
4.1. Jingpho\\
\fascicletablebegin
Jingpho &\textbf{śəgrau} &outer skin, as of fruit &\mbox{STC}:121 &\hspace*{1ex}\\
\end{longtable}
7. Karenic\\
\fascicletablebegin
Palaychi &\textbf{shzù} &skin &\mbox{JAM-Ety}; \mbox{RBJ-KLS}:554 &\hspace*{1ex}\\
\end{longtable}
}

\vspace{1em}
\addcontentsline{toc}{section}{(171)  \textbf{*d(y)al} LIP / SKIN}
\markright{(171) [\#448]  \textbf{*d(y)al} LIP / SKIN}
{\large
\parindent=-1em
(171) \hspace{\stretch{1}} \textbf{*d(y)al}\hspace{\stretch{1}}\textbf{LIP / SKIN} \textit{\tiny[\#448]}}

The final liquid in this root is well-attested, directly in Mizo, Garo, Khaling, and Tangut, and indirectly by Jingpho final \textbf{‑n}.

The first syllables of a pair of forms from the North Assam group (Darang \textbf{thɯ⁵⁵no⁵⁵}, Idu \textbf{tia³⁵ndze⁵⁵} ‘lip’) look rather like the reflexes of this etymon, but they appear to represent a body part prefix which occurs in several other words related loosely to MOUTH: Darang \textbf{tʰɯ³¹ liɯŋ⁵³ nɑ³⁵} ‘tongue’, \textbf{tʰɯ³¹ ɹɯm⁵³ m̩⁵⁵} ‘beard’ \textbf{tʰɯ³¹ tiɑ⁵⁵} ‘chin’; Idu \textbf{tia³⁵pɹa³⁵} ‘tooth’.



{\footnotesize
1.2. Kuki-Chin\\
\fascicletablebegin
Lushai [Mizo] &\textbf{dal} &skin &\mbox{JAM-Ety} &\hspace*{1ex}\\
\end{longtable}
1.3. Naga\\
\fascicletablebegin
Liangmei &cha\,mun\,\textbf{tai} &lip &\mbox{GEM-CNL} &\hspace*{1ex}{\tiny 2095,467,\textasciitilde}\\
Maram &ka\,\textbf{tei} &lip &\mbox{GEM-CNL} &\hspace*{1ex}{\tiny 466,\textasciitilde}\\
Tangkhul &mor\,\textbf{chai} &lip &\mbox{GEM-CNL}; \mbox{JAM-Ety} &\hspace*{1ex}{\tiny 467,\textasciitilde}\\
\end{longtable}
1.7. Bodo-Garo = Barish\\
\fascicletablebegin
Dimasa &khu\,\textbf{jer} &lip &\mbox{GEM-CNL} &\hspace*{1ex}{\tiny 468,\textasciitilde}\\
Garo (Bangladesh) &ku'-\textbf{chil} &lip &\mbox{RB-GB} &\hspace*{1ex}{\tiny 468,\textasciitilde}\\
 &ri-ku-\textbf{chil} &foreskin &\mbox{RB-GB} &\hspace*{1ex}{\tiny 1284,m,\textasciitilde}\\
\end{longtable}
2.1.2. Bodic\\
\fascicletablebegin
Kaike &chhyo\,\textbf{chyā} &lip &\mbox{JAM-Ety} &\hspace*{1ex}{\tiny 442,\textasciitilde}\\
\end{longtable}
2.3.2. Kiranti\\
\fascicletablebegin
Bahing &kok-\textbf{te} &skin &\mbox{JAM-Ety}; \mbox{STC}:342 &\hspace*{1ex}{\tiny 586,\textasciitilde}\\
Bantawa &do\,si\,\textbf{ja} &lip &\mbox{JAM-Ety} &\hspace*{1ex}{\tiny 478,686,\textasciitilde}\\
Khaling &kwām\,to\,\textbf{tar} &lip &\mbox{JAM-Ety} &\hspace*{1ex}{\tiny 454,2095,\textasciitilde}\\
Limbu &kon\,\textbf{de} &lip &\mbox{JAM-Ety} &\hspace*{1ex}{\tiny 473,\textasciitilde}\\
 &mu\,kon\,\textbf{de} &lower lip &\mbox{JAM-Ety} &\hspace*{1ex}{\tiny m,473,\textasciitilde}\\
 &thāŋ\,kon\,\textbf{de} &upper lip &\mbox{JAM-Ety} &\hspace*{1ex}{\tiny 1722,473,\textasciitilde}\\
Thulung &kokå\,\textbf{te} &skin &\mbox{JAM-Ety} &\hspace*{1ex}{\tiny 586,p,\textasciitilde}\\
 &si\,kok\,\textbf{te} &lip &\mbox{NJA-Thulung} &\hspace*{1ex}{\tiny 686,586,\textasciitilde}\\
 &sī\,ko\,kaʔ\,\textbf{te} &lip &\mbox{JAM-Ety} &\hspace*{1ex}{\tiny 686,586,m,\textasciitilde}\\
\end{longtable}
3.1. Tangut\\
\fascicletablebegin
Tangut [Xixia] &\textbf{dɑr} &skin &\mbox{DQ-Xixia}:8.2 &\hspace*{1ex}\\
\end{longtable}
3.2. Qiangic\\
\fascicletablebegin
Qiang (Mawo) &\textbf{dʐa} pi &hide  /  leather (dried animal s &\mbox{SHK-MawoQ}:8.2.6 &\hspace*{1ex}{\tiny \textasciitilde,590}\\
 &\textbf{ɣdʑæ(a?)ʴ} &lip &\mbox{SHK-MawoQ}:3.9 &\hspace*{1ex}\\
 &\textbf{ɣdʑɑːʴ} &lip &\mbox{JZ-Qiang} &\hspace*{1ex}\\
\end{longtable}
3.3. rGyalrongic\\
\fascicletablebegin
Ergong (Northern) &dʒə³³ \textbf{dʒua³³} &skin &\mbox{SHK-ErgNQ}:8.2 &\hspace*{1ex}{\tiny 596,\textasciitilde}\\
Ergong (Danba) &dʑi \textbf{dʑa} &skin &\mbox{SHK-ErgDQ}:8.2 &\hspace*{1ex}{\tiny 596,\textasciitilde}\\
Ergong (Daofu) &huə \textbf{dʑa} &scalp &\mbox{DQ-Daofu}:2.8 &\hspace*{1ex}{\tiny 386,\textasciitilde}\\
 &rədɔ \textbf{tɕa} &hide  /  leather &\mbox{DQ-Daofu}:8.2.6 &\hspace*{1ex}{\tiny 1862,\textasciitilde}\\
Ergong (Northern) &ʁo⁵³ dʑi³³ \textbf{dʑua³³} &scalp &\mbox{SHK-ErgNQ}:2.8 &\hspace*{1ex}{\tiny 386,596,\textasciitilde}\\
Ergong (Daofu) &ʔmaɹ \textbf{dʑa} &lip &\mbox{DQ-Daofu}:3.9 &\hspace*{1ex}{\tiny 467,\textasciitilde}\\
\end{longtable}
4.1. Jingpho\\
\fascicletablebegin
Jingpho &n-\textbf{ten} &lip &\mbox{JAM-Ety} &\hspace*{1ex}{\tiny p,\textasciitilde}\\
 &ning\,\textbf{ten} &lip &\mbox{JAM-Ety} &\hspace*{1ex}{\tiny p,\textasciitilde}\\
 &n̩³¹ \textbf{te̱n³³} &lip &\mbox{JZ-Jingpo} &\hspace*{1ex}{\tiny p,\textasciitilde}\\
 &n\,\textbf{ten} &lip &\mbox{GEM-CNL}; \mbox{JAM-Ety} &\hspace*{1ex}{\tiny p,\textasciitilde}\\
 &sai\,len n-\textbf{ten} &lip &\mbox{JAM-Ety} &\hspace*{1ex}{\tiny m,m,p,\textasciitilde}\\
 &shing\,\textbf{ten} &lip &\mbox{JAM-Ety} &\hspace*{1ex}{\tiny p,\textasciitilde}\\
\end{longtable}
}

\vspace{1em}
\addcontentsline{toc}{section}{(172)  \textbf{*kop \STEDTU{⪤} *kwap} SKIN / LIP / SCALES (fish) / SHELL}
\markright{(172) [\#783]  \textbf{*kop} \STEDTU{⪤} \textbf{*kwap} SKIN / LIP / SCALES (fish) / SHELL}
{\large
\parindent=-1em
(172) \hspace{\stretch{1}} \textbf{*kop} \STEDTU{⪤} \textbf{*kwap}\hspace{\stretch{1}}\textbf{SKIN / LIP / SCALES (fish) / SHELL} \textit{\tiny[\#783]}}

This etymon is probably related to TBRS \textbf{(11a)} \textbf{*gop} \STEDTU{⪤} \textbf{*kop} HATCH / INCUBATE / COVER. Cf.\ e.g.\ Ao (Chungli) \textbf{si küp} ‘skin’, \textbf{küp bang} ‘cover’. Jg. \textbf{ǹ‑gùp} ‘mouth’, despite its resemblance to Kaman \textbf{gᴜ̀p} ‘skin’ and a possible semantic connection with LIP, is only a remote candidate for relationship.

In several putative cognates (see Yakha, Achang, and perhaps Ahi and Lolopho) this etymon  means ‘scales of fish’.



{\footnotesize
1.1. North Assam\\
\fascicletablebegin
Kaman [Miju] &\textbf{gᴜ̀p} &skin &\mbox{AW-TBT}:170 &\hspace*{1ex}\\
\end{longtable}
1.3. Naga\\
\fascicletablebegin
{}*Northern Naga &*\textbf{goːp} &shell &\mbox{WTF-PNN}:544 &\hspace*{1ex}\\
 &*\textbf{kʰoːp} &lip &\mbox{WTF-PNN}:515 &\hspace*{1ex}\\
Ao (Chungli) &si\,\textbf{küp} &skin &\mbox{GEM-CNL} &\hspace*{1ex}{\tiny m,\textasciitilde}\\
Ao (Mongsen) &hŋa-\textbf{kəp} &fish scale ('fish' + 'skin') &\mbox{ARC-GMA} &\hspace*{1ex}{\tiny 1455,\textasciitilde}\\
 &paŋ-\textbf{kəp} &lip ('mouth' + 'skin') &\mbox{ARC-GMA} &\hspace*{1ex}{\tiny 476,\textasciitilde}\\
 &səŋ-\textbf{kəp} &bark ('wood' + 'skin') &\mbox{ARC-GMA} &\hspace*{1ex}{\tiny 2658,\textasciitilde}\\
 &tü\,\textbf{kap} &skin &\mbox{GEM-CNL} &\hspace*{1ex}{\tiny m,\textasciitilde}\\
 &[tə]-\textbf{kəp} &rind (skin); skin &\mbox{ARC-GMA} &\hspace*{1ex}{\tiny p,\textasciitilde}\\
Chang &\textbf{kop} &shell (used of any hard covering of animals or plants) &\mbox{WTF-PNN}:544 &\hspace*{1ex}\\
 &\textbf{kop} lin bu &thin (of persons) &\mbox{WTF-PNN}:545 &\hspace*{1ex}{\tiny \textasciitilde,m,m}\\
 &ñek \textbf{kop} &eyelid &\mbox{WTF-PNN}:515 &\hspace*{1ex}{\tiny 682,\textasciitilde}\\
 &ñiŋ ke \textbf{kap} tuŋ &snail &\mbox{WTF-PNN}:545 &\hspace*{1ex}{\tiny m,m,\textasciitilde,m}\\
 &sam puŋ \textbf{kop} &lip &\mbox{WTF-PNN}:515 &\hspace*{1ex}{\tiny m,476,\textasciitilde}\\
 &sam\,pung\,\textbf{kop} &lip &\mbox{GEM-CNL} &\hspace*{1ex}{\tiny m,476,\textasciitilde}\\
Nocte &tʰun \textbf{kʰuap} &lip &\mbox{WTF-PNN}:515 &\hspace*{1ex}{\tiny 442,\textasciitilde}\\
Sangtam &\textbf{kep} &skin &\mbox{GEM-CNL} &\hspace*{1ex}\\
 &pe\,\textbf{küp} &lip &\mbox{GEM-CNL} &\hspace*{1ex}{\tiny 2096,\textasciitilde}\\
Wancho &\textbf{kap} tuŋ &shell &\mbox{WTF-PNN}:544 &\hspace*{1ex}{\tiny \textasciitilde,m}\\
Yacham-Tengsa &ta\,\textbf{kap} &skin &\mbox{GEM-CNL} &\hspace*{1ex}{\tiny p,\textasciitilde}\\
\end{longtable}
2.1.2. Bodic\\
\fascicletablebegin
Tsangla (Central) &\textbf{khop}\,tang &skin &\mbox{SER-HSL/T}:36  10 &\hspace*{1ex}{\tiny \textasciitilde,785}\\
 &ming\,\textbf{khop}\,tang &eyelid &\mbox{SER-HSL/T}:32  3 &\hspace*{1ex}{\tiny 682,\textasciitilde,711}\\
Tsangla (Motuo) &\textbf{kʰop⁵⁵} taŋ⁵⁵ &skin &\mbox{JZ-CLMenba} &\hspace*{1ex}{\tiny \textasciitilde,785}\\
 &no¹³ waŋ¹³ \textbf{kʰop⁵⁵} taŋ⁵⁵ &lip &\mbox{JZ-CLMenba} &\hspace*{1ex}{\tiny 470,476,\textasciitilde,m}\\
 &no\,waŋ \textbf{khop}\,taŋ &lip &\mbox{ZMYYC}:243.7 &\hspace*{1ex}{\tiny 470,476,\textasciitilde,m}\\
Tsangla (Tilang) &no-waŋ-\textbf{kʰop}-taŋ &lip &\mbox{JZ-CLMenba} &\hspace*{1ex}{\tiny 470,476,\textasciitilde,m}\\
\end{longtable}
2.3.2. Kiranti\\
\fascicletablebegin
Yakha &na\,saga ɔ\,\textbf{kɔp} &scales (of fish) &\mbox{TK-Yakha}:8.2.5 &\hspace*{1ex}{\tiny 1455,m,m,\textasciitilde}\\
\end{longtable}
4.2. Nungic\\
\fascicletablebegin
Trung [Dulong] &ñi⁴⁴ \textbf{kop⁴⁴} &lip &\mbox{JAM-Ety} &\hspace*{1ex}{\tiny 467,\textasciitilde}\\
Trung [Dulong] (Dulonghe) &me⁵⁵ \textbf{kɔ̆p⁵⁵} &eyelid &\mbox{JZ-Dulong} &\hspace*{1ex}{\tiny 682,\textasciitilde}\\
 &nɯi⁵⁵ \textbf{kɔp⁵⁵} &lip &\mbox{JZ-Dulong} &\hspace*{1ex}{\tiny 467,\textasciitilde}\\
\end{longtable}
6.1. Burmish\\
\fascicletablebegin
Achang (Luxi) &ŋa⁵⁵ \textbf{kjap⁵⁵} &scales &\mbox{JZ-Achang} &\hspace*{1ex}{\tiny 1455,\textasciitilde}\\
Achang (Xiandao) &ŋ̊ɔ̆³¹ ʂɔ³¹ a³¹ \textbf{kɤp³⁵} &scales &\mbox{DQ-Xiandao}:409 &\hspace*{1ex}{\tiny 1455,m,p,\textasciitilde}\\
Atsi [Zaiwa] &ŋŏ²¹ \textbf{kja̱p⁵⁵} &scales &\mbox{JZ-Zaiwa} &\hspace*{1ex}{\tiny 1455,\textasciitilde}\\
\end{longtable}
6.2. Loloish\\
\fascicletablebegin
Ahi &ɑ³³ ŋo²¹ i³³ \textbf{kɯ⁵⁵} &scales (of fish) &\mbox{LMZ-AhiQ}:8.2.5 &\hspace*{1ex}{\tiny p,1455,m,\textasciitilde}\\
Lolopho &\textbf{kɤ̱⁴⁴} &scales (of fish) &\mbox{DQ-Lolopho}:8.2.5 &\hspace*{1ex}\\
\end{longtable}
}

\vspace{1em}
\addcontentsline{toc}{section}{(173)  \textbf{*ko \STEDTU{⪤} *kwa} SKIN}
\markright{(173) [\#589]  \textbf{*ko} \STEDTU{⪤} \textbf{*kwa} SKIN}
{\large
\parindent=-1em
(173) \hspace{\stretch{1}} \textbf{*ko} \STEDTU{⪤} \textbf{*kwa}\hspace{\stretch{1}}\textbf{SKIN} \textit{\tiny[\#589]}}

This root seems independent of \textbf{\textit{\tiny[\#586]}} \textbf{*s/r‑kok} \STEDTU{⪤} \textbf{*(r‑)kwak} SKIN, BARK, RIND. Cf.\ WT \textbf{kó‑ba} (below) vs.\ WT \textbf{kog‑pa} in \textbf{\textit{\tiny[\#586]}}. A number of supporting forms for this etymon have fricative initials \textbf{h‑} or \textbf{f‑}. Interplay between laryngeal and velar initials is a common phenomenon in TB; see Matisoff 1997:33 and \textit{HPTB}:54-58.

Forms like Lisu \textbf{ko³⁵dʒi³³} and \textbf{hwa⁵‑ji⁴} seem to reflect such variation internally among Lisu dialects.



{\footnotesize
1.1. North Assam\\
\fascicletablebegin
Apatani &a-\textbf{ha} &skin &\mbox{JS-Tani} &\hspace*{1ex}{\tiny p,\textasciitilde}\\
Damu &ȵok-\textbf{ko} &lip &\mbox{JS-Tani} &\hspace*{1ex}{\tiny 2100,\textasciitilde}\\
Darang [Taraon] &blm-\textbf{ka} &eyelid &\mbox{JAM-Ety} &\hspace*{1ex}{\tiny 1427,\textasciitilde}\\
 &bɯ³¹lɯm⁵⁵ \textbf{ko⁵⁵} &eyelid &\mbox{SLZO-MLD} &\hspace*{1ex}{\tiny 1427,\textasciitilde}\\
 &\textbf{ka} &skin &\mbox{JAM-Ety} &\hspace*{1ex}\\
 &\textbf{ko⁵⁵} &skin &\mbox{SLZO-MLD} &\hspace*{1ex}\\
Darang Deng &\textbf{ko⁵⁵} &skin &\mbox{TBL}:0120.22 &\hspace*{1ex}\\
Darang [Taraon] &\textbf{ko⁵⁵} &skin &\mbox{ZMYYC}:266.49 &\hspace*{1ex}\\
Idu &e\,\textbf{ko}\,\textbf{ko} &scalp &\mbox{JP-Idu} &\hspace*{1ex}{\tiny m,\textasciitilde,\textasciitilde}\\
 &e\,lõ\,\textbf{ko} &eyelid &\mbox{JP-Idu} &\hspace*{1ex}{\tiny m,1427,\textasciitilde}\\
 &\textbf{ko} \textbf{ko}\,pra &leather &\mbox{JP-Idu} &\hspace*{1ex}{\tiny \textasciitilde,\textasciitilde,792}\\
 &\textbf{ko⁵⁵}pɹa⁵⁵ &skin &\mbox{SHK-Idu}:8.2 &\hspace*{1ex}{\tiny \textasciitilde,792}\\
 &\textbf{ko⁵⁵}pɹɑ⁵⁵ &skin &\mbox{ZMYYC}:266.50 &\hspace*{1ex}{\tiny \textasciitilde,792}\\
 &\textbf{ko}\,li\,pra &scale(fish) &\mbox{JP-Idu} &\hspace*{1ex}{\tiny \textasciitilde,562,792}\\
 &\textbf{ko}\,pra &bark; skin &\mbox{JP-Idu} &\hspace*{1ex}{\tiny \textasciitilde,792}\\
Sulung &a³³\textbf{kə⁵³} &skin &\mbox{SHK-Sulung}; \mbox{ZMYYC}:266.52 &\hspace*{1ex}{\tiny p,\textasciitilde}\\
Darang [Taraon] &\textbf{ka} &skin (of fruit); skin; hide (skin, leather) &\mbox{NEFA-Taraon} &\hspace*{1ex}\\
Yidu &\textbf{ko⁵⁵}pɹɑ⁵⁵ &skin &\mbox{TBL}:0120.25 &\hspace*{1ex}{\tiny \textasciitilde,792}\\
\end{longtable}
1.2. Kuki-Chin\\
\fascicletablebegin
Anal &\textbf{wɔ̀}\,kɔ̀ɔŋ &skin &\mbox{AW-TBT}:170 &\hspace*{1ex}{\tiny \textasciitilde,780}\\
\end{longtable}
1.3. Naga\\
\fascicletablebegin
Lotha &ō\,\textbf{fə̀} &skin &\mbox{A-LPD} &\hspace*{1ex}{\tiny p,\textasciitilde}\\
Lotha Naga &men\,\textbf{fu} &lip &\mbox{GEM-CNL} &\hspace*{1ex}{\tiny 467,\textasciitilde}\\
 &o\,\textbf{fhu} &skin &\mbox{GEM-CNL} &\hspace*{1ex}{\tiny p,\textasciitilde}\\
 &so\,\textbf{fu} &leather &\mbox{GEM-CNL} &\hspace*{1ex}{\tiny p,\textasciitilde}\\
Mao &o\,\textbf{hei} &skin &\mbox{GEM-CNL} &\hspace*{1ex}{\tiny p,\textasciitilde}\\
Sema &a\,yi\,\textbf{khwo} &skin &\mbox{GEM-CNL} &\hspace*{1ex}{\tiny p,m,\textasciitilde}\\
 &²a²i\textbf{¹ko} &skin &\mbox{AW-TBT}:170 &\hspace*{1ex}{\tiny m,m,\textasciitilde}\\
Tangkhul &a\,\textbf{hui} &skin &\mbox{GEM-CNL} &\hspace*{1ex}{\tiny p,\textasciitilde}\\
 &\textbf{huy} &skin &\mbox{Bhat-TNV}:96 &\hspace*{1ex}\\
Wancho &\textbf{ho}-lop &skin &\mbox{JAM-Ety} &\hspace*{1ex}{\tiny \textasciitilde,598}\\
\end{longtable}
1.4. Meithei\\
\fascicletablebegin
Meithei &mə\,\textbf{ku} &scales (of fish) &\mbox{CYS-Meithei}:8.2.5 &\hspace*{1ex}{\tiny p,\textasciitilde}\\
\end{longtable}
1.5. Mikir\\
\fascicletablebegin
Mikir &\textbf{hū} &skin &\mbox{KHG-Mikir}:192 &\hspace*{1ex}\\
\end{longtable}
1.7. Bodo-Garo = Barish\\
\fascicletablebegin
Kokborok &\textbf{huy} &hide &\mbox{PT-Kok} &\hspace*{1ex}\\
\end{longtable}
2.1.2. Bodic\\
\fascicletablebegin
Tshona (Mama) &phe⁵⁵\textbf{khu⁵³} &skin &\mbox{ZMYYC}:266.6 &\hspace*{1ex}{\tiny m,\textasciitilde}\\
 &pʰe⁵⁵ \textbf{kʰu⁵³} &skin &\mbox{SLZO-MLD} &\hspace*{1ex}{\tiny m,\textasciitilde}\\
Tibetan (Alike) &wok \textbf{kwa} &skin (leather, hide) &\mbox{TBL}:0265.05 &\hspace*{1ex}{\tiny 588,\textasciitilde}\\
Tibetan (Amdo:Bla-brang) &wak \textbf{kwa} &skin &\mbox{ZMYYC}:266.4 &\hspace*{1ex}{\tiny 588,\textasciitilde}\\
Tibetan (Batang) &\textbf{xha⁵⁵}pɑ⁵³ &skin &\mbox{TBL}:0120.03 &\hspace*{1ex}{\tiny \textasciitilde,s}\\
Tibetan (Sherpa:Helambu) &kha\,ge \textbf{ko}\,ba &lip &\mbox{B-ShrpaHQ}:3.9 &\raisebox{-0.5ex}{\footnotemark}\footnotetext{Lit. “mouth+genitive + skin”.}
{\tiny 466,m,\textasciitilde,s}\\
Sherpa &\textbf{ko}wa &skin &\mbox{JAM-Ety} &\hspace*{1ex}{\tiny \textasciitilde,s}\\
Tibetan (Sherpa:Helambu) &\textbf{kōba} kambū &hide  /  leather (dried animal s &\mbox{B-ShrpaHQ}:8.2.6 &\hspace*{1ex}{\tiny \textasciitilde,s,m,m}\\
 &\textbf{kō}\,ba &scalp; skin &\mbox{B-ShrpaHQ}:2.8,8.2 &\hspace*{1ex}{\tiny \textasciitilde,s}\\
 &mē\,\textbf{kō}\,ba &eyelid &\mbox{B-ShrpaHQ}:3.4.1 &\hspace*{1ex}{\tiny 682,\textasciitilde,m}\\
 &nya \textbf{kō}\,ba &scales (of fish) &\mbox{B-ShrpaHQ}:8.2.5 &\hspace*{1ex}{\tiny 1455,\textasciitilde,s}\\
Spiti &\textbf{ku}\,wa &hide  /  leather (dried animal s &\mbox{CB-SpitiQ}:8.2.6 &\hspace*{1ex}{\tiny \textasciitilde,s}\\
Tibetan (Written) &\textbf{kó}-ba &hide, skin &\mbox{JAM-Ety} &\hspace*{1ex}{\tiny \textasciitilde,s}\\
 &\textbf{ko}\,ba &skin &\mbox{GEM-CNL} &\hspace*{1ex}{\tiny \textasciitilde,s}\\
Tibetan (Xiahe) &\textbf{ko} &skin (leather, hide) &\mbox{TBL}:0265.04 &\hspace*{1ex}\\
 &ʔoχ \textbf{kwa} &skin (with hair) &\mbox{TBL}:0265.04 &\hspace*{1ex}{\tiny m,\textasciitilde}\\
Cuona Menba &phe⁵⁵\textbf{khu⁵³} &skin &\mbox{TBL}:0120.06 &\hspace*{1ex}{\tiny m,\textasciitilde}\\
 &phe⁵⁵\textbf{ku⁵³} &skin &\mbox{TBL}:0265.06 &\hspace*{1ex}{\tiny m,\textasciitilde}\\
\end{longtable}
2.1.4. Tamangic\\
\fascicletablebegin
Thakali (Tukche) &sung-\textbf{kha} &lip &\mbox{JAM-Ety} &\hspace*{1ex}{\tiny 477,\textasciitilde}\\
 &suŋ-\textbf{kʰɔ} &lip &\mbox{SIL-Thak}:2.A.25 &\hspace*{1ex}{\tiny 477,\textasciitilde}\\
\end{longtable}
2.2. Newar\\
\fascicletablebegin
Newar &chen\,\textbf{gu} (li) &skin &\mbox{JAM-Ety} &\hspace*{1ex}{\tiny m,\textasciitilde,m}\\
 &cheŋ\,\textbf{gu:} &hide  /  leather &\mbox{SH-KNw}:8.2.6 &\hspace*{1ex}{\tiny m,\textasciitilde}\\
 &chẽ\,\textbf{gu:} pi-ye &skin, to &\mbox{SH-KNw}:8.2 &\hspace*{1ex}{\tiny m,\textasciitilde}\\
Newar (Kathmandu) &chyɔŋ\,\textbf{gu} &hide / skin &\mbox{CG-Kath} &\hspace*{1ex}{\tiny m,\textasciitilde}\\
\end{longtable}
2.3.1. Kham-Magar-Chepang-Sunwar\\
\fascicletablebegin
Chepang (Eastern) &\textbf{kaw} &scales (of fish) &\mbox{RC-ChepQ}:8.2.5 &\hspace*{1ex}\\
Kham &ol\textbf{ko}tā &skin &\mbox{JAM-Ety} &\hspace*{1ex}{\tiny 591,\textasciitilde,m}\\
 &ol\,\textbf{ko}\,ta &skin &\mbox{DNW-KhamQ}:1.28 &\hspace*{1ex}{\tiny 591,\textasciitilde,m}\\
Sunwar &\textbf{'ku}sul &skin &\mbox{JAM-Ety} &\hspace*{1ex}{\tiny \textasciitilde,591}\\
 &\textbf{ku}\,sul 'u:-cā &skin &\mbox{AH-CSDPN}:03b.31 &\hspace*{1ex}{\tiny \textasciitilde,591,m,m}\\
\end{longtable}
2.3.2. Kiranti\\
\fascicletablebegin
Khaling &\textbf{'kaa} &skin &\mbox{BM-PK7}:157 &\hspace*{1ex}\\
 &sa\textbf{ka} &skin &\mbox{JAM-Ety} &\hspace*{1ex}{\tiny 34,\textasciitilde}\\
Yakha &hɛː\,rana \textbf{wa}\,rik &hide  /  leather (dried animal s &\mbox{TK-Yakha}:8.2.6 &\hspace*{1ex}{\tiny m,m,\textasciitilde,782}\\
 &i\,\textbf{ɦwa}\,rik &skin &\mbox{AW-TBT}:170 &\hspace*{1ex}{\tiny p,\textasciitilde,782}\\
 &\textbf{uɦa}\,rik &skin &\mbox{AW-TBT}:170 &\hspace*{1ex}{\tiny \textasciitilde,782}\\
 &\textbf{wəha}\,rik &scalp &\mbox{TK-Yakha}:2.8 &\hspace*{1ex}{\tiny \textasciitilde,782}\\
 &\textbf{wəha}\,riːk &skin &\mbox{TK-Yakha}:8.2 &\hspace*{1ex}{\tiny \textasciitilde,782}\\
 &\textbf{ɦwa}\,rik &skin &\mbox{AW-TBT}:170 &\hspace*{1ex}{\tiny \textasciitilde,782}\\
\end{longtable}
3.2. Qiangic\\
\fascicletablebegin
Namuyi &ɣɯ³³\textbf{əʴ³³qa⁵⁵} &hide  /  leather &\mbox{SHK-NamuQ}:8.2.6 &\raisebox{-0.5ex}{\footnotemark}\footnotetext{First syllable means ‘cattle’.}
{\tiny m,\textasciitilde}\\
 &əʴ³¹\textbf{qa⁵³} &skin &\mbox{TBL}:0265.46 &\hspace*{1ex}{\tiny p,\textasciitilde}\\
\end{longtable}
3.3. rGyalrongic\\
\fascicletablebegin
rGyalrong (Eastern) &mɲɐ ndʐi \textbf{r̥kʰo} &eyelid &\mbox{SHK-rGEQ}:3.4.1 &\hspace*{1ex}{\tiny 681,596,\textasciitilde}\\
rGyalrong &tə mȵak \textbf{r̥kʰo} &eyelid &\mbox{DQ-Jiarong}:3.4.1 &\hspace*{1ex}{\tiny p,681,\textasciitilde}\\
rGyalrong (Eastern) &tʃi\,wjo \textbf{wu}\,r̥kʰo &scales (of fish) &\mbox{SHK-rGEQ}:8.2.5 &\raisebox{-0.5ex}{\footnotemark}\footnotetext{\textbf{tʃiwjo} means ‘fish’.}
{\tiny m,m,\textasciitilde,586}\\
\end{longtable}
4.3. Luish\\
\fascicletablebegin
Lui &la\,\textbf{ho} &skin &\mbox{JAM-Ety} &\hspace*{1ex}{\tiny 602,\textasciitilde}\\
\end{longtable}
5. Tujia\\
\fascicletablebegin
Tujia &a²¹ la⁵⁵ \textbf{kɯe²¹} lo²¹ &eyelid &\mbox{CK-TujMQ}:3.4.1 &\raisebox{-0.5ex}{\footnotemark}\footnotetext{\textbf{a²¹ la⁵⁵} means ‘eye’.}
{\tiny p,1427,\textasciitilde,m}\\
 &\textbf{kɯe²¹} lo²¹ &hide; leather (dried animal skin); skin &\mbox{CK-TujMQ}:8.2,8.2.6 &\hspace*{1ex}{\tiny \textasciitilde,595}\\
\end{longtable}
6.2. Loloish\\
\fascicletablebegin
Lisu &\textbf{hwa⁵}-ji⁴ &skin &\mbox{JAM-Ety} &\hspace*{1ex}{\tiny \textasciitilde,596}\\
Lisu (Central) &\textbf{hwa⁵}-ji⁴ &skin &\mbox{JF-HLL} &\hspace*{1ex}{\tiny \textasciitilde,596}\\
Lisu &\textbf{kaw³}-ji⁴ &skin &\mbox{JAM-Ety} &\hspace*{1ex}{\tiny \textasciitilde,596}\\
Lisu (Nujiang) &\textbf{ko³⁵} dʒi³³ &skin &\mbox{JZ-Lisu} &\hspace*{1ex}{\tiny \textasciitilde,596}\\
Lisu &\textbf{ko³⁵}dʒi³³ &skin &\mbox{ZMYYC}:266.27 &\hspace*{1ex}{\tiny \textasciitilde,596}\\
 &\textbf{ko³⁵}dʑi³³ &skin &\mbox{TBL}:0120.40 &\hspace*{1ex}{\tiny \textasciitilde,596}\\
Nusu (Bijiang) &\textbf{khu³¹}ɹi³⁵ &skin &\mbox{ZMYYC}:266.45 &\hspace*{1ex}{\tiny \textasciitilde,596}\\
Sani [Nyi] &\textbf{ku⁴⁴} &molted skin (insect) &\mbox{MXL-SaniQ}:325.1 &\hspace*{1ex}\\
 &o⁵⁵ \textbf{ku³³} &scalp &\mbox{WAH-Sani}:325.5; \mbox{CK-YiQ}:2.8 &\hspace*{1ex}{\tiny 385,\textasciitilde}\\
 &\textbf{qɤ⁵⁵} &shell, skin; skin, outer covering &\mbox{MXL-SaniQ}:332.2; \mbox{YHJC-Sani} &\hspace*{1ex}\\
Yi (Sani) &\textbf{qɤ⁵⁵}tsz̊³³ &skin &\mbox{TBL}:0120.39 &\hspace*{1ex}{\tiny \textasciitilde,596}\\
Phunoi &bàn\,\textbf{khò} &lip &\mbox{JAM-Ety} &\hspace*{1ex}{\tiny 453,\textasciitilde}\\
 &bɑn³³ \textbf{kʰo³³} &lip &\mbox{DB-Phunoi} &\hspace*{1ex}{\tiny 453,\textasciitilde}\\
 &bɑn¹¹ \textbf{kʰo¹¹} &lip &\mbox{DB-Phunoi} &\hspace*{1ex}{\tiny 453,\textasciitilde}\\
Sangkong &aŋ³³ \textbf{hu³¹} &skin &\mbox{LYS-Sangkon} &\hspace*{1ex}{\tiny p,\textasciitilde}\\
Yi (Mile) &\textbf{xo²¹}tɕi³³ &skin &\mbox{ZMYYC}:266.25 &\hspace*{1ex}{\tiny \textasciitilde,596}\\
Yi (Nanhua) &\textbf{xo²¹}dʑi³³ &skin &\mbox{ZMYYC}:266.24 &\hspace*{1ex}{\tiny \textasciitilde,596}\\
\end{longtable}
6.4. Jinuo\\
\fascicletablebegin
Jinuo &a⁴⁴\textbf{kho⁴²} &skin &\mbox{TBL}:0265.44; \mbox{ZMYYC}:266.34 &\hspace*{1ex}{\tiny p,\textasciitilde}\\
Jinuo (Buyuan) &ɑ⁴²\textbf{khu⁴⁴} &skin &\mbox{JZ-Jinuo} &\hspace*{1ex}{\tiny p,\textasciitilde}\\
\end{longtable}
}

\vspace{1em}
\addcontentsline{toc}{section}{(174)  \textbf{*s-lip} FISH SCALES}
\markright{(174) [\#562]  \textbf{*s-lip} FISH SCALES}
{\large
\parindent=-1em
(174) \hspace{\stretch{1}} \textbf{*s-lip}\hspace{\stretch{1}}\textbf{FISH SCALES} \textit{\tiny[\#562]}}

The Kuki-Chin forms are reconstructed as \textbf{\textit{\tiny[\#5048]}} \textbf{*lip} SCAB / SCALE in VanBik 2009:\#1048.



{\footnotesize
1.1. North Assam\\
\fascicletablebegin
Idu &ko\,\textbf{li}\,pra &scale(fish) &\mbox{JP-Idu} &\hspace*{1ex}{\tiny 589,\textasciitilde,792}\\
\end{longtable}
1.2. Kuki-Chin\\
\fascicletablebegin
Ashö [Sho] &ŋò-\textbf{léʔ} &scale (fish) &\mbox{AW-TBT}:1028 &\hspace*{1ex}{\tiny 1455,\textasciitilde}\\
Kom Rem &ŋə \textbf{rip} &scales (of fish) &\mbox{T-KomRQ}:8.2.5 &\hspace*{1ex}{\tiny 1455,\textasciitilde}\\
Lakher [Mara] &phu-\textbf{hlu(}-pa) &scales (of fish) &\mbox{JAM-Ety} &\hspace*{1ex}{\tiny m,\textasciitilde,s}\\
Lushai [Mizo] &phů-\textbf{hlip} &scale (fish) &\mbox{AW-TBT}:1028 &\hspace*{1ex}{\tiny m,\textasciitilde}\\
 &phu-\textbf{hlip} &scales (of fish) &\mbox{JAM-Ety} &\hspace*{1ex}{\tiny m,\textasciitilde}\\
Moyon &ŋʌ\,\textbf{phrìp} &scales (of fish) &\mbox{DK-Moyon}:8.2.5 &\hspace*{1ex}{\tiny 1455,\textasciitilde}\\
Tiddim &\textbf{lip³} &scaly [as of fish] &\mbox{PB-TCV} &\hspace*{1ex}\\
\end{longtable}
1.3. Naga\\
\fascicletablebegin
Liangmei &ka-kha-\textbf{lêp} &scale (fish) &\mbox{AW-TBT}:1028 &\hspace*{1ex}{\tiny m,m,\textasciitilde}\\
Lotha &ó\,\textbf{lèp>} &skin of the fish &\mbox{A-LPD} &\hspace*{1ex}{\tiny p,\textasciitilde}\\
Rongmei &ka-\textbf{lip} &scale (fish) &\mbox{AW-TBT}:1028 &\hspace*{1ex}{\tiny m,\textasciitilde}\\
Tangkhul &¹ə\textbf{¹rip} &scale (fish) &\mbox{AW-TBT}:1028 &\hspace*{1ex}{\tiny p,\textasciitilde}\\
Zeme &¹he⁵ka\textbf{¹lip} &scale (fish) &\mbox{AW-TBT}:1028 &\hspace*{1ex}{\tiny m,m,\textasciitilde}\\
\end{longtable}
1.5. Mikir\\
\fascicletablebegin
Mikir &\textbf{lip} &scales (of fish) &\mbox{JAM-Ety} &\hspace*{1ex}\\
 &\textbf{lìp} &scales (of fish) &\mbox{KHG-Mikir}:184 &\hspace*{1ex}\\
\end{longtable}
2.3.2. Kiranti\\
\fascicletablebegin
Chamling &hu\,\textbf{lep}\,pa &skin &\mbox{WW-Cham}:15 &\hspace*{1ex}{\tiny 586,\textasciitilde,s}\\
\end{longtable}
}

\vspace{1em}
\addcontentsline{toc}{section}{(175)  \textbf{*sep} SCALE}
\markright{(175) [\#1454]  \textbf{*sep} SCALE}
{\large
\parindent=-1em
(175) \hspace{\stretch{1}} \textbf{*sep}\hspace{\stretch{1}}\textbf{SCALE} \textit{\tiny[\#1454]}}

Lepcha \textbf{a‑ší} looks like a possible reflex of this root, but the lack of a final \textbf{‑p} is a problem.

The final velar in the Limbu form is unexplained.



{\footnotesize
2.1.2. Bodic\\
\fascicletablebegin
Tsangla (Motuo) &\textbf{sep⁵⁵} &scales &\mbox{JZ-CLMenba} &\hspace*{1ex}\\
\end{longtable}
2.3.2. Kiranti\\
\fascicletablebegin
Limbu &\textbf{seːk} &scale (of a fish) &\mbox{BM-Lim} &\hspace*{1ex}\\
\end{longtable}
3.3. rGyalrongic\\
\fascicletablebegin
Ergong (Northern) &(ȵɛ¹³\textbf{) tʂʰɛp⁵³} &scales (of fish) &\mbox{SHK-ErgNQ}:8.2.5 &\hspace*{1ex}{\tiny 1455,\textasciitilde}\\
\end{longtable}
4.1. Jingpho\\
\fascicletablebegin
Jingpho &nga-\textbf{sep} &scales (of fish) &\mbox{JAM-Ety} &\hspace*{1ex}{\tiny 1455,\textasciitilde}\\
 &ŋá-\textbf{sèp} &scale (fish) &\mbox{JAM-TJLB}:341 &\hspace*{1ex}{\tiny 1455,\textasciitilde}\\
 &ə\textbf{sep} &scales (of fish) &\mbox{JAM-Ety} &\hspace*{1ex}{\tiny p,\textasciitilde}\\
\end{longtable}
4.2. Nungic\\
\fascicletablebegin
Dulong &ɹɯ⁵⁵\textbf{sɛʔ⁵⁵} &scale &\mbox{TBL}:0352.20 &\hspace*{1ex}{\tiny m,\textasciitilde}\\
Trung [Dulong] (Dulonghe) &ŋɑ⁵⁵ plɑ̆ʔ⁵⁵ ɹɯ³¹ \textbf{sĕʔ⁵⁵} &scales &\mbox{JZ-Dulong} &\hspace*{1ex}{\tiny 1455,m,m,\textasciitilde}\\
\end{longtable}
6.2. Loloish\\
\fascicletablebegin
Lahu (Black) &ɔ³¹ \textbf{sɛʔ⁵⁴} &scales &\mbox{JZ-Lahu} &\hspace*{1ex}{\tiny p,\textasciitilde}\\
 &ɔ̀-\textbf{šɛ̂ʔ} &scales (of fish); scale (fish) &\mbox{JAM-Ety}; \mbox{JAM-TJLB}:341 &\hspace*{1ex}{\tiny p,\textasciitilde}\\
Nusu (Central/Zhizhiluo) &kʰu̱³¹ \textbf{sɿ³¹} &scales &\mbox{DQ-NusuA}:409. &\hspace*{1ex}{\tiny 586,\textasciitilde}\\
Nusu (Central) &ŋa⁵⁵ \textbf{ʂi⁵⁵} &scales &\mbox{DQ-NusuB}:409. &\hspace*{1ex}{\tiny 1455,\textasciitilde}\\
Sani [Nyi] &\textbf{sa̱⁴⁴} &scales &\mbox{WAH-Sani}:164.1 &\hspace*{1ex}\\
 &ŋɑ⁵⁵ \textbf{sa⁵⁵} &scale (of fish) &\mbox{MXL-SaniQ}:328.2 &\hspace*{1ex}{\tiny 1455,\textasciitilde}\\
\end{longtable}
}

\vspace{1em}
\addcontentsline{toc}{section}{(176)  \textbf{*pra \STEDTU{⪤} *pya} SKIN}
\markright{(176) [\#792]  \textbf{*pra} \STEDTU{⪤} \textbf{*pya} SKIN}
{\large
\parindent=-1em
(176) \hspace{\stretch{1}} \textbf{*pra} \STEDTU{⪤} \textbf{*pya}\hspace{\stretch{1}}\textbf{SKIN} \textit{\tiny[\#792]}}

Tamangic points to a possible connection between this etymon and the concept FLAT. Cf.\ Thakali (Tukche) \textbf{'plja‑lɔ} ‘flatten dough (with roll)’.

The allofam with medial \textbf{‑y‑} is reconstructed because of vowel fronting in several forms (Pattani, Tshona, Lepcha, Thakali). There are two possible Chinese comparanda (see below).



{\footnotesize
1.1. North Assam\\
\fascicletablebegin
Idu &ki\,\textbf{pra} shrega &skin &\mbox{JP-Idu} &\hspace*{1ex}{\tiny m,\textasciitilde,m}\\
 &ko ko\,\textbf{pra} &leather &\mbox{JP-Idu} &\hspace*{1ex}{\tiny 589,589,\textasciitilde}\\
 &ko⁵⁵\textbf{pɹa⁵⁵} &skin &\mbox{SHK-Idu}:8.2 &\hspace*{1ex}{\tiny 589,\textasciitilde}\\
 &ko⁵⁵\textbf{pɹɑ⁵⁵} &skin &\mbox{ZMYYC}:266.50 &\hspace*{1ex}{\tiny 589,\textasciitilde}\\
 &ko\,li\,\textbf{pra} &scale(fish) &\mbox{JP-Idu} &\hspace*{1ex}{\tiny 589,562,\textasciitilde}\\
 &ko\,\textbf{pra} &skin; bark &\mbox{JP-Idu} &\hspace*{1ex}{\tiny 589,\textasciitilde}\\
Yidu &ko⁵⁵\textbf{pɹɑ⁵⁵} &skin &\mbox{TBL}:0120.25 &\hspace*{1ex}{\tiny 589,\textasciitilde}\\
\end{longtable}
2.1.1. Western Himalayish\\
\fascicletablebegin
Pattani [Manchati] &trà\,\textbf{pRi} &skin &\mbox{STP-ManQ}:8.2 &\hspace*{1ex}{\tiny m,\textasciitilde}\\
\end{longtable}
2.1.2. Bodic\\
\fascicletablebegin
Tshona (Wenlang) &\textbf{pʰiu⁵⁵} &skin &\mbox{JZ-CNMenba} &\hspace*{1ex}\\
\end{longtable}
2.1.3. Lepcha\\
\fascicletablebegin
Lepcha &mig-\textbf{be} &eyelid; edges of lid &\mbox{JAM-Ety} &\hspace*{1ex}{\tiny 682,\textasciitilde}\\
\end{longtable}
2.1.4. Tamangic\\
\fascicletablebegin
Gurung (Ghachok) &ʈu\,hbi \textbf{plaː}\,baq &skin &\mbox{SIL-Gur}:3.B.31 &\hspace*{1ex}{\tiny 585,590,\textasciitilde,m}\\
Thakali (Tukche) &ʈih \textbf{'pli}-lɔ &skin &\mbox{SIL-Thak}:3.B.31 &\hspace*{1ex}{\tiny 596,\textasciitilde,595}\\
\end{longtable}
2.3.2. Kiranti\\
\fascicletablebegin
Bahing &kheʔ\,\textbf{be}\,liŋ\,ma &skin &\mbox{BM-Bah} &\hspace*{1ex}{\tiny 586,\textasciitilde,3611,s}\\
\end{longtable}
3.2. Qiangic\\
\fascicletablebegin
Guiqiong &ȵi³⁵ \textbf{pɑ⁵⁵} \textbf{pɑ⁵⁵} &hide / leather &\mbox{SHK-GuiqQ} &\raisebox{-0.5ex}{\footnotemark}\footnotetext{First syllable means ‘cow’.}
{\tiny 2538,\textasciitilde,\textasciitilde}\\
Qiang (Mawo) &nə\,\textbf{ɹɑ} &eyelid &\mbox{SHK-MawoQ}:3.4.1 &\hspace*{1ex}{\tiny 681,\textasciitilde}\\
\end{longtable}
3.3. rGyalrongic\\
\fascicletablebegin
Ergong (Northern) &meɣ⁵³ \textbf{pe³³} &eyelid &\mbox{SHK-ErgNQ}:3.4.1 &\hspace*{1ex}{\tiny 682,\textasciitilde}\\
\end{longtable}
5. Tujia\\
\fascicletablebegin
Tujia &tʰa⁵⁵ \textbf{pʰa²¹} &hide; leather; skin &\mbox{CK-TujBQ}:8.2,8.2.6 &\hspace*{1ex}{\tiny m,\textasciitilde}\\
Tujia (Northern) &tʰa⁵⁵ \textbf{pʰa²¹} &skin &\mbox{JZ-Tujia} &\hspace*{1ex}{\tiny m,\textasciitilde}\\
\end{longtable}
6.1. Burmish\\
\fascicletablebegin
Burmese (Written) &(pûn-)liŋ-ʔəre-\textbf{prâ} &foreskin &\mbox{JAM-Ety} &\raisebox{-0.5ex}{\footnotemark}\footnotetext{First syllable means ‘hide (v.)’, i.e.\ “hidden skin”. Second syllable is a borrowing from Sanskrit \textbf{lingam} (private parts, penis).}
{\tiny m,b,596,\textasciitilde}\\
 &ʔə-re-\textbf{prâ} &skin &\mbox{JAM-Ety} &\hspace*{1ex}{\tiny p,596,\textasciitilde}\\
\end{longtable}
6.2. Loloish\\
\fascicletablebegin
Akha &\textbf{baq}-xoq &skin &\mbox{JAM-TSR}:71(a) &\hspace*{1ex}{\tiny \textasciitilde,586}\\
Sani [Nyi] &ɪ²¹ \textbf{pɪ³³} qɤ⁵⁵ tʂʅ³³ &skin of belly &\mbox{WAH-Sani}:313.4 &\hspace*{1ex}{\tiny m,\textasciitilde,596,m}\\
\end{longtable}
9. Sinitic\\
\fascicletablebegin
Chinese (Mandarin) &\textbf{pʼi} &skin &\mbox{GSR}:25a-c &\hspace*{1ex}\\
Chinese (Middle) &\textbf{pju} &skin &\mbox{WSC-SH}:134 &\hspace*{1ex}\\
Chinese (Old) &\textbf{b(r)jaj} &skin &\mbox{WHB-OC}:1057 &\hspace*{1ex}\\
 &\textbf{bʼia} &skin, peel, bark &\mbox{GHL-PPB}:H.125 &\hspace*{1ex}\\
 &\textbf{prja} &skin (human) &\mbox{WHB-OC}:1858 &\hspace*{1ex}\\
Chinese (GSR \#) &\textbf{bʼiɑ}/bʼjie̯ &skin, peel, bark &\mbox{GHL-PPB}:H.125 &\hspace*{1ex}\\
Chinese (Old/Mid) &\textbf{bʼiɑ}/bʼjie̯ &skin &\mbox{GSR}:25a-c &\hspace*{1ex}\\
\end{longtable}
}

{\large \textbf{Chinese comparandum}}

\TC{皮} OC b(r)jaj WHB-OC:1057, bʼiɑ/bʼjie̯ (\textit{GSR} 25a-c) > Mand.\ \textbf{pí}.

\TC{膚} MC pju WSC-SH:134, OC *prja WHB/*pljwo (\textit{GSR} 69g) > Mand.\ \textbf{fū}.

Which one fits better?

\vspace{1em}
\addcontentsline{toc}{section}{(177)  \textbf{*l-pak} SKIN}
\markright{(177) [\#588]  \textbf{*l-pak} SKIN}
{\large
\parindent=-1em
(177) \hspace{\stretch{1}} \textbf{*l-pak}\hspace{\stretch{1}}\textbf{SKIN} \textit{\tiny[\#588]}}

Several forms from Tibetan dialects show lenition of the initial to \textbf{w‑}. For the widespread TB variation between labial stops and \textbf{w‑}, see Matisoff 2000.



{\footnotesize
2.1.2. Bodic\\
\fascicletablebegin
Tshona (Mama) &\textbf{pʌː¹³} wo⁵³ &hide &\mbox{SLZO-MLD} &\raisebox{-0.5ex}{\footnotemark}\footnotetext{Cf.\ Spiti \textbf{pawo} ‘skin’.}
{\tiny \textasciitilde,s}\\
Tibetan (Alike) &\textbf{wok} kwa &skin (leather, hide) &\mbox{TBL}:0265.05 &\hspace*{1ex}{\tiny \textasciitilde,589}\\
Tibetan (Amdo:Bla-brang) &kha \textbf{xha} &lip &\mbox{ZMYYC}:243.4 &\hspace*{1ex}{\tiny 466,\textasciitilde}\\
 &\textbf{wak} kwa &skin &\mbox{ZMYYC}:266.4 &\hspace*{1ex}{\tiny \textasciitilde,589}\\
Tibetan (Amdo:Zeku) &\textbf{wok} kwa &skin &\mbox{ZMYYC}:266.5 &\hspace*{1ex}{\tiny \textasciitilde,s}\\
Tibetan (Batang) &dʑɛ⁵⁵ \textbf{baʔ⁵³} &foreskin &\mbox{DQ-Batang}:10.3.3 &\hspace*{1ex}{\tiny 1284,\textasciitilde}\\
 &xhaʔ⁵⁵ \textbf{baʔ⁵⁵} &skin &\mbox{DQ-Batang}:8.2 &\hspace*{1ex}{\tiny 586,\textasciitilde}\\
 &zə¹³ taʔ⁵³ \textbf{baʔ⁵⁵} baʔ⁵³ &hide  /  leather &\mbox{DQ-Batang}:8.2.6 &\hspace*{1ex}{\tiny m,m,\textasciitilde,m}\\
 &pɑ⁵⁵\textbf{pɑʔ⁵³} &skin &\mbox{TBL}:0265.03 &\hspace*{1ex}{\tiny p,\textasciitilde}\\
Tibetan (Khams:Dege) &\textbf{pa⁵⁵}pa⁵³ &skin &\mbox{ZMYYC}:266.3 &\hspace*{1ex}{\tiny \textasciitilde,s}\\
Tibetan (Lhasa) &\textbf{pak⁵³}pa⁵³ &skin &\mbox{ZMYYC}:266.2 &\hspace*{1ex}{\tiny \textasciitilde,s}\\
 &\textbf{pak⁵⁵}pa⁵⁵ &skin &\mbox{TBL}:0265.02 &\hspace*{1ex}{\tiny \textasciitilde,s}\\
Spiti &\textbf{pa}\,wo &skin &\mbox{CB-SpitiQ}:8.2 &\hspace*{1ex}{\tiny \textasciitilde,s}\\
Tibetan (Spiti) &\textbf{pa}\,wo &skin &\mbox{SRS-PSS} &\hspace*{1ex}{\tiny \textasciitilde,s}\\
Tibetan (Written) &k'a-\textbf{lpags} gon̊-ma &upper lip &\mbox{JAM-Ety} &\hspace*{1ex}{\tiny 466,\textasciitilde,m,s}\\
 &kha-\textbf{lpags} &lip &\mbox{GEM-CNL} &\hspace*{1ex}{\tiny 466,\textasciitilde}\\
 &kʻa-\textbf{lpágs} &lip &\mbox{JAM-Ety} &\hspace*{1ex}{\tiny 466,\textasciitilde}\\
 &\textbf{lpags} &skin &\mbox{JAM-Ety}; \mbox{WSC-SH}:134 &\hspace*{1ex}\\
 &\textbf{lpags} pa &skin &\mbox{TBL}:0265.01 &\hspace*{1ex}{\tiny \textasciitilde,s}\\
 &mdun-\textbf{lpags} &foreskin (vulg.) &\mbox{JAM-Ety} &\raisebox{-0.5ex}{\footnotemark}\footnotetext{The first syllable means ‘front’.}
{\tiny 796,\textasciitilde}\\
 &\textbf{pags} pa &skin; skin, peel, bark; foreskin; skin, leather &\mbox{ZMYYC}:266.1; \mbox{GHL-PPB}:H.125; \mbox{JAM-Ety}; \mbox{ZLS-Tib}:2,49 &\hspace*{1ex}{\tiny \textasciitilde,s}\\
 &\textbf{pags}-rigs &skin, leather &\mbox{ZLS-Tib}:2 &\hspace*{1ex}{\tiny \textasciitilde,782}\\
 &\textbf{pags}\,pa &skin &\mbox{GEM-CNL} &\hspace*{1ex}{\tiny \textasciitilde,s}\\
 &sdoms-\textbf{lpags} &foreskin &\mbox{JAM-Ety} &\raisebox{-0.5ex}{\footnotemark}\footnotetext{The first syllable seems to be derived from \textbf{dom} ‘private’.}
{\tiny m,\textasciitilde}\\
 &z̀al-\textbf{lpags} &lip &\mbox{JAM-Ety} &\raisebox{-0.5ex}{\footnotemark}\footnotetext{\textbf{z̀al} is the honorific form of \textbf{kha} ‘mouth’.}
{\tiny 1190,\textasciitilde}\\
 &ɕa \textbf{pags} &skin &\mbox{TBL}:0120.01 &\hspace*{1ex}{\tiny 34,\textasciitilde}\\
\end{longtable}
3.2. Qiangic\\
\fascicletablebegin
Guiqiong &khɑ⁵⁵ \textbf{pɑ⁵⁵} &lip &\mbox{SHK-GuiqQ} &\raisebox{-0.5ex}{\footnotemark}\footnotetext{Tibetan loan; cf.\ WT \textbf{kha‑lpags}.}
{\tiny 466,\textasciitilde}\\
 &\textbf{pa⁵⁵}pa⁵⁵ &skin &\mbox{TBL}:0265.16 &\raisebox{-0.5ex}{\footnotemark}\footnotetext{This apparently reduplicated form is actually a Tibetan loan; cf.\ WT \textbf{lpags‑pa}.}
{\tiny \textasciitilde,s}\\
\end{longtable}
6.2. Loloish\\
\fascicletablebegin
Hani (Shuikui) &mɛ³¹ \textbf{pʰa³¹} &lip &\mbox{JZ-Hani} &\hspace*{1ex}{\tiny 467,\textasciitilde}\\
\end{longtable}
}

\vspace{1em}
\addcontentsline{toc}{section}{(178)  \textbf{*p(y)ik} SKIN / PEEL}
\markright{(178) [\#590]  \textbf{*p(y)ik} SKIN / PEEL}
{\large
\parindent=-1em
(178) \hspace{\stretch{1}} \textbf{*p(y)ik}\hspace{\stretch{1}}\textbf{SKIN / PEEL} \textit{\tiny[\#590]}}

Several Bodo-Garo forms begin with a syllable \textbf{bi(ŋ)}, which superficially resembles this root. However, R. Burling points out that this is merely a prefix which occurs with many body part terms, e.g.\ Garo \textbf{bi‑bik} ‘intestines’, \textbf{bi‑bil} ‘afterbirth’, \textbf{bi‑gil} ‘skin, bark, peel, pod, leather’, \textbf{bi‑gron} ‘testicle’, \textbf{bi‑ka} ‘liver’, \textbf{bi‑kit} ‘gall bladder’, \textbf{bi‑mang} ‘body of person or thing; shape, form; middle portion of an object, such as a glass’, \textbf{bi‑rot} ‘pimple, boil, pox’; also Meche \textbf{biŋgur}, \textbf{bigur} ‘skin’. See Burling 2003.

The first syllables of Tshona (Mama) \textbf{phe⁵⁵khu⁵³} ‘skin’ and WT \textbf{pʻyi srin} ‘vermin living on skin’ also superficially resemble this etymon, but are really a different morpheme meaning ‘outside’. See Jäschke (1881:348-50). Thus, ‘vermin living on skin’ means lit. “outside vermin”.

A few Karenic forms (Pa-O \textbf{ ʼbeŋ¹},  Pho (Delta) \textbf{ɓẽ⁴}, Sgaw \textbf{ɓẽ⁴} ‘skin, bark’) point to a final nasal, but these could well reflect a different etymon. The morphemic identity of the second syllable of Karen (Sgaw/Hinthada) \textbf{a³¹ be̱³¹} and \textbf{pʰi⁵⁵ be̱³¹} ‘skin’ remains obscure.



{\footnotesize
1. Kamarupan\\
\fascicletablebegin
Miji &\textbf{pʰriʔ} &skin &\mbox{IMS-Miji} &\hspace*{1ex}\\
\end{longtable}
1.1. North Assam\\
\fascicletablebegin
Idu &a\,ka\,\textbf{bi} &lip(lower) &\mbox{JP-Idu} &\hspace*{1ex}{\tiny p,466,\textasciitilde}\\
\end{longtable}
1.6. Mru\\
\fascicletablebegin
Mru &\textbf{pei²} &skin &\mbox{GHL-PPB}:Q.68 &\hspace*{1ex}\\
 &\textbf{pei⁴} &skin &\mbox{GHL-PPB}:Q.68 &\hspace*{1ex}\\
 &\textbf{pik} &skin &\mbox{JAM-Ety} &\hspace*{1ex}\\
\end{longtable}
2.1.4. Tamangic\\
\fascicletablebegin
Gurung (Ghachok) &\textbf{piq}\,ba &peel &\mbox{SIL-Gur}:7.B.1.13 &\hspace*{1ex}{\tiny \textasciitilde,s}\\
 &pʰi \textbf{piq}\,ba &peel &\mbox{SIL-Gur}:7.B.1.14 &\raisebox{-0.5ex}{\footnotemark}\footnotetext{The first syllable means ‘peel, bark (n.)’.}
{\tiny m,\textasciitilde,s}\\
 &Tuh\textbf{bi} &skin &\mbox{JAM-Ety} &\hspace*{1ex}{\tiny 585,\textasciitilde}\\
 &³ʈu\,\textbf{bi} &skin &\mbox{MM-K78}:42 &\hspace*{1ex}{\tiny 585,\textasciitilde}\\
 &ʈuh\,\textbf{bi} &skin &\mbox{SIL-Gur}:1.28 &\hspace*{1ex}{\tiny 585,\textasciitilde}\\
 &ʈu\,\textbf{hbi} plaː\,baq &skin &\mbox{SIL-Gur}:3.B.31 &\hspace*{1ex}{\tiny 585,\textasciitilde,792,m}\\
Manang (Gyaru) &\textbf{piː¹} ba &peel &\mbox{YN-Man}:166 &\hspace*{1ex}{\tiny \textasciitilde,s}\\
Manang (Prakaa) &\textbf{¹pʰiː}- &peel &\mbox{HM-Prak}:0343 &\hspace*{1ex}\\
Tamang (Risiangku) &²miː-\textbf{¹pʰiː} &eyelid &\mbox{MM-TamRisQ}:3.4.1 &\hspace*{1ex}{\tiny 682,\textasciitilde}\\
 &¹suŋ-\textbf{¹pʰiː} &lip &\mbox{MM-TamRisQ}:3.9 &\hspace*{1ex}{\tiny 477,\textasciitilde}\\
Tamang (Sahu) &\textbf{'pʰiː}-pa &peel, sharpen (with knife) &\mbox{SIL-Sahu}:18.A.13 &\hspace*{1ex}{\tiny \textasciitilde,s}\\
 &sung \textbf{phi:} &lip &\mbox{JAM-Ety} &\hspace*{1ex}{\tiny 477,\textasciitilde}\\
\end{longtable}
2.3.2. Kiranti\\
\fascicletablebegin
Bantawa &do-\textbf{pheʔ}\,wa &lip &\mbox{WW-Bant}:24; \mbox{JAM-Ety} &\hspace*{1ex}{\tiny 442,\textasciitilde,s}\\
\end{longtable}
3.2. Qiangic\\
\fascicletablebegin
Ersu &ndʐo⁵⁵\textbf{pi⁵⁵} &skin &\mbox{ZMYYC}:266.18 &\hspace*{1ex}{\tiny 596,\textasciitilde}\\
Lyuzu &ngaʴ³³\textbf{pi⁵³} &skin &\mbox{TBL}:0120.18 &\hspace*{1ex}{\tiny 596,\textasciitilde}\\
Qiang (Mawo) &dʐa \textbf{pi} &hide  /  leather (dried animal s &\mbox{SHK-MawoQ}:8.2.6 &\hspace*{1ex}{\tiny 448,\textasciitilde}\\
 &nə ɹə \textbf{pi} &skin &\mbox{ZMYYC}:266.8; \mbox{JZ-Qiang} &\hspace*{1ex}{\tiny m,596,\textasciitilde}\\
 &qə pa\,tʂ ɹa\,\textbf{bi} &scalp &\mbox{SHK-MawoQ}:2.8 &\hspace*{1ex}{\tiny 1224,m,m,596,\textasciitilde}\\
 &ɹa\,\textbf{pi} &skin &\mbox{SHK-MawoQ}:8.2 &\hspace*{1ex}{\tiny 596,\textasciitilde}\\
 &ɹə\,\textbf{pi} &skin &\mbox{JS-Mawo} &\hspace*{1ex}{\tiny 596,\textasciitilde}\\
 &ɹɛ \textbf{piɛ} &skin &\mbox{TBL}:0265.08 &\hspace*{1ex}{\tiny 596,\textasciitilde}\\
 &ʴuɛ \textbf{piɛ} &skin &\mbox{TBL}:0120.08 &\hspace*{1ex}{\tiny 591,\textasciitilde}\\
Qiang (Yadu) &ʐe \textbf{pi} &skin (animal) &\mbox{DQ-QiangN}:331 &\hspace*{1ex}{\tiny 596,\textasciitilde}\\
Ersu &ndʐo⁵⁵ \textbf{pi⁵⁵} &skin &\mbox{SHK-ErsCQ} &\hspace*{1ex}{\tiny 596,\textasciitilde}\\
 &sɿ⁵⁵ mphɑ⁵⁵ ndʐo⁵⁵ \textbf{pi⁵⁵} &lip &\mbox{SHK-ErsCQ} &\raisebox{-0.5ex}{\footnotemark}\footnotetext{\textbf{sɿ⁵⁵ mphɑ⁵⁵} ‘mouth’ + \textbf{ndʐo⁵⁵ pi⁵⁵} ‘skin’.}
{\tiny 686,2098,596,\textasciitilde}\\
\end{longtable}
4.1. Jingpho\\
\fascicletablebegin
Jili &mə\,\textbf{phik} &skin, bark &\mbox{STC}:145n392 &\hspace*{1ex}{\tiny p,\textasciitilde}\\
Jingpho &a³¹ \textbf{phjiʔ³¹} &skin &\mbox{JZ-Jingpo} &\hspace*{1ex}{\tiny p,\textasciitilde}\\
 &\textbf{hpi} &skin &\mbox{GEM-CNL} &\hspace*{1ex}\\
 &\textbf{hpyi} &skin &\mbox{JAM-Ety} &\hspace*{1ex}\\
 &myìʔ-\textbf{hpyi} &eye socket / eyelid &\mbox{JAM-Ety} &\hspace*{1ex}{\tiny 682,\textasciitilde}\\
 &ne³¹\textbf{phjiʔ³¹} &foreskin / prepuce &\mbox{JCD} &\hspace*{1ex}{\tiny 545,\textasciitilde}\\
 &\textbf{phjiʔ³¹} &skin &\mbox{JZ-Jingpo}; \mbox{TBL}:0265.19; \mbox{ZMYYC}:266.47 &\hspace*{1ex}\\
 &ə\textbf{hpyi} &skin &\mbox{JAM-Ety} &\hspace*{1ex}{\tiny p,\textasciitilde}\\
\end{longtable}
6.2. Loloish\\
\fascicletablebegin
Gazhuo &ni³¹ na³²³ \textbf{piɛ⁵⁵} &lip &\mbox{DQ-Gazhuo}:3.9 &\hspace*{1ex}{\tiny m,m,\textasciitilde}\\
Lalo &m̩²¹ lɑ³³ \textbf{pʰɪ̱³³} &lip &\mbox{CK-YiQ}:3.9 &\hspace*{1ex}{\tiny 467,456,\textasciitilde}\\
Lolopho &v̩⁵⁵ \textbf{phi³¹} &scalp &\mbox{DQ-Lolopho}:2.8 &\hspace*{1ex}{\tiny 385,\textasciitilde}\\
Nesu &ȵɪ̱²¹ \textbf{pe̱³³} &lip &\mbox{CK-YiQ}:3.9 &\hspace*{1ex}{\tiny 467,\textasciitilde}\\
Nusu (Southern) &n̥i³¹ \textbf{bə⁵⁵} &lip &\mbox{JZ-Nusu} &\hspace*{1ex}{\tiny 467,\textasciitilde}\\
Sani [Nyi] &n̩⁵⁵ \textbf{pʰɿ³³} &lip &\mbox{MXL-SaniQ}:322.6 &\raisebox{-0.5ex}{\footnotemark}\footnotetext{\textit{original gloss: koupi}}
{\tiny m,\textasciitilde}\\
Yi (Mojiang) &dʑi⁵⁵\textbf{phi⁵⁵} &skin &\mbox{ZMYYC}:266.26 &\hspace*{1ex}{\tiny 596,\textasciitilde}\\
Yi (Nanjian) &mu̪²¹ la³³ \textbf{pʰi²¹} &lip &\mbox{JZ-Yi} &\hspace*{1ex}{\tiny 467,456,\textasciitilde}\\
\end{longtable}
6.3. Naxi\\
\fascicletablebegin
Naxi (Eastern) &nv³³\textbf{bi³³}li⁵⁵ &lip &\mbox{JZ-Naxi} &\hspace*{1ex}{\tiny 467,\textasciitilde,m}\\
Naxi (Western) &nv⁵⁵\textbf{bi³¹} &lip &\mbox{JZ-Naxi} &\hspace*{1ex}{\tiny 467,\textasciitilde}\\
Naxi (Lijiang) &ɣɯ³³\textbf{phi³¹} &skin &\mbox{ZMYYC}:266.28 &\hspace*{1ex}{\tiny 596,\textasciitilde}\\
\end{longtable}
7. Karenic\\
\fascicletablebegin
{}*Karen (Sgaw) &*\textbf{phiʔ} &skin &\mbox{RBJ-KLS}:554 &\hspace*{1ex}\\
{}*Karen (Pho) &*\textbf{phɛ́ʔ} &skin &\mbox{RBJ-KLS}:554 &\hspace*{1ex}\\
Bwe &\textbf{-phe} &skin, hide, bark, scales &\mbox{EJAH-BKD} &\hspace*{1ex}\\
 &\textbf{-phe}-kó &skin, outer covering &\mbox{EJAH-BKD} &\hspace*{1ex}{\tiny \textasciitilde,586}\\
 &\textbf{-phe}-kú &skin, outer covering &\mbox{EJAH-BKD} &\hspace*{1ex}{\tiny \textasciitilde,586}\\
 &ko-\textbf{phe} &skin of the scalp &\mbox{EJAH-BKD} &\hspace*{1ex}{\tiny 386,\textasciitilde}\\
Bwe (Western) &\textbf{pʻe²} &skin, peel, bark; skin, bark &\mbox{GHL-PPB}:H.125,J.242 &\hspace*{1ex}\\
 &\textbf{ɓe²} &skin, bark &\mbox{GHL-PPB}:J.242 &\hspace*{1ex}\\
Bwe &ə-\textbf{phe} də-blɔ̀ &skin / hide (portion of) &\mbox{EJAH-BKD} &\hspace*{1ex}{\tiny p,\textasciitilde,p,m}\\
 &ə-\textbf{phe} tə-ɓe &skin / hide (whole) &\mbox{EJAH-BKD} &\hspace*{1ex}{\tiny p,\textasciitilde,p,m}\\
Geba &\textbf{be²} &skin, bark &\mbox{GHL-PPB}:J.242 &\hspace*{1ex}\\
 &\textbf{pʻe²} &skin, bark &\mbox{GHL-PPB}:J.242 &\hspace*{1ex}\\
 &\textbf{pʻe³} &skin, peel, bark &\mbox{GHL-PPB}:H.125 &\hspace*{1ex}\\
 &θi¹ \textbf{pʻi²} &skin, bark &\mbox{GHL-PPB}:J.242 &\hspace*{1ex}{\tiny 2658,\textasciitilde}\\
Karen &a³¹\textbf{be̱³¹} &skin &\mbox{TBL}:0120.50 &\hspace*{1ex}{\tiny p,\textasciitilde}\\
 &\textbf{phi⁵⁵}be̱³¹ &skin &\mbox{TBL}:0265.50 &\hspace*{1ex}{\tiny \textasciitilde,m}\\
Pa-O (Northern) &ă\,\textbf{pʻeʔ³} &skin, peel, bark &\mbox{GHL-PPB}:H.125 &\hspace*{1ex}{\tiny p,\textasciitilde}\\
Pa-O &\textbf{pì} &skin of human, fruit; skin &\mbox{DBS-PaO}; \mbox{JAM-Ety}; \mbox{RBJ-KLS}:554 &\hspace*{1ex}\\
 &\textbf{pi} &skin, bark &\mbox{STC}:145n393 &\hspace*{1ex}\\
Pa-O (Northern) &\textbf{ʼbeŋ¹} &skin, bark &\mbox{GHL-PPB}:J.242 &\hspace*{1ex}\\
Paku &\textbf{pʻi¹} &skin, peel, bark &\mbox{GHL-PPB}:H.125 &\hspace*{1ex}\\
 &\textbf{ɓe³} &skin, bark &\mbox{GHL-PPB}:J.242 &\hspace*{1ex}\\
Pho &\textbf{phiʔ} &skin, bark &\mbox{STC}:145n393 &\hspace*{1ex}\\
Pho (Delta) &\textbf{pʻæʔ¹} &skin, peel, bark &\mbox{GHL-PPB}:H.125 &\hspace*{1ex}\\
 &\textbf{ɓẽ⁴} &skin, bark &\mbox{GHL-PPB}:J.242 &\hspace*{1ex}\\
Pho (Tenasserim) &ə̆\,\textbf{pʻɑiʔ³} &skin, peel, bark &\mbox{GHL-PPB}:H.125 &\hspace*{1ex}{\tiny p,\textasciitilde}\\
Pho (Bassein) &\textbf{phɛ̀ʔ} &skin &\mbox{JAM-Ety}; \mbox{RBJ-KLS}:554 &\hspace*{1ex}\\
Pho (Moulmein) &\textbf{phàiʔ} &skin &\mbox{JAM-Ety}; \mbox{RBJ-KLS}:554 &\hspace*{1ex}\\
Karen (Sgaw) &\textbf{phiʔ} &skin, bark &\mbox{STC}:145n393 &\hspace*{1ex}\\
Sgaw &\textbf{pʻiʔ²} &skin, peel, bark &\mbox{GHL-PPB}:H.125 &\hspace*{1ex}\\
 &\textbf{ɓẽ⁴} &skin, bark &\mbox{GHL-PPB}:J.242 &\hspace*{1ex}\\
Sgaw (Bassein) &\textbf{phiʔ} &skin &\mbox{JAM-Ety}; \mbox{RBJ-KLS}:554 &\hspace*{1ex}\\
Karen (Sgaw/Hinthada) &\textbf{pʰi⁵⁵} be̱³¹ &skin; skin (animal) &\mbox{DQ-KarenB}:150.1,315 &\hspace*{1ex}{\tiny \textasciitilde,m}\\
Sgaw (Moulmein) &\textbf{phiʔ} &skin &\mbox{JAM-Ety}; \mbox{RBJ-KLS}:554 &\hspace*{1ex}\\
Karen (Sgaw/Yue) &\textbf{pʰi̱ʔ⁵⁵} &skin; skin (animal) &\mbox{DQ-KarenA}:150,331 &\hspace*{1ex}\\
\end{longtable}
8. Bai\\
\fascicletablebegin
Bai (Dali) &\textbf{pe 7} &skin, peel (n) &\mbox{Dell-Bai}:303 &\hspace*{1ex}\\
 &\textbf{pe²¹} &skin &\mbox{ZMYYC}:266.35 &\hspace*{1ex}\\
 &\textbf{pe̱²¹} &skin &\mbox{JZ-Bai} &\hspace*{1ex}\\
 &\textbf{pe⁷} &skin &\mbox{FD-Bai}:pp.150-169 &\hspace*{1ex}\\
 &tɕui³³ \textbf{pe̱²¹} &lip &\mbox{JZ-Bai} &\raisebox{-0.5ex}{\footnotemark}\footnotetext{The first syllables of the Dali and Jianchuan compounds are undoubtedly loans from Chinese \TC{嘴} Mand.\ \textbf{zuǐ}.}
{\tiny 442,\textasciitilde}\\
 &ue³ \textbf{pe⁷} &eyelid &\mbox{FD-Bai}:pp.150-169 &\hspace*{1ex}{\tiny 33,\textasciitilde}\\
Bai (Jianchuan) &\textbf{pe²¹} &skin &\mbox{ZMYYC}:266.36 &\hspace*{1ex}\\
 &\textbf{pe̱²¹} &skin &\mbox{JZ-Bai}; \mbox{ZYS-Bai}:8.2 &\hspace*{1ex}\\
 &\textbf{pe⁴²} &skin &\mbox{TBL}:0265.48 &\hspace*{1ex}\\
 &tɕui³³ \textbf{pe̱²¹} &lip &\mbox{JZ-Bai} &\hspace*{1ex}{\tiny 442,\textasciitilde}\\
 &ŋuĩ³³ \textbf{pe̱²¹} &eyelid &\mbox{ZYS-Bai}:3.4.1 &\hspace*{1ex}{\tiny 33,\textasciitilde}\\
\end{longtable}
}

\vspace{1em}
\addcontentsline{toc}{section}{(179)  \textbf{*bak} STRIP OFF}
\markright{(179) [\#3000]  \textbf{*bak} STRIP OFF}
{\large
\parindent=-1em
(179) \hspace{\stretch{1}} \textbf{*bak}\hspace{\stretch{1}}\textbf{STRIP OFF} \textit{\tiny[\#3000]}}

This root is so far sparsely attested, with support only from Lahu and Chinese. The voiced Lahu initial points to a PLB prenasalized consonant \textbf{*mb‑}.



{\footnotesize
6.2. Loloish\\
\fascicletablebegin
Lahu (Black) &\textbf{bàʔ} &slice/cut into lengths &\mbox{JAM-DL}:933 &\hspace*{1ex}\\
 &\textbf{bàʔ} bà ve &slice off (as a strip of bark) &\mbox{JAM-DL}:933 &\raisebox{-0.5ex}{\footnotemark}\footnotetext{The second morpheme in the collocation means ‘throw away’ in isolation, but as an auxiliary verb means ‘to perform an action so that something is gotten rid of’. See \textit{HPTB} p.\ 231.}
{\tiny \textasciitilde,m,s}\\
\end{longtable}
9. Sinitic\\
\fascicletablebegin
Chinese (Old/Mid) &\textbf{pŭk} &cut, flay, peel; lay bare &\mbox{GSR}:1228a &\hspace*{1ex}\\
\end{longtable}
}

{\large \textbf{Chinese comparandum}}

\TC{剥}

\vspace{1em}
\addcontentsline{toc}{section}{(180)  \textbf{*phoo} RIND / SHIELD / SKIN (n.)}
\markright{(180) [\#4353]  \textbf{*phoo} RIND / SHIELD / SKIN (n.)}
{\large
\parindent=-1em
(180) \hspace{\stretch{1}} \textbf{*phoo}\hspace{\stretch{1}}\textbf{RIND / SHIELD / SKIN (n.)} \textit{\tiny[\#4353]}}

KVB-PKC \#353



{\footnotesize
1.2. Kuki-Chin\\
\fascicletablebegin
Lai (Falam) &\textbf{phôo} &rind, shield, animal skin &\mbox{KVB-PKC}:353 &\hspace*{1ex}\\
Lai (Hakha) &\textbf{phòo} &rind, shield, animal skin &\mbox{KVB-PKC}:353 &\hspace*{1ex}\\
 &sa-\textbf{phɔ} &skin &\mbox{JAM-Ety} &\hspace*{1ex}{\tiny 34,\textasciitilde}\\
Lushai [Mizo] &\textbf{phâw} &a shield, the long feathers or ruff round a cock's neck which stand out like a shield when angry &\mbox{KVB-PKC}:353 &\hspace*{1ex}\\
Paite &\textbf{phaw} &rind, shield &\mbox{KVB-PKC}:353 &\hspace*{1ex}\\
\end{longtable}
}

\vspace{1em}
\addcontentsline{toc}{section}{(180)  \textbf{*paw} RIND / SHIELD / SKIN (n.)}
\markright{(180) [\#784]  \textbf{*paw} RIND / SHIELD / SKIN (n.)}
{\large
\parindent=-1em
(180) \hspace{\stretch{1}} \textbf{*paw}\hspace{\stretch{1}}\textbf{RIND / SHIELD / SKIN (n.)} \textit{\tiny[\#784]}}

This root occurs mostly in Kuki-Chin (cf.\ PKC \textbf{*phoo} VanBik 2009:\#353), but also apparently in Nungic and Loloish. The semantics of this root are interesting, since the skin is here conceived of as a protection or shield from the outside world.



{\footnotesize
1.5. Mikir\\
\fascicletablebegin
Mikir &\textbf{phú}-rēng &skin &\mbox{KHG-Mikir}:146 &\hspace*{1ex}{\tiny \textasciitilde,781}\\
\end{longtable}
4.2. Nungic\\
\fascicletablebegin
Nung &sɑ⁵⁵\textbf{pho³¹} &skin &\mbox{TBL}:0120.21 &\hspace*{1ex}{\tiny 34,\textasciitilde}\\
Anong &sɑ⁵⁵\textbf{pho³¹} &skin &\mbox{ZMYYC}:266.44 &\hspace*{1ex}{\tiny 34,\textasciitilde}\\
\end{longtable}
6.2. Loloish\\
\fascicletablebegin
Gazhuo &sa³¹\textbf{pɤ³¹} &skin &\mbox{TBL}:0265.47 &\hspace*{1ex}{\tiny 34,\textasciitilde}\\
Sani [Nyi] &ni⁵⁵\textbf{phu³³} &lip &\mbox{YHJC-Sani} &\hspace*{1ex}{\tiny 467,\textasciitilde}\\
 &n̩⁵⁵ \textbf{pʰv̩³³} &lip &\mbox{MXL-SaniQ}:306.3 &\hspace*{1ex}{\tiny 467,\textasciitilde}\\
 &n̩⁵⁵\textbf{phu³³} &lip &\mbox{CK-YiQ}:3.9 &\hspace*{1ex}{\tiny 467,\textasciitilde}\\
Yi (Xide) &mi²¹ \textbf{pu²¹} &lip &\mbox{JZ-Yi} &\hspace*{1ex}{\tiny 467,\textasciitilde}\\
\end{longtable}
}

\vspace{1em}
\addcontentsline{toc}{section}{(181)  \textbf{*s-ryo} SKIN}
\markright{(181) [\#599]  \textbf{*s-ryo} SKIN}
{\large
\parindent=-1em
(181) \hspace{\stretch{1}} \textbf{*s-ryo}\hspace{\stretch{1}}\textbf{SKIN} \textit{\tiny[\#599]}}

This root has been reconstructed for Proto-Tani (J. Sun 1993), and receives support from the Chepang form.



{\footnotesize
1.1. North Assam\\
\fascicletablebegin
{}*Tani &*\textbf{rjo} &skin &\mbox{JS-HCST}:367 &\hspace*{1ex}\\
Padam [Abor] &a-\textbf{jo} &skin &\mbox{JS-HCST} &\hspace*{1ex}{\tiny p,\textasciitilde}\\
Padam-Mising [Abor-Miri] &a-\textbf{yo} &skin &\mbox{JAM-Ety} &\hspace*{1ex}{\tiny p,\textasciitilde}\\
Apatani &a-\textbf{ljo} &skin &\mbox{JS-HCST}; \mbox{JS-Tani} &\hspace*{1ex}{\tiny p,\textasciitilde}\\
 &à-\textbf{ljò} &skin &\mbox{JS-Tani} &\hspace*{1ex}{\tiny p,\textasciitilde}\\
 &a-mì mi-\textbf{ljo} &eyelid &\mbox{JS-Tani} &\hspace*{1ex}{\tiny p,682,682,\textasciitilde}\\
 &mí-\textbf{ljo} &eyelid &\mbox{JS-Tani} &\hspace*{1ex}{\tiny 682,\textasciitilde}\\
 &miʔ-\textbf{ljo} &eyelid &\mbox{JS-Tani} &\hspace*{1ex}{\tiny 682,\textasciitilde}\\
\end{longtable}
2.3.1. Kham-Magar-Chepang-Sunwar\\
\fascicletablebegin
Chepang &\textbf{hlyu}-sa &skin &\mbox{SIL-Chep}:3.B.31 &\hspace*{1ex}{\tiny \textasciitilde,34}\\
\end{longtable}
}

\vspace{1em}
\addcontentsline{toc}{section}{(182)  \textbf{*s-yam} SKIN}
\markright{(182) [\#795]  \textbf{*s-yam} SKIN}
{\large
\parindent=-1em
(182) \hspace{\stretch{1}} \textbf{*s-yam}\hspace{\stretch{1}}\textbf{SKIN} \textit{\tiny[\#795]}}

Despite its rather limited distribution, this looks like a valid etymon.



{\footnotesize
2.1.1. Western Himalayish\\
\fascicletablebegin
Pattani [Manchati] &\textbf{cəm} &skin &\mbox{DS-Patt} &\hspace*{1ex}\\
\end{longtable}
2.3.2. Kiranti\\
\fascicletablebegin
Bantawa &\textbf{jim} &old skin &\mbox{NKR-Bant} &\hspace*{1ex}\\
Thulung &\textbf{ji} &slough skin (of a snake) &\mbox{NJA-Thulung} &\hspace*{1ex}\\
 &\textbf{jim} &slough skin (of a snake) &\mbox{NJA-Thulung} &\hspace*{1ex}\\
\end{longtable}
4.2. Nungic\\
\fascicletablebegin
Nung &\textbf{sɑm³¹} &skin &\mbox{TBL}:0265.21 &\hspace*{1ex}\\
\end{longtable}
6.1. Burmish\\
\fascicletablebegin
Burmese (Written) &ʔəpo \textbf{yam} &upper layer, outer surface &\mbox{AJ-BED} &\raisebox{-0.5ex}{\footnotemark}\footnotetext{According to Judson, WB \textbf{yam} is a loanword from Pali, meaning ‘division; watch of the night’ (cf.\ Lahu \textbf{yâ(n)} ‘time’ <~Shan <~Bs.), but the other Burmish forms as well as those from Himalayish languages suggest that this actually may reflect a PTB root.}
{\tiny m,\textasciitilde}\\
Langsu (Luxi) &ʃɔ̆³⁵\textbf{jam⁵⁵} &skin &\mbox{TBL}:0120.31 &\hspace*{1ex}{\tiny 34,\textasciitilde}\\
Maru [Langsu] &au³⁵ \textbf{jam⁵⁵} &scalp &\mbox{DQ-Langsu}:2.8 &\hspace*{1ex}{\tiny 385,\textasciitilde}\\
 &mjoʔ³¹ \textbf{jam⁵⁵} &eyelid &\mbox{DQ-Langsu}:3.4.1 &\hspace*{1ex}{\tiny 681,\textasciitilde}\\
 &sɔ̆³⁵\textbf{jam⁵⁵} &skin &\mbox{ZMYYC}:266.43 &\hspace*{1ex}{\tiny 34,\textasciitilde}\\
 &ʃɔ̆³⁵ \textbf{jam⁵⁵} &skin &\mbox{DQ-Langsu}:8.2 &\hspace*{1ex}{\tiny 34,\textasciitilde}\\
\end{longtable}
}

\vspace{1em}
\addcontentsline{toc}{section}{(183)  \textbf{*reŋ} SKIN}
\markright{(183) [\#781]  \textbf{*reŋ} SKIN}
{\large
\parindent=-1em
(183) \hspace{\stretch{1}} \textbf{*reŋ}\hspace{\stretch{1}}\textbf{SKIN} \textit{\tiny[\#781]}}

This rare root has so far only been attested in Mikir and Chepang. The semantic structure of the Mikir forms meaning ‘ear’ is unclear.



{\footnotesize
1.5. Mikir\\
\fascicletablebegin
Mikir &a\,no ar\,lo a\,\textbf{reng} &ear &\mbox{JAM-Ety} &\hspace*{1ex}{\tiny p,811,p,m,p,\textasciitilde}\\
 &bùm a-\textbf{rèng} &foreskin &\mbox{KHG-Mikir}:160 &\raisebox{-0.5ex}{\footnotemark}\footnotetext{First syllable means “penis”.}
{\tiny m,p,\textasciitilde}\\
 &bum a\,\textbf{reng} &foreskin &\mbox{JAM-Ety} &\hspace*{1ex}{\tiny m,p,\textasciitilde}\\
 &bum a\,\textbf{reng} kerot &circumcise &\mbox{JAM-Ety} &\hspace*{1ex}{\tiny m,p,\textasciitilde,m}\\
 &cheng a\,\textbf{reng} &ear &\mbox{JAM-Ety} &\hspace*{1ex}{\tiny m,p,\textasciitilde}\\
 &mék a-\textbf{rèng} &eyelid &\mbox{KHG-Mikir}:167 &\hspace*{1ex}{\tiny 682,p,\textasciitilde}\\
 &mek a\,\textbf{reng} &eyelid &\mbox{JAM-Ety} &\hspace*{1ex}{\tiny 682,p,\textasciitilde}\\
 &mek kung a\,\textbf{reng} &eyelid &\mbox{JAM-Ety} &\hspace*{1ex}{\tiny 682,m,p,\textasciitilde}\\
 &mék-kúng a-\textbf{rèng} &eyelid &\mbox{KHG-Mikir}:168 &\hspace*{1ex}{\tiny 682,m,p,\textasciitilde}\\
 &òk-\textbf{rèng} &hide; leather; skin &\mbox{KHG-Mikir}:34,34 &\hspace*{1ex}{\tiny m,\textasciitilde}\\
 &phú-\textbf{rēng} &skin &\mbox{KHG-Mikir}:146 &\hspace*{1ex}{\tiny 784,\textasciitilde}\\
 &\textbf{reng} &skin &\mbox{GEM-CNL}; \mbox{JAM-Ety} &\hspace*{1ex}\\
 &\textbf{rèng} &skin &\mbox{KHG-Mikir}:192 &\hspace*{1ex}\\
\end{longtable}
2.3.1. Kham-Magar-Chepang-Sunwar\\
\fascicletablebegin
Chepang (Eastern) &təyŋ\,thyo\,\textbf{reŋ} &foreskin &\mbox{RC-ChepQ}:10.3.3 &\hspace*{1ex}{\tiny 3421,m,\textasciitilde}\\
\end{longtable}
}

\vspace{1em}
\addcontentsline{toc}{section}{(184)  \textbf{*rik-s} SKIN}
\markright{(184) [\#782]  \textbf{*rik-s} SKIN}
{\large
\parindent=-1em
(184) \hspace{\stretch{1}} \textbf{*rik-s}\hspace{\stretch{1}}\textbf{SKIN} \textit{\tiny[\#782]}}

The North Assam forms with initial sibilants perhaps point to a variant with prefixal \textbf{*s‑} which preempted the root initial.



{\footnotesize
1.1. North Assam\\
\fascicletablebegin
Padam-Mising [Abor-Miri] &a-\textbf{shik} &skin &\mbox{JAM-Ety} &\hspace*{1ex}{\tiny p,\textasciitilde}\\
Damu &mik-\textbf{ɕik} &eyelid &\mbox{JS-Tani} &\hspace*{1ex}{\tiny 682,\textasciitilde}\\
 &ʔa-\textbf{ɕik} &skin &\mbox{JS-Tani} &\hspace*{1ex}{\tiny p,\textasciitilde}\\
\end{longtable}
2.1.2. Bodic\\
\fascicletablebegin
Tibetan (Written) &pags-\textbf{rigs} &skin, leather &\mbox{ZLS-Tib}:2 &\hspace*{1ex}{\tiny 588,\textasciitilde}\\
\end{longtable}
2.3.2. Kiranti\\
\fascicletablebegin
Limbu &hok-\textbf{rik} &skin &\mbox{JAM-Ety} &\hspace*{1ex}{\tiny 586,\textasciitilde}\\
 &hoː\,\textbf{rik} &skin, bark; skin &\mbox{BM-Lim}; \mbox{BM-PK7}:157 &\hspace*{1ex}{\tiny 586,\textasciitilde}\\
 &[hoː\,\textbf{rik}\,pa] &skin &\mbox{AW-TBT}:170 &\hspace*{1ex}{\tiny 586,\textasciitilde,s}\\
 &[hoː\,\textbf{riʔ}\,pa] &skin &\mbox{AW-TBT}:170 &\hspace*{1ex}{\tiny 586,\textasciitilde,s}\\
 &[hu\,\textbf{rik]} &skin &\mbox{AW-TBT}:170 &\hspace*{1ex}{\tiny 586,\textasciitilde}\\
 &[hu\,\textbf{riʔ]} &skin &\mbox{AW-TBT}:170 &\hspace*{1ex}{\tiny 586,\textasciitilde}\\
Yakha &hɛː\,rana wa\,\textbf{rik} &hide  /  leather (dried animal s &\mbox{TK-Yakha}:8.2.6 &\hspace*{1ex}{\tiny m,m,589,\textasciitilde}\\
 &i\,ɦwa\,\textbf{rik} &skin &\mbox{AW-TBT}:170 &\hspace*{1ex}{\tiny p,589,\textasciitilde}\\
 &liː\,gəuː\,wə\,ha\,\textbf{rik} &foreskin &\mbox{TK-Yakha}:10.3.3 &\hspace*{1ex}{\tiny 1284,m,m,m,\textasciitilde}\\
 &miʔ wə\,ha\,\textbf{rik} &eyelid &\mbox{TK-Yakha}:3.4.1 &\hspace*{1ex}{\tiny 682,m,m,\textasciitilde}\\
 &uɦa\,\textbf{rik} &skin &\mbox{AW-TBT}:170 &\hspace*{1ex}{\tiny 589,\textasciitilde}\\
 &wəha\,\textbf{rik} &scalp &\mbox{TK-Yakha}:2.8 &\hspace*{1ex}{\tiny 589,\textasciitilde}\\
 &wəha\,\textbf{riːk} &skin &\mbox{TK-Yakha}:8.2 &\hspace*{1ex}{\tiny 589,\textasciitilde}\\
 &ɦwa\,\textbf{rik} &skin &\mbox{AW-TBT}:170 &\hspace*{1ex}{\tiny 589,\textasciitilde}\\
\end{longtable}
4.3. Luish\\
\fascicletablebegin
Lui &lök\,\textbf{rök} &skin &\mbox{JAM-Ety} &\hspace*{1ex}{\tiny 602,\textasciitilde}\\
 &\textbf{rak} &skin &\mbox{JAM-Ety} &\hspace*{1ex}\\
\end{longtable}
}

\vspace{1em}
\addcontentsline{toc}{section}{(185)  \textbf{*s-lwa} SKIN}
\markright{(185) [\#595]  \textbf{*s-lwa} SKIN}
{\large
\parindent=-1em
(185) \hspace{\stretch{1}} \textbf{*s-lwa}\hspace{\stretch{1}}\textbf{SKIN} \textit{\tiny[\#595]}}

This root resembles \textbf{\textit{\tiny[\#598]}} \textbf{*s‑lwap} (next etymon), and Phunoi seems to show variation between an open and a stopped reflex. We tentatively include the Phunoi forms under the open syllable etymon because Phunoi does preserve PTB \textbf{*‑p}.



{\footnotesize
2.1.4. Tamangic\\
\fascicletablebegin
Thakali (Tukche) &ʈih 'pli-\textbf{lɔ} &skin &\mbox{SIL-Thak}:3.B.31 &\hspace*{1ex}{\tiny 596,792,\textasciitilde}\\
\end{longtable}
4.3. Luish\\
\fascicletablebegin
Sak &ă\textbf{là³} &skin &\mbox{JAM-Ety} &\hspace*{1ex}{\tiny p,\textasciitilde}\\
\end{longtable}
5. Tujia\\
\fascicletablebegin
Tujia (Southern) &kɨ²¹ \textbf{lo²¹} &skin &\mbox{JZ-Tujia} &\hspace*{1ex}{\tiny m,\textasciitilde}\\
Tujia &kɯe²¹ \textbf{lo²¹} &hide; leather (dried animal skin); skin &\mbox{CK-TujMQ}:8.2,8.2.6 &\hspace*{1ex}{\tiny 589,\textasciitilde}\\
\end{longtable}
6.1. Burmish\\
\fascicletablebegin
Burmese (Written) &myak-\textbf{hlwa} &eyelid &\mbox{JAM-Ety} &\hspace*{1ex}{\tiny 681,\textasciitilde}\\
\end{longtable}
6.2. Loloish\\
\fascicletablebegin
Sani [Nyi] &\textbf{ɬu²¹} &skin &\mbox{WAH-Sani}:115.1 &\hspace*{1ex}\\
 &\textbf{ɬɿ¹¹} &skin &\mbox{MXL-SaniQ}:361.1 &\hspace*{1ex}\\
Phunoi &ʔãˊ \textbf{hlùʔ} &skin &\mbox{JAM-Ety} &\hspace*{1ex}{\tiny p,\textasciitilde}\\
 &ʔã́ \textbf{hlù} &skin &\mbox{DB-PLolo} &\hspace*{1ex}{\tiny p,\textasciitilde}\\
 &ʔɑ̃⁵⁵ \textbf{hluʔ¹¹} &skin &\mbox{DB-Phunoi} &\hspace*{1ex}{\tiny p,\textasciitilde}\\
\end{longtable}
}

\vspace{1em}
\addcontentsline{toc}{section}{(186)  \textbf{*s-lwap} SKIN}
\markright{(186) [\#598]  \textbf{*s-lwap} SKIN}
{\large
\parindent=-1em
(186) \hspace{\stretch{1}} \textbf{*s-lwap}\hspace{\stretch{1}}\textbf{SKIN} \textit{\tiny[\#598]}}

The Tamangic and Nungic forms show preemption by the \textbf{*s‑} prefix.



{\footnotesize
1.3. Naga\\
\fascicletablebegin
Wancho &ho-\textbf{lop} &skin &\mbox{JAM-Ety} &\hspace*{1ex}{\tiny 589,\textasciitilde}\\
\end{longtable}
2.1.4. Tamangic\\
\fascicletablebegin
{}*Tamang &*\textbf{ᴬsjop} &snake skin &\mbox{MM-Thesis}:992 &\hspace*{1ex}\\
 &*\textbf{ᴬɕop} &snake skin &\mbox{MM-Thesis}:992 &\hspace*{1ex}\\
Tamang (Taglung) &\textbf{ʃjop} &snake skin &\mbox{MM-Thesis}:992 &\hspace*{1ex}\\
\end{longtable}
4.2. Nungic\\
\fascicletablebegin
Trung [Dulong] (Dulonghe) &kɔɹ⁵⁵ \textbf{sɔp⁵⁵} &scalp &\mbox{JZ-Dulong} &\hspace*{1ex}{\tiny m,\textasciitilde}\\
\end{longtable}
}

\vspace{1em}
\addcontentsline{toc}{section}{(187)  \textbf{*m-bwat} SKIN}
\markright{(187) [\#592]  \textbf{*m-bwat} SKIN}
{\large
\parindent=-1em
(187) \hspace{\stretch{1}} \textbf{*m-bwat}\hspace{\stretch{1}}\textbf{SKIN} \textit{\tiny[\#592]}}

The reconstruction of a nasal prefix in this poorly attested root is on the basis of the Muya form.



{\footnotesize
2.1.1. Western Himalayish\\
\fascicletablebegin
Bunan &\textbf{bat}\,si &hide; leather (dried animal skin); skin &\mbox{SBN-BunQ}:8.2,8.2.6 &\hspace*{1ex}{\tiny \textasciitilde,m}\\
Kanauri &\textbf{bod} &skin / bark &\mbox{DS-Kan}:25 &\hspace*{1ex}\\
 &\textbf{bodʻ} &skin of man, dogs, cats; peel / rind &\mbox{BAI1911} &\hspace*{1ex}\\
 &\textbf{bŏdʻ} &skin of people, dogs, cats &\mbox{JAM-Ety} &\hspace*{1ex}\\
 &\textbf{bodʻ} khōmigʻ &skin / take bark off tree or wand &\mbox{BAI1911} &\hspace*{1ex}{\tiny \textasciitilde,m}\\
 &\textbf{bɔd} &skin &\mbox{DS-Kan}:13 &\hspace*{1ex}\\
\end{longtable}
3.2. Qiangic\\
\fascicletablebegin
Muya [Minyak] &ʐɯ³⁵\textbf{mbɐ⁵³} &skin &\mbox{SHK-MuyaQ}:8.2 &\hspace*{1ex}{\tiny 596,\textasciitilde}\\
\end{longtable}
6.2. Loloish\\
\fascicletablebegin
Yi (Xide) &\textbf{bu̱³³}-ku̱³³ &skin; scales (fish) &\mbox{CSL-YIzd} &\hspace*{1ex}{\tiny \textasciitilde,586}\\
 &hɯ³³-\textbf{bu̱³³}-ku̱³³ &scales &\mbox{CSL-YIzd} &\hspace*{1ex}{\tiny m,\textasciitilde,586}\\
\end{longtable}
}

\vspace{1em}
\addcontentsline{toc}{section}{(188)  \textbf{*hooŋ} BARK (of tree) / COVER / SHELL}
\markright{(188) [\#4698]  \textbf{*hooŋ} BARK (of tree) / COVER / SHELL}
{\large
\parindent=-1em
(188) \hspace{\stretch{1}} \textbf{*hooŋ}\hspace{\stretch{1}}\textbf{BARK (of tree) / COVER / SHELL} \textit{\tiny[\#4698]}}

This etymon is so far only attested in Kuki-Chin. See KVB-PKC \#698.



{\footnotesize
1.2. Kuki-Chin\\
\fascicletablebegin
Lai (Hakha) &\textbf{hôoŋ} &bark (of tree), cover, shell &\mbox{KVB-Lai}:698 &\hspace*{1ex}\\
Lai (Falam) &\textbf{hǒoŋ} &bark, cover, shell &\mbox{KVB-PKC}:698 &\hspace*{1ex}\\
Lakher [Mara] &\textbf{hỳ} &the shell of an egg, the bark of a tree &\mbox{KVB-PKC}:698 &\hspace*{1ex}\\
Lushai [Mizo] &\textbf{háwng} &bark, shell (as of eggs, etc) &\mbox{KVB-PKC}:698 &\hspace*{1ex}\\
Tiddim &\textbf{ho:ng2} &bark (of a tree) &\mbox{KVB-PKC}:698 &\hspace*{1ex}\\
\end{longtable}
}

\vspace{1em}
\addcontentsline{toc}{section}{(189)  \textbf{*tsəw} SKIN}
\markright{(189) [\#790]  \textbf{*tsəw} SKIN}
{\large
\parindent=-1em
(189) \hspace{\stretch{1}} \textbf{*tsəw}\hspace{\stretch{1}}\textbf{SKIN} \textit{\tiny[\#790]}}

This tentatively reconstructed root is largely restricted to the final syllables of Loloish compounds, with a possible cognate from Kulung.

Baima \textbf{ʃhɑ⁵³ pɑ⁵³} ‘skin’ (SHK-BaimaQ) and Tamang (Sahu) \textbf{grahm‑pa‑sa} ‘lip’ and \textbf{sya} ‘skin’ (SIL-Sahu1.28, 2.24) look suspiciously like Baima \textbf{shɑ¹³ kɛ³⁵} ‘flesh, meat’ and Tamang (Sahu) \textbf{ra sya} ‘meat’, suggesting that these syllables derive from \textbf{(H:448)} \textbf{*sya} FLESH / MEAT / GAME ANIMAL, rather than from the present etymon.

Several forms with initial palatal sibilants are assigned by W. T. French to PNN \textbf{*C\STEDTU{Ⓥ}‑kʰuar}, which we in turn assign to \textbf{\textit{\tiny[\#593]}} \textbf{*gul} \STEDTU{⪤} \textbf{*gil} SKIN.



{\footnotesize
2.3.2. Kiranti\\
\fascicletablebegin
Kulung &\textbf{so}\,go &skin &\mbox{AW-TBT}:170 &\hspace*{1ex}{\tiny \textasciitilde,586}\\
 &\textbf{so}\,ko\,warə &skin &\mbox{RPHH-Kul} &\hspace*{1ex}{\tiny \textasciitilde,586,m}\\
\end{longtable}
6.2. Loloish\\
\fascicletablebegin
Lipho &dzi³³\textbf{tʂho³³} &skin &\mbox{CK-YiQ}:8.2 &\hspace*{1ex}{\tiny 596,\textasciitilde}\\
Nasu &ȵe̱⁵⁵ \textbf{tʂʰɵ³³} &lip &\mbox{CK-YiQ}:3.9 &\hspace*{1ex}{\tiny 467,\textasciitilde}\\
Nosu &ndʑɿ⁴⁴\textbf{sɯ³³} &skin &\mbox{CK-YiQ}:8.2 &\hspace*{1ex}{\tiny 596,\textasciitilde}\\
Nusu (Northern) &nu³¹ \textbf{ʂu⁵⁵} &lip &\mbox{JZ-Nusu} &\hspace*{1ex}{\tiny 467,\textasciitilde}\\
Nusu (Central) &n̥ɯ³¹ \textbf{sui⁵⁵} &lip &\mbox{JZ-Nusu} &\hspace*{1ex}{\tiny 467,\textasciitilde}\\
Nusu (Central/Zhizhiluo) &n̥ɯ⁵⁵ \textbf{sui⁵⁵} &lip &\mbox{DQ-NusuA}:108. &\hspace*{1ex}{\tiny 467,\textasciitilde}\\
Nusu (Central) &n̥ɯ⁵⁵ \textbf{ʂui⁵⁵} &lip &\mbox{DQ-NusuB}:108. &\hspace*{1ex}{\tiny 467,\textasciitilde}\\
Yi (Nanjian) &gɯ⁵⁵\textbf{tʂu̪²¹} &skin &\mbox{ZMYYC}:266.23 &\hspace*{1ex}{\tiny 596,\textasciitilde}\\
Yi (Xide) &ndʑɿ³⁴\textbf{ʂɯ³³} &skin &\mbox{TBL}:0120.35 &\hspace*{1ex}{\tiny 596,\textasciitilde}\\
 &ndʑɿ⁴⁴\textbf{ʂɯ³³} &skin &\mbox{ZMYYC}:266.21 &\hspace*{1ex}{\tiny 596,\textasciitilde}\\
 &n\,dʑɿ³⁴-\textbf{ʂɯ³³} &skin &\mbox{CSL-YIzd} &\hspace*{1ex}{\tiny p,596,\textasciitilde}\\
\end{longtable}
}

\vspace{1em}
\addcontentsline{toc}{section}{(190)  \textbf{*koŋ OR kwaŋ} SKIN}
\markright{(190) [\#780]  \textbf{*koŋ} OR kwaŋ SKIN}
{\large
\parindent=-1em
(190) \hspace{\stretch{1}} \textbf{*koŋ} OR kwaŋ\hspace{\stretch{1}}\textbf{SKIN} \textit{\tiny[\#780]}}

This root is distinct from \textbf{\textit{\tiny[\#593]}} \textbf{*gul} \STEDTU{⪤} \textbf{*gil} SKIN. Cf.\ Lalung \textbf{khongʔ} ‘skin’ (from the present etymon) vs.\ \textbf{kur} ‘skin’ and \textbf{mokun} ‘eyelid’ (from \textbf{*gul}). Neither does there seem to be a connection between this root and \textbf{\textit{\tiny[\#586]}} \textbf{*s/r‑kok} \STEDTU{⪤} \textbf{*(r‑)kwak} SKIN, BARK, RIND. The second syllable of Lahu \textbf{mə̂‑qɔ} ‘lip’ might well be derived from PLB \textbf{*ʔgaŋ}, but this syllable also appears in several other words (\textbf{mɔ̀ʔ‑qɔ} ‘mouth’, \textbf{làʔ=tɔ‑qɔ} ‘palm of the hand’, \textbf{mû‑lɔ̀ʔ‑qɔ} ‘daytime; afternoon’), and is glossed ‘noun formative of elusive meaning’ in Matisoff 1988:250.



{\footnotesize
1.2. Kuki-Chin\\
\fascicletablebegin
Anal &sa-\textbf{kɔŋ} &skin &\mbox{JAM-Ety} &\hspace*{1ex}{\tiny 34,\textasciitilde}\\
 &wɔ̀\,\textbf{kɔ̀ɔŋ} &skin &\mbox{AW-TBT}:170 &\hspace*{1ex}{\tiny 589,\textasciitilde}\\
\end{longtable}
1.3. Naga\\
\fascicletablebegin
Tangsa &¹ʌ\textbf{³khoŋ} &skin &\mbox{AW-TBT}:170 &\hspace*{1ex}{\tiny p,\textasciitilde}\\
Tangsa (Yogli) &a\,\textbf{khaung} &skin &\mbox{GEM-CNL} &\hspace*{1ex}{\tiny p,\textasciitilde}\\
\end{longtable}
1.7. Bodo-Garo = Barish\\
\fascicletablebegin
Lalung &\textbf{khongʔ} &skin &\mbox{MB-Lal}:15 &\hspace*{1ex}\\
\end{longtable}
6.1. Burmish\\
\fascicletablebegin
Burmese (Written) &myak-\textbf{khwam} &eyelid &\mbox{JAM-Ety} &\raisebox{-0.5ex}{\footnotemark}\footnotetext{The final labial in this form is perhaps due to assimilation to a preceding \textbf{‑w‑}.}
{\tiny 681,\textasciitilde}\\
\end{longtable}
}

\vspace{1em}
\addcontentsline{toc}{section}{(191)  \textbf{*taŋ OR tuŋ} SKIN}
\markright{(191) [\#785]  \textbf{*taŋ} OR tuŋ SKIN}
{\large
\parindent=-1em
(191) \hspace{\stretch{1}} \textbf{*taŋ} OR tuŋ\hspace{\stretch{1}}\textbf{SKIN} \textit{\tiny[\#785]}}

This rather poorly attested root is found in the second syllable of compounds in Bodic and Burmish, as well as in Kiranti, where it apparently occurs independently. No Written Tibetan cognate has yet been identified.



{\footnotesize
2.1.2. Bodic\\
\fascicletablebegin
Tsangla (Central) &khop\,\textbf{tang} &skin &\mbox{SER-HSL/T}:36  10 &\hspace*{1ex}{\tiny 783,\textasciitilde}\\
Tsangla (Motuo) &kʰop⁵⁵ \textbf{taŋ⁵⁵} &skin &\mbox{JZ-CLMenba} &\hspace*{1ex}{\tiny 783,\textasciitilde}\\
Tibetan (Jirel) &kok\textbf{tenq} &skin &\mbox{JAM-Ety} &\hspace*{1ex}{\tiny 586,\textasciitilde}\\
\end{longtable}
2.3.2. Kiranti\\
\fascicletablebegin
Bantawa &Na\,\textbf{TUN} &fish scale &\mbox{NKR-Bant} &\raisebox{-0.5ex}{\footnotemark}\footnotetext{The first syllable means ‘fish’.}
{\tiny 1455,\textasciitilde}\\
 &\textbf{TUN} &scale (fish, etc.) &\mbox{NKR-Bant} &\raisebox{-0.5ex}{\footnotemark}\footnotetext{This peculiar transcription with capital letters, used by Novel Kishore Rai, represents [\textbf{t̺ɯŋ}] (with an apico-alveolar initial).}
\\
\end{longtable}
6.1. Burmish\\
\fascicletablebegin
Achang (Longchuan) &ȵ̥ot⁵⁵ \textbf{tuŋ⁵⁵} &lip &\mbox{JZ-Achang} &\hspace*{1ex}{\tiny 471,\textasciitilde}\\
Achang (Luxi) &nut⁵⁵ \textbf{tɔŋ⁵¹} &lip &\mbox{JZ-Achang} &\hspace*{1ex}{\tiny 471,\textasciitilde}\\
\end{longtable}
}

