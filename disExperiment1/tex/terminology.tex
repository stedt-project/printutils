\vspace{0.25em}

\renewcommand{\thefootnote}{\arabic{footnote}}
\setcounter{footnote}{0}

\chapter*{Conventions}\markright{Conventions}
\addcontentsline{toc}{chapter}{Conventions}

\renewcommand\thefootnote{*}

\section*{Abbreviations for Languages Commonly Referred to in the text}
\begin{multicols}{2}
\begin{description}
\item[Gk.] Greek
\item[HM]	Hmong-Mien (= Miao-Yao)
\item[IA]	Indo-Aryan
\item[IE]	Indo-European
\item[Jg.]	Jingpho (= Kachin)
\item[KC]	Kuki-Chin
\item[LB]	Lolo-Burmese
\item[Lh.]	Lahu
\item[MC]	Middle Chinese
\item[NN]	Northern Naga
\item[NT]	Northern Tai
\item[OC]	Old Chinese
\item[PIE]	Proto-Indo-European
\item[PLB]	Proto-Lolo-Burmese
\item[PNC]	Proto-Northern Chin
\item[PNN]	Proto-Northern Naga
\item[PST]	Proto-Sino-Tibetan
\item[PTB]	Proto-Tibeto-Burman
\item[Skt.]	Sanskrit
\item[ST]	Sino-Tibetan
\item[TB]	Tibeto-Burman
\item[TK]	Tai-Kadai
\item[WB]	Written Burmese
\item[WT]	Written Tibetan
\end{description}
\end{multicols}

\section*{Conventions Used in Citing the Source of Individual Wordforms}
\begin{multicols}{2}
The wide variety of sources included in the database make it difficult to provide a single pointer to a wordform in a source.
Typically, an inline bibliographic citation includes something like the author’s name, a year (with alphabetic disambiguator if needed)
and a page number.

In the STEDT database, many sources do not have page numbers; and at any rate a page number is often not the most useful identifier.
We therefore developed an informal system for marking provenance:

\begin{itemize}
\item If the source is short and has no page numbers, we have usually omitted source identfiers. A reader who is curious to examine 
the wordform in its context must locate the hardcopy and find the item using the collation and indexing conventions provided.
\item For etymological sources, e.g.\ the \textit{Conspectus} (\textit{\citetalias{STC}}), we used the etymological set number;
since many forms are cited that do not belong to sets, we have indicated the page and note number where the form is to be found, e.g.~123n5.
\item A number of sources utilize some sort of hierarchical (e.g.\ semantic) organization of the forms. In most of these cases, we used the
hierarchical numbering system provided.
\end{itemize}

\end{multicols}


\section*{Miscellaneous}
\begin{multicols}{2}
\begin{description}
% \item[“add-sourcing”]	the often laborious process by which a new source is added to the STEDT database
\item[allofam]	a variant of an etymon
\item[allofam box]	a computer display where all the allofams of an etymon are grouped together within a box on the screen
% \item[ancillary databases]	searchable databases on particular languages or sources, accompaniments to the main STEDT databases (lexical and etymological)
\item[chapter]	the smallest section of a fascicle, containing all the etyma reconstructed with a given gloss, e.g. KIDNEY
% \item[etyma table]	the database table where all STEDT etymologies are stored
\item[extra-fascicular etyma]	etyma that are cited outside of their normal fascicle to illustrate a particular point
\item[fascicle]	a major section of a STEDT volume, comprising many chapters with closely related glosses, e.g. INTERNAL ORGANS
% \item[lexicon table]	the database table where the forms from individual languages are stored
\item[metastatic flowchart]	a diagram illustrating the patterns of semantic association displayed by an etymon
\item[meso-reconstruction]	reconstruction of an etymon at the subgroup level, deemed to be descended from that same etymon at a higher taxonomic level
\item[meso-root]	an etymon reconstructed at the subgroup level
\item[pan-allofamic formula (PAF)]	an abstract reconstruction intended to display simultaneously all the patterns of variation attested for a given etymon
\item[rectification]	the process by which etymologies are subjected to reanalysis, to determine whether they should be accepted as is, accepted with modifications, or rejected
\item[root canal]	the process through which a STEDT etymology is made public, so that it can be commented upon 
\item[semantic flowchart]	SEE metastatic flowchart
\item[semcat]	a semantic category; each STEDT etymology is assigned to one or more of them
\item[supporting forms]	forms from individual languages that support an etymology
\item[volume]	one of the ten major divisions of STEDT, e.g. BODY-PARTS
\item[weakly attested root]	a promising etymology, but with insufficient support
\item[>]	goes to; becomes
\item[<]	comes from; is derivable from
\item[*]    indicates a reconstructed form (e.g.\ PTB \textbf{*b‑ləy}), or acts as a synonym for ``proto-" (e.g.\ *labials, *Kuki-Chin)
\item[A \STEDTU{⪤} B]	A and B are co-allofams; A and B are members of the same word-family
\item[A \STEDTU{↭} B]	Are A and B co-allofams? Do A and B belong to the same word family?
\item[A \STEDTU{↮} B]	A and B are not co-allofams; A and B do not belong to the same word family
\item[Clf.]	classifier
\item[lit.]	literally
% \item[OICC]	Obscure internal channels and connections‚ (see Ch.~III)
\item[ult.]	ultimately
\end{description}
\end{multicols}

\section*{Abbreviations for People}
\begin{multicols}{2}
\begin{description}
\item[AH] Andr\'e-Georges Haudricourt
\item[B \& S] (William H.) Baxter \& (Laurent) Sagart
\item[JAM] James A. Matisoff
\item[JS] Jackson Sun
\item[KVB] Ken VanBik
\item[WHB] William H. Baxter
\item[WTF] Walter T. French (cf.\ \citealt{WTF-PNN})
\item[ZJH] Zev J. Handel
\end{description}
\end{multicols}

\section*{Abbreviations for Sources and Venues}
\begin{multicols}{2}
\begin{description}
\item[AD] \textit{Analytic Dictionary of Chinese and Sino-Japanese} (\citealt{BK-AD})
\item[DL] \textit{The Dictionary of Lahu} (\citealt{JAM-DL})
\item[GL] \textit{The Grammar of Lahu} (\citealt{JAM-GL})
\item[GSR] \textit{Grammata Serica Recensa} (\citealt{GSR})
\item[GSTC/G\&C] \textit{God and the Sino-Tibetan Copula} (\citealt{JAM-GSTC})
\item[HCT] \textit{A Handbook of Comparative Tai} (\citealt{LI1977})
\item[HPTB] Handbook of Proto-Tibeto-Burman (\citealt{JAM-HPTB})
\item[ICSTLL] International Conference on Sino-Tibetan Languages and Linguistics
\item[LTBA] \textit{Linguistics of the Tibeto-Burman Area}
\item[(L)TSR] \textit{The Loloish Tonal Split Revisited} (\citealt{JAM-TSR})
\item[STC] \textit{Sino-Tibetan: a Conspectus} (\citealt{STC})
\item[TBL]  \SC{藏缅语族语言词汇} \textit{[A Tibeto-Burman lexicon]} (\citealt{TBL})
\item[TBRS] \textit{The Tibeto-Burman Reproductive System: Toward an Etymological Thesaurus} (\citealt{JAM-TBRS})
\item[VSTB] Variational Semantics in Tibeto-Burman: The ‘Organic’ Approach to Linguistic Comparison (\citealt{JAM-VSTB})
\item[ZMYYC] \SC{ 藏缅语语音和词汇} \textit{[Tibeto-Burman phonology and lexicon]} (\citealt{ZMYYC})
\end{description}
\end{multicols}

