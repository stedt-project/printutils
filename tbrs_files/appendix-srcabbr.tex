\backmatter
\appendix
\chapter{Appendix: Source Abbreviations}

{\footnotesize
\begin{longtable}{l>{\hangindent=0.25in}p{5.6in}}

AAAM-SSM &
Abbi, Anvita and Awadhesh K.~Mishra.
1985.
“Consonant clusters and syllable structure of Meitei.”
\textit{LTBA} 8.2:81-92.
\\[0.8\parskip]

ACST &
Chou Fa-kao \TC{周法高}.
1972.
“Archaic Chinese and Sino-Tibetan.”
\textit{Journal of the Institute of Chinese Studies of the Chinese University of Hong Kong}  5.1:159-237.
\\[0.8\parskip]

AH-CSDPN &
Hale, Austin.
1973.
\textit{Clause, Sentence, and Discourse Patterns in Selected Languages of Nepal IV: Word Lists.}
Summer Institute of Linguistics Publications in Linguistics and Related Fields 40. Kathmandu: SIL and Tribhuvan University Press.
\\[0.8\parskip]

AT-MPB &
Tayeng, Aduk.
1976.
\textit{Milang phrase book.}
Shillong: The Director of Information and Public Relations, Government of Arunachal Pradesh.
\\[0.8\parskip]

AW-TBT &
Weidert, Alfons K.
1987.
\textit{Tibeto-Burman Tonology: a comparative account.}
\textit{Current Issues in Linguistic Theory}, Vol.~54.  Amsterdam and Philadelphia: John Benjamins Publishing Co.
\\[0.8\parskip]

B-ShrpaHQ &
Bishop, Naomi.
1989.
Body Parts Questionnaire (Sherpa Helambu).
\\[0.8\parskip]

BB-Belhare &
Bickel, Balthasar.
1995.
“The possessive of experience in Belhare.”
In David Bradley, ed., \textit{Tibeto-Burman Languages of the Himalayas.} Canberra: Pacific Linguistics (A-86), pp.~135-55.
\\[0.8\parskip]

Bhat-Boro &
Bhat, D.~N.~Shankara.
1968.
\textit{Boro Vocabulary, with a grammatical sketch.}
Deccan College Building Centenary and Silver Jubilee Series \#59.  Poona: Deccan College Postgraduate and Research Institute.
\\[0.8\parskip]

Bhat-TNV &
Bhat, D.~N.~Shankara.
1969.
\textit{Tankhur Naga Vocabulary.}
Deccan College Building Centenary and Silver Jubilee Series \#67.  Poona: Deccan College Postgraduate and Research Institute.
\\[0.8\parskip]

BM-Bah &
Michailovsky, Boyd.
1989.
“Bahing.”
Electronic ms.
\\[0.8\parskip]

BM-Hay &
Michailovsky, Boyd.
1989.
“Hayu.”
Electronic ms.
\\[0.8\parskip]

BM-Lim &
Michailovsky, Boyd.
1989.
“Limbu.”
Electronic ms.
\\[0.8\parskip]

BM-PK7 &
Michailovsky, Boyd.
1991.
“Proto-Kiranti forms.”
Unpublished ms.
\\[0.8\parskip]

CB-SpitiQ &
Bodh, Sri Chhimed.
1991.
Body Parts Questionnaire (Spiti).
\\[0.8\parskip]

CG-Dolak &
Genetti, Carol\@.
ca.~1990.
Dolakhali (Newari) word list.
\\[0.8\parskip]

CG-Kath &
Genetti, Carol\@.
ca.~1990.
Kathmandu Newari word list.
\\[0.8\parskip]

CG-NewariQ3 &
Genetti, Carol.
1990.
Natural Objects Questionnaire.
\\[0.8\parskip]

CK-TujBQ &
Chen Kang \SC{陈康}.
1986.
Body Parts Questionnaire (Tujia, Bizika dialect).
\\[0.8\parskip]

CK-TujMQ &
Chen Kang \SC{陈康}.
1986.
Body Parts Questionnaire (Tujia, Mondzi dialect).
\\[0.8\parskip]

CK-YiQ &
Chen Kang \SC{陈康}.
1986.
Body Parts Questionnaire (8 Yi dialects).
\\[0.8\parskip]

CSL-YIzd &
Chen Shilin \SC{陈士林}, Li Min \SC{李民}, et al., eds.
1979.
\SC{彝汉字典} \textit{Yí-Hàn zìdiǎn [Yi-Chinese dictionary].}
Chengdu: Yi Language Work Unit, People’s Committee of Sichuan.
\\[0.8\parskip]

CYS-Meithei &
Singh, Chungkham Yashawanta.
1991.
Body Parts Questionnaire (Meithei).
\\[0.8\parskip]

DAP-Chm &
Peterson, David A.
2008.
“Bangladesh Khumi verbal classifiers and Kuki-Chin ‘chiming’.”
\textit{LTBA}, to appear.
\\[0.8\parskip]

DB-Bisu &
Bradley, David\@.
ca.~1993.
Bisu vocabulary, extracted from DB-PLolo.
\\[0.8\parskip]

DB-Lahu &
Bradley, David.
1979.
\textit{Lahu Dialects.}
Oriental Monograph Series, \#23. Canberra: Australian National University.
\\[0.8\parskip]

DB-Lisu &
Bradley, David.
1994.
\textit{A Dictionary of the Northern Dialect of Lisu (China and Southeast Asia).} % [Based on Xu, Mu et al. 1985]
Pacific Linguistics Series C-126.  Canberra: Australian National University.
\\[0.8\parskip]

DB-Phunoi &
Bradley, David\@.
ca.~1993.
Phunoi vocabulary, extracted from DB-PLolo.
\\[0.8\parskip]

DB-PLolo &
Bradley, David.
1979.
\textit{Proto-Loloish.}
Scandinavian Institute of Asian Studies Monograph Series, \#39.  London and Malmö: Curzon Press.
\\[0.8\parskip]

DB-Ugong &
Bradley, David.
1993.
Body Parts Questionnaire (Ugong).
\\[0.8\parskip]

DBS-PaO &
Solnit, David.
1989.
Pa-O word list.
Electronic ms.
\\[0.8\parskip]

Deuri &
Anonymous\@.
n.d.
Deuri body part terms.
\\[0.8\parskip]

DHFRL &
Dai Qingxia \SC{戴庆厦} et al., eds.
1991.
\SC{藏缅语十五种} \textit{Zàngmiǎnyǔ shíwǔzhǒng [Fifteen Tibeto-Burman languages].}
Beijing: \SC{燕山出版社} Yānshān Chūbǎnshè.
\\[0.8\parskip]

DK-Moyon &
Kosha, Donald.
1990.
Body Parts Questionnaire (Moyon).
\\[0.8\parskip]

DLF-Gazhuo &
Dai Qingxia \SC{戴庆厦}, Liu Juhuang \SC{刘菊黄}, and Fu Ailan \SC{傅爱兰}.
1987.
\SC{云南蒙古族嘎卓语研究} “On the Gazhuo language of the Mongolian people of Yunnan Province.”
\SC{语言研究} \textit{Yǔyán Yánjiū}, No.~1.
\\[0.8\parskip]

DNW-KhamQ &
Watters, David and Nancy Watters.
1989.
Body Parts Questionnaire (Kham).
unpublished computer file.
\\[0.8\parskip]

DQ-Batang &
Dai Qingxia \SC{戴庆厦}.
1989.
Body Parts Questionnaire (Batang).
\\[0.8\parskip]

DQ-Bola &
Dai Qingxia \SC{戴庆厦}.
1989.
Field Notebook on Bola.
\\[0.8\parskip]

DQ-Daofu &
Dai Qingxia \SC{戴庆厦}.
1989.
Body Parts Questionnaire (Daofu).
\\[0.8\parskip]

DQ-Gazhuo &
Dai Qingxia \SC{戴庆厦}.
1989.
Body Parts Questionnaire (Gazhuo).
\\[0.8\parskip]

DQ-Jiarong &
Dai Qingxia \SC{戴庆厦}.
1989.
Body Parts Questionnaire (rGyalrong).
\\[0.8\parskip]

DQ-JinA &
Dai Qingxia \SC{戴庆厦}.
1989.
Field Notebook on Jinuo A.
\\[0.8\parskip]

DQ-JinB &
Dai Qingxia \SC{戴庆厦}.
1989.
Field Notebook on Jinuo B.
\\[0.8\parskip]

DQ-KarenA &
Dai Qingxia \SC{戴庆厦}.
1989.
Field Notebook on Karen A.
\\[0.8\parskip]

DQ-KarenB &
Dai Qingxia \SC{戴庆厦}.
1989.
Field Notebook on Karen B.
\\[0.8\parskip]

DQ-Langsu &
Dai Qingxia \SC{戴庆厦}.
1989.
Field Notebook on Langsu [Maru].
\\[0.8\parskip]

DQ-Lashi &
Dai Qingxia \SC{戴庆厦}.
1989.
Field Notebook on Leqi [Lashi].
\\[0.8\parskip]

DQ-Lolopho &
Dai Qingxia \SC{戴庆厦}.
1989.
Field Notebook on Lolopho.
\\[0.8\parskip]

DQ-NusuA &
Dai Qingxia \SC{戴庆厦}.
1989.
Field Notebook on Nusu A.
\\[0.8\parskip]

DQ-NusuB &
Dai Qingxia \SC{戴庆厦}.
1989.
Field Notebook on Nusu B.
\\[0.8\parskip]

DQ-QiangN &
Dai Qingxia \SC{戴庆厦}.
1989.
Field Notebook on Northern Qiang.
\\[0.8\parskip]

DQ-Xiandao &
Dai Qingxia \SC{戴庆厦}.
1989.
Field Notebook on Achang (Xiandao).
\\[0.8\parskip]

DQ-Xixia &
Dai Qingxia \SC{戴庆厦}.
1989.
Body Parts Questionnaire (Xixia = Tangut).
\\[0.8\parskip]

DS-Kan &
Sharma, D.D.
1988.
\textit{A Descriptive Grammar of Kinnauri.}
Delhi: Mittal Publications (Studies in Tibeto-Himalayan Languages \#1).
\\[0.8\parskip]

DS-Patt &
Sharma, D.D.
1982.
\textit{Studies in Tibeto-Himalayan Linguistics: a descriptive analysis of Pattani (a dialect of Lahaul).}
Hoshiarpur: Vishveshvaranand Vishva Bandhu Institute of Sanskrit and Indological Studies, Panjab University.
\\[0.8\parskip]

EA-Tsh &
Andvik, Eric.
1993.
“Tshangla verb inflections.”
\textit{LTBA} 16.1:75-136.
\\[0.8\parskip]

EJAH-BKD &
Henderson, Eugénie J.~A.
1997.
\textit{Bwe Karen Dictionary.}
School of Oriental and African Studies, University of London.
\\[0.8\parskip]

EJAH-Hpun &
Henderson, Eugénie J.~A.
1986.
“Some hitherto unpublished material on Northern (Megyaw) Hpun.”
In John McCoy and Timothy Light, eds., \textit{Contributions to Sino-Tibetan Studies}, pp.\ 101-34.  Leiden: E.J. Brill.
\\[0.8\parskip]

EJAH-TC &
Henderson, Eugénie J.~A.
1965.
\textit{Tiddim Chin: a descriptive analysis of two texts.}
London Oriental Series \#15.   London and New York: Oxford University Press.
\\[0.8\parskip]

FD-Bai &
Dell, François.
1981.
\textit{La langue Bai: phonologie et lexique.}
Paris: Centre de Recherches Linguistiques sur l’Asie Orientale de l’Ecole des Hautes Etudes en Sciences Sociales.
\\[0.8\parskip]

GBM-Lepcha &
Mainwaring, G.B.
1898.
\textit{Dictionary of the Lepcha Language.}
Revised and completed by Albert Grünwedel. Berlin: Unger Brothers.
\\[0.8\parskip]

GDW-DML &
Walker, George David.
1925.
\textit{A Dictionary of the Mikir language, Mikir-English and English-Mikir.}
Shillong: Assam Government Press.
\\[0.8\parskip]

GEM-CNL &
Marrison, G.E.
1967.
\textit{The Classification of the Naga Languages of Northeast India.}
Ph.D. dissertation, School of Oriental and African Studies, University of London.  2 vols.
\\[0.8\parskip]

GHL-PPB &
Luce, G.~H.
1986.
\textit{Phases of Pre-Pagán Burma: languages and history.}
Vol.~2. Oxford: Oxford University Press.
\\[0.8\parskip]

GSR &
Karlgren, Bernhard.
1957.
\textit{Grammata Serica Recensa.}
Stockholm: Museum of Far Eastern Antiquities, Publication 29.
\\[0.8\parskip]

HAJ-TED &
Jäschke, Heinrich August.
1881/1958.
\textit{A Tibetan-English Dictionary, with special reference to the prevailing dialects.}
London. Reprinted (1958) by Routledge and Kegan Paul.
\\[0.8\parskip]

HM-Prak &
Hoshi Michiyo.
1984.
\textit{A Prakaa Vocabulary: a dialect of the Manang language.}
Anthropological and Linguistic Studies of the Gandaki Area in Nepal II.  (\textit{Monumenta Serindica} \#12.)  Tokyo: ILCAA.
\\[0.8\parskip]

ILH-PL &
Hansson, Inga-Lill.
1989.
“A comparison of Akha, Hani, Khatu, and Pijo.”
\textit{LTBA} 12.1:1-91.
\\[0.8\parskip]

IMS-HMLG &
Simon, Ivan Martin.
1976.
\textit{Hill Miri Language Guide.}
Shillong: Philological Section, Research Dept., Government of Arunachal Pradesh.
\\[0.8\parskip]

IMS-Miji &
Simon, Ivan Martin.
1979.
“Miji language guide.”
Shillong: Directorate of Research (Philological Section) Government of Arunachal Pradesh.
\\[0.8\parskip]

JAM-DL &
Matisoff, James A.
1988.
\textit{The Dictionary of Lahu.}
UCPL \#111.  Berkeley, Los Angeles, London: University of California Press.
\\[0.8\parskip]

JAM-Ety &
Matisoff, James A.
1987.
Body part card file.
\\[0.8\parskip]

JAM-GSTC &
Matisoff, James A.
1985.
“God and the Sino-Tibetan copula, with some good news concerning selected Tibeto-Burman rhymes.”
\textit{Journal of Asian and African Studies} (Tokyo) 29:1-81.
\\[0.8\parskip]

JAM-II &
Matisoff, James A.
1993.
Personal communications from JAM, more recent than the Body Part Card File.
\\[0.8\parskip]

JAM-MLBM &
Matisoff, James A.
1978.
“Mpi and Lolo-Burmese microlinguistics.”
\textit{Monumenta Serindica} (ILCAA, Tokyo) 4:1-36.
\\[0.8\parskip]

JAM-Rong &
Matisoff, James A.
1994.
Rongmei elicitation.
\\[0.8\parskip]

JAM-TIL &
Matisoff, James A.
1983.
“Translucent insights:  a look at Proto-Sino-Tibetan through Gordon H. Luce’s comparative word-list.”
\textit{BSOAS} 46.3:462-76.
\\[0.8\parskip]

JAM-TJLB &
Matisoff, James A.
1974.
“The tones of Jinghpaw and Lolo-Burmese: common origin vs. independent development.”
\textit{Acta Linguistica Hafniensia} (Copenhagen) 15.2, 153-212.
\\[0.8\parskip]

JAM-TSR &
Matisoff, James A.
1972.
\textit{The Loloish Tonal Split Revisited.}
Research Monograph \#7.  Berkeley: Center for South and Southeast Asian Studies, University of California, Berkeley.
\\[0.8\parskip]

JAM-VSTB &
Matisoff, James A.
1978.
\textit{Variational Semantics in Tibeto-Burman: the ‘organic’ approach to linguistic comparison.}
\textit{OPWSTBL} \#6.  Philadelphia: Institute for the Study of Human Issues.
\\[0.8\parskip]

JCD &
Dai Qingxia \SC{戴庆厦}, Xu Xijian \SC{徐悉艰} , et al.
1983.
\SC{景汉辞典} \textit{Jing-Han cidian – Jinghpo Miwa ga ginsi chyum – Jinghpo-Chinese dictionary.}
Kunming: Yunnan Nationalities Press.
\\[0.8\parskip]

JF-HLL &
Fraser, James Outram.
1922.
\textit{Handbook of the Lisu (Yawyin) Language.}
Rangoon: Office of the Superintendent of Government Printing.
\\[0.8\parskip]

JHL-AM &
Lorrain, J.~Herbert.
1907.
\textit{A Dictionary of the Abor-Miri Language, with illustrative sentences and notes.}
Shillong: Eastern Bengal and Assam Secretariat Printing Office.
\\[0.8\parskip]

JHL-Lu &
Lorrain, J.~Herbert.
1940.
\textit{Dictionary of the Lushai Language.}
Bibliotheca Indica 261. Calcutta: Royal Asiatic Society of Bengal.
\\[0.8\parskip]

JK-Dh &
King, John.
1994.
Dhimal body parts.
Personal communication.
\\[0.8\parskip]

JO-PB &
Okell, John.
1971.
“\textbf{K-} clusters in Proto-Burmese.”
Paper presented at ICSTLL \#4, Indiana University, Bloomington, IN.
\\[0.8\parskip]

JP-Idu &
Pulu, Jatan.
1978.
\textit{Idu Phrase Book.}
Shillong: The Director of Information and Public Relations, Arunachal Pradesh.
\\[0.8\parskip]

JS-Amdo &
Sun, Jackson \TC{孫天心}.
1985.
\textit{Aspects of the Phonology of Amdo Tibetan.}
M.A. thesis, Institute of English, National Normal University, Taipei. Published 1986, Monumenta Serindica No.~16, Tokyo: ILCAA.
\\[0.8\parskip]

JS-Ch &
Sun, Jackson \TC{孫天心}.
1985.
Chinese glosses, excerpted from JS-Amdo.
\\[0.8\parskip]

JS-HCST &
Sun, Jackson \TC{孫天心}.
1993.
\textit{A Historical-Comparative Study of the Tani (Mirish) Branch in Tibeto-Burman.}
Ph.D. dissertation, University of California, Berkeley.
\\[0.8\parskip]

JS-Mawo &
Sun, Jackson \TC{孫天心}\@.
ca.~1986.
Qiang Mawo body part word list.
Unpublished ms.
\\[0.8\parskip]

JS-Tani &
Sun, Jackson \TC{孫天心}.
1993.
“Tani synonym sets.”
Electronic ms.
\\[0.8\parskip]

JS-Tib &
Sun, Jackson \TC{孫天心}.
1985.
Tibetan glosses, excerpted from JS-Amdo.
\\[0.8\parskip]

JZ-Achang &
Dai Qingxia \SC{戴庆厦} and Cui Zhichao \SC{崔志超}, eds.
1985.
\SC{阿昌语简志} \textit{Āchāngyǔ jiǎnzhì [Brief description of the Achang language].}
Beijing: \SC{民族出版社} Nationalities Press.
\\[0.8\parskip]

JZ-Bai &
Xu Lin \SC{徐琳} and Zhao Yansun \SC{赵衍荪}, eds.
1984.
\SC{白语简志} \textit{Báiyǔ jiǎnzhì [Brief description of the Bai language].}
Beijing: \SC{民族出版社} Nationalities Press.
\\[0.8\parskip]

JZ-CLMenba &
Zhang Jichuan \SC{张济川}, ed.
1986.
\SC{仓洛门巴语简志} \textit{Cāngluò Ménbāyǔ jiǎnzhì [Brief description of the Cangluo Menba language].}
Beijing: \SC{民族出版社} Nationalities Press.
\\[0.8\parskip]

JZ-CNMenba &
Lu Shaozun \SC{陆绍尊}, ed.
1986.
\SC{错那门巴语简志} \textit{Cuònà Ménbāyǔ jiǎnzhì [Brief description of the Cuona Menba language].}
Beijing: \SC{民族出版社} Nationalities Press.
\\[0.8\parskip]

JZ-Dulong &
Sun Hongkai \SC{孙宏开}, ed.
1982.
\SC{独龙语简志} \textit{Dúlóngyǔ jiǎnzhì [Brief description of the Dulong language].}
Beijing: \SC{民族出版社} Nationalities Press.
\\[0.8\parskip]

JZ-Hani &
Li Yongsui \SC{李永燧} and Wang Ersong \SC{王尔松}, eds.
1986.
\SC{哈尼语简志} \textit{Hāníyǔ jiǎnzhì [Brief description of the Hani language].}
Beijing: \SC{民族出版社} Nationalities Press.
\\[0.8\parskip]

JZ-Jingpo &
Liu Lu \SC{刘璐}, ed.
1984.
\SC{景颇族语言简志(景颇语)} \textit{Jǐngpōzú yǔyán jiǎnzhì (Jǐngpōyǔ) [Brief description of the Jingpo language of the Jingpo people].}
Beijing: \SC{民族出版社} Nationalities Press.
\\[0.8\parskip]

JZ-Jinuo &
Gai Xingzhi \SC{盖兴之}, ed.
1986.
\SC{基诺语简志} \textit{Jīnuòyǔ jiǎnzhì [Brief description of the Jinuo language].}
Beijing: \SC{民族出版社} Nationalities Press.
\\[0.8\parskip]

JZ-Lahu &
Chang Hong’en \SC{常竑恩} et al., eds.
1986.
\SC{拉祜语简志} \textit{Lāhùyǔ jiǎnzhì [Brief description of the Lahu language].}
Beijing: \SC{民族出版社} Nationalities Press.
\\[0.8\parskip]

JZ-Lisu &
Xu Lin \SC{徐琳}, Mu Yuzhang \SC{木玉璋}, Gai Xingzhi \SC{盖兴之}, eds.
1986.
\SC{傈僳语简志} \textit{Lìsùyǔ jiǎnzhì [Brief description of the Lisu language].}
Beijing: \SC{民族出版社} Nationalities Press.
\\[0.8\parskip]

JZ-Naxi &
He Jiren \SC{和即仁} and Jiang Zhuyi \SC{姜竹仪}, eds.
1985.
\SC{纳西语简志} \textit{Nàxīyǔ jiǎnzhì [Brief description of the Naxi language].}
Beijing: \SC{民族出版社} Nationalities Press.
\\[0.8\parskip]

JZ-Nusu &
Sun Hongkai \SC{孙宏开} and Liu Lu \SC{刘璐}, eds.
1986.
\SC{怒族语言简志(怒苏语)} \textit{Nùzú yǔyán jiǎnzhì (Nùsūyǔ) [Brief description of the Nusu language of the Nu people].}
Beijing: \SC{民族出版社} Nationalities Press.
\\[0.8\parskip]

JZ-Pumi &
Lu Shaozun \SC{陆绍尊}, ed.
1983.
\SC{普米语简志} \textit{Pǔmǐyǔ jiǎnzhì [Brief description of the Pumi language].}
Beijing: \SC{民族出版社} Nationalities Press.
\\[0.8\parskip]

JZ-Qiang &
Sun Hongkai \SC{孙宏开}, ed.
1981.
\SC{羌语简志} \textit{Qiāngyǔ jiǎnzhì [Brief description of the Qiang language].}
Beijing: \SC{民族出版社} Nationalities Press.
\\[0.8\parskip]

JZ-Tujia &
Tian Desheng \SC{田德生}, He Tianzhen \SC{何天贞} et al., eds.
1986.
\SC{土家语简志} \textit{Tǔjiāyǔ jiǎnzhì [Brief description of the Tujia language].}
Beijing: \SC{民族出版社} Nationalities Press.
\\[0.8\parskip]

JZ-Yi &
Chen Shilin \SC{陈士林}, Bian Shiming \SC{边仕明}, Li Xiuqing \SC{李秀清}, eds.
1985.
\SC{彝语简志} \textit{Yíyǔ jiǎnzhì [Brief description of the Yi language].}
Beijing: \SC{民族出版社} Nationalities Press.
\\[0.8\parskip]

JZ-Zaiwa &
Xu Xijian \SC{徐悉艰} and Xu Guizhen \SC{徐桂珍}, eds.
1984.
\SC{景颇族语言简志(载瓦语)} \textit{Jǐngpōzú yǔyán jiǎnzhì (Zàiwǎyǔ) [Brief description of the Zaiwa language of the Jingpo people].}
Beijing: \SC{民族出版社} Nationalities Press.
\\[0.8\parskip]

KDG-ICM &
Das Gupta, K.
1968.
\textit{An Introduction to Central Monpa.}
Shillong: Philology Section, Research Department, North-East Frontier Agency.
\\[0.8\parskip]

KDG-IGL &
Das Gupta, K.
1963.
\textit{An Introduction to the Gallong Language.}
Shillong: Philological Section, Research Department, North-East Frontier Agency.
\\[0.8\parskip]

KDG-Tag &
Das Gupta, K.
1983.
\textit{An Outline on Tagin Language.}
Directorate of Research, Government of Arunachal Pradesh.
\\[0.8\parskip]

KHG-Mikir &
Grüssner, Karl-Heinz.
1978.
\textit{Arleng Alam, die Sprache der Mikir: Grammatik und Texte.}
Wiesbaden: Franz Steiner.
\\[0.8\parskip]

KPM-pc &
Malla, Kamal P.
2007.
Personal communications.
\\[0.8\parskip]

KVB-Lai &
Van Bik, Kenneth.
1995-.
Personal communications.
\\[0.8\parskip]

KVB-PKC &
Van Bik, Kenneth.
2007.
\textit{Proto-Kuki-Chin.}
Ph.D. dissertation, University of California, Berkeley.
\\[0.8\parskip]

LL-PRPL &
Löffler, Lorenz G.
1985.
“A preliminary report on the Paangkhua language.”
In Graham Thurgood, et al., eds., \textit{Linguistics of the Sino-Tibetan area: the state of the art}, pp.\ 279-286.  (Pacific Linguistics Series C, No.~87).  Canberra: Australian National University.
\\[0.8\parskip]

LMZ-AhiQ &
Luo Meizhen\@.
ca.~1990.
Body Parts Questionnaire (Yi: Ahi).
\\[0.8\parskip]

LYS-Sangkon &
Li Yongsui \SC{李永燧}.
1991.
\SC{缅彝语言调查的新收获:桑孔语} “Mian-Yi yuyan diaocha de xin shouhuo: Sangkongyu [A new harvest from research into Burmese-Yi: the Sangkong language].”
Presented at the Fifth International Yi-Burmese Conference. Xichang, Sichuan.  Beijing: Institute of Nationality Studies, Chinese Academy of Social Sciences.
\\[0.8\parskip]

MB-Lal &
Balawan, M.
1965.
\textit{A First Lalung Dictionary, with the corresponding words in English and Khasi.}
Shillong.
\\[0.8\parskip]

MF-PhnQ &
Ferlus, Michel.
1991.
Body Parts Questionnaire (Phunoi).
\\[0.8\parskip]

MM-K78 &
Mazaudon, Martine.
1978.
“Consonantal mutation and tonal split in the Tamang sub-family of Tibeto-Burman.”
\textit{Kailash} 6.3:157-79.
\\[0.8\parskip]

MM-TamRisQ &
Mazaudon, Martine.
1991.
Body Parts Questionnaire (Tamang: Risiangku).
\\[0.8\parskip]

MM-Thesis &
Mazaudon, Martine.
1994.
\textit{Problèmes de comparatisme et de reconstruction dans quelques langues de la famille tibéto-birmane.}
Thèse d’État, Université de la Sorbonne Nouvelle, Paris.
\\[0.8\parskip]

MVS-Grin &
Sofronov, M.V\@.
ca.~1978.
“Annotations to Grinstead 1972.”
Reconstructions of Tangut body part terms, personally entered into the glossary of Grinstead 1972. % EG-Tangut
\\[0.8\parskip]

MXL-Lolo &
Ma Xueliang \SC{马学良}.
1948.
\TC{倮文作祭獻藥供牲經譯注} \textit{Luǒwén \textnormal{Zuòjì, xiànyào, gōngshēngjīng} yìzhù [Annotated Translation of \textnormal{The Lolo Classic of Rites, Cures, and Sacrifices}].}
\textit{AS/BIHP} 20:577-666.
\\[0.8\parskip]

MXL-SaniQ &
Ma Xueliang \SC{马学良}\@.
ca.~1989.
Field Notebook.
\\[0.8\parskip]

NEFA-PBI &
Anonymous.
1962.
\textit{A Phrase Book in Idu.}
Shillong: Philological Section, Research Department,  North-East Frontier Agency.
\\[0.8\parskip]

NEFA-Taraon &
Anonymous\@.
n.d.
\textit{Taraon.}
Shillong: Philological Section, Research Department,  North-East Frontier Agency.
\\[0.8\parskip]

NJA-Thulung &
Allen, N.J.
1975.
\textit{Sketch of Thulung Grammar.}
East Asian Papers \#6.  Ithaca: China-Japan Program, Cornell University.
\\[0.8\parskip]

NKR-Bant &
Rai, Novel Kishore.
1985.
\textit{A Descriptive Study of Bantawa.}
Poona: Deccan College Post-Graduate and Research Institute.
\\[0.8\parskip]

NPB-ChanQ &
Noonan, Michael, W.~Pagliuca, and R.~Bhulanja.
1992.
Body Parts Questionnaire (Chantyal).
\\[0.8\parskip]

NT-SGK &
Nishida Tatsuo \TC{西田龍雄}.
1964, 1966.
\TC{西夏語の研究} \textit{Seikago no kenkyū [A Study of the Hsi-Hsia Language: reconstruction of the Hsi-Hsia language and decipherment of the Hsi-Hsia script].}
Tokyo: \TC{座右宝刊行会} Zauhō Kankōkai.  2 vols. Vol.~I (1964), Vol.~II (1966).
\\[0.8\parskip]

OH-DKL &
Hanson, Ola.
1906.
\textit{A Dictionary of the Kachin Language.}
Rangoon. Reprinted (1954, 1966), Rangoon: Baptist Board of Publications.
\\[0.8\parskip]

PB-Bisu &
Beaudouin, Patrick.
1988.
\textit{Glossary English-French-Bisu; Bisu-English-French.}
Nice, France: Section de Linguistique. U.E.R.\ Lettres, Université de Nice.
\\[0.8\parskip]

PB-CLDB &
Bhaskararao, Peri.
1996.
“A computerized lexical database of Tiddim Chin and Lushai.”
In Nara, Tsuyoshi and Machida, Kazuhiko (eds.), A Computer-Assisted Study of South-Asian Languages, pp.\ 27-143. Report \#6. Tokyo: ILCAA.
\\[0.8\parskip]

PB-TCV &
Bhaskararao, Peri.
1994.
“Tiddim Chin verbs and their alternants.”
\textit{Journal of Asian and African Studies.}  Nos.\ 46-47.
\\[0.8\parskip]

PKB-KSEA &
Benedict, Paul K.
1941/2008.
\textit{Kinship in Southeastern Asia.}
Ph.D. dissertation, Department of Anthropology, Harvard University (1941). To be published as STEDT Monograph \#6, 2008.
\\[0.8\parskip]

PKB-WBRD &
Benedict, Paul K.
1976.
\textit{Rhyming Dictionary of Written Burmese.}
\textit{LTBA} 3.1:1-93.
\\[0.8\parskip]

PL-AED &
Lewis, Paul.
1968.
\textit{Akha-English Dictionary.}
Data Paper \#70, Linguistics Series III.  Ithaca: Cornell University, Southeast Asia Program.
\\[0.8\parskip]

PL-AETD &
Lewis, Paul.
1989.
\textit{Akha-English-Thai Dictionary.}
Chiang Rai, Thailand: Development \& Agricultural Project for Akha.
\\[0.8\parskip]

PT-Kok &
Tripuri, Prashanta and Dan Jurafsky.
1988.
\textit{Kokborok Word List.}
Unpublished ms.
\\[0.8\parskip]

Qbp-KC &
Thien Haokip.
1998.
Body Parts Questionnaire (Kuki-Chin).
\\[0.8\parskip]

RAN1975 &
Rangan, K.
1975.
\textit{Balti Phonetic Reader.}
Phonetic Reader Series, \#17.  Mysore: CIIL.
\\[0.8\parskip]

RB-GB &
Burling, Robbins.
1992.
\textit{Garo (Bangladesh dialect) Semantic Dictionary.}
\\[0.8\parskip]

RB-LMMG &
Burling, Robbins.
2003.
\textit{The Language of the Modhupur Mandi (Garo). Vol.~III: Glossary.}
Ann Arbor, Michigan.
\\[0.8\parskip]

RBJ-KLS &
Jones, Robert B., Jr.
1961.
\textit{Karen Linguistic Studies:  description, comparison, and texts.}
UCPL~\#25.  Berkeley and Los Angeles: University of California Press.
\\[0.8\parskip]

RC-ChepQ &
Caughley, Ross.
1990.
Body Parts Questionnaire (Chepang).
\\[0.8\parskip]

RJL-DPTB &
LaPolla, Randy J.
1987.
“Dulong and Proto-Tibeto-Burman.”
\textit{LTBA} 10.1:1-43.
\\[0.8\parskip]

RPHH-Kul &
Rai, Krishna Prasad, Anna Holzhausen, and Andreas Holzhausen.
1975.
“Kulung body part index from \textit{Kulung-Nepali-English Glossary}.”
Kathmandu: SIL and Institute of Nepal and Asian Studies, Tribhuvan University.
\\[0.8\parskip]

RSB-STV &
Bauer, Robert S.
1991.
“Sino-Tibetan *vulva.”
\textit{LTBA} 14.1:147-72.
\\[0.8\parskip]

SB-Lalo &
Björverud, Susanna.
1994.
“The phonology of Lalo.”
Paper presented at ICSTLL \#27, Sèvres/Paris.
\\[0.8\parskip]

SBN-BunQ &
Sharma, S.R.
1991.
Body Parts Questionnaire (Bunan).
\\[0.8\parskip]

SD-MPD &
Srinuan Duanghom.
1976.
\textit{An Mpi dictionary.}
Ed. by Woranoot Pantupong.  Bangkok: Working Papers in Phonetics and Phonology \#1, Indigenous Languages of Thailand Research Project, Central Institute of English Language.
\\[0.8\parskip]

SER-HSL/T &
Egli-Toduner, Susanna\@.
n.d.
\textit{Handbook of the Sharchhokpa-Lo/Tsangla (language of the people of eastern Bhutan).}
Thimphu, Bhutan: Helvetas.
\\[0.8\parskip]

SH-KNw &
Shakya, Daya Ratna and David Hargreaves.
1989.
Body Parts Questionnaire (Newari).
\\[0.8\parskip]

SHK-Anong &
Sun Hongkai \SC{孙宏开}.
1988.
“Notes on Anong, a new language.”
\textit{LTBA} 11.1:27-63.
\\[0.8\parskip]

SHK-BaimaQ &
Sun Hongkai \SC{孙宏开}.
1991.
Body Parts Questionnaire (Baima).
\\[0.8\parskip]

SHK-ErgDQ &
Sun Hongkai \SC{孙宏开}.
1991.
Body Parts Questionnaire (Ergong: Danba).
\\[0.8\parskip]

SHK-ErgNQ &
Sun Hongkai \SC{孙宏开}.
1991.
Body Parts Questionnaire (Ergong: Northern).
\\[0.8\parskip]

SHK-ErsCQ &
Sun Hongkai \SC{孙宏开}.
1991.
Body Parts Questionnaire (Ersu).
\\[0.8\parskip]

SHK-GuiqQ &
Sun Hongkai \SC{孙宏开}.
1991.
Body Parts Questionnaire (Guiqiong).
\\[0.8\parskip]

SHK-Idu &
Sun Hongkai \SC{孙宏开}.
1991.
Body Parts Questionnaire (Idu).
\\[0.8\parskip]

SHK-MawoQ &
Sun Hongkai \SC{孙宏开}.
1991.
Body Parts Questionnaire (Mawo).
\\[0.8\parskip]

SHK-MuyaQ &
Sun Hongkai \SC{孙宏开}.
1991.
Body Parts Questionnaire (Muya).
\\[0.8\parskip]

SHK-NamuQ &
Sun Hongkai \SC{孙宏开}.
1991.
Body Parts Questionnaire (Namuyi).
\\[0.8\parskip]

SHK-rGEQ &
Sun Hongkai \SC{孙宏开}.
1991.
Body Parts Questionnaire (rGyalrong: Eastern).
\\[0.8\parskip]

SHK-rGNQ &
Sun Hongkai \SC{孙宏开}.
1991.
Body Parts Questionnaire (rGyalrong: Northern).
\\[0.8\parskip]

SHK-rGNWQ &
Sun Hongkai \SC{孙宏开}.
1991.
Body Parts Questionnaire (rGyalrong: Northwest).
\\[0.8\parskip]

SHK-ShixQ &
Sun Hongkai \SC{孙宏开}.
1991.
Body Parts Questionnaire (Shixing).
\\[0.8\parskip]

SHK-Sulung &
Sun Hongkai \SC{孙宏开}.
1993.
Body Parts Questionnaire (Sulong).
\\[0.8\parskip]

SHK-ZhabQ &
Sun Hongkai \SC{孙宏开}.
1991.
Body Parts Questionnaire (Zhaba).
\\[0.8\parskip]

SIL-Chep &
Caughley, Ross.
1972.
\textit{A Vocabulary of the Chepang Language.}
Kirtipur, Kathmandu: SIL, Tribhuvan University.
\\[0.8\parskip]

SIL-Gur &
Glover, Warren W.
1972.
\textit{A Vocabulary of the Gurung Language.}
Kirtipur, Kathmandu: SIL, Tribhuvan University.
\\[0.8\parskip]

SIL-Sahu &
Taylor, Doreen, Fay Everitt, and Karna Bahadur Tamang.
1972.
\textit{A Vocabulary of the Tamang Language.}
Kirtipur, Kathmandu: SIL, Tribhuvan University.
\\[0.8\parskip]

SIL-Thak &
Hari, Maria.
1971.
\textit{A Vocabulary of the Thakali Language.}
Kirtipur, Kathmandu: SIL, Tribhuvan University.
\\[0.8\parskip]

SLZO-MLD &
Sun Hongkai \SC{孙宏开}, Lu Shaozun \SC{陆绍尊}, Zhang Jichuan \SC{张济川}, and Ouyang Jueya \SC{欧阳觉亚}, eds.
1980.
\SC{门巴、珞巴、僜人的语言} \textit{Menba, Luoba, Dengren de yuyan [The languages of the Monpa, Lhoba, and Deng peoples].}
Beijing: Social Sciences Press.
\\[0.8\parskip]

STC &
Benedict, Paul K.
1972.
\textit{Sino-Tibetan: a Conspectus.}
James A.\ Matisoff, contributing editor. Princeton-Cambridge Series in Chinese Linguistics, \#2.  New York: Cambridge University Press.
\\[0.8\parskip]

STP-ManQ &
Sharma, S.R.
1991.
Body Parts Questionnaire (Manchati).
\\[0.8\parskip]

SVD-Dum &
Driem, George van.
1993.
\textit{A Grammar of Dumi.}
Mouton Grammar Library \#10.  Berlin, New York: Mouton de Gruyter.
\\[0.8\parskip]

SVD-LimA &
Driem, George van.
1987.
\textit{A Grammar of Limbu.}
Mouton Grammar Library \#4.  Berlin, New York, Amsterdam: Mouton de Gruyter.
\\[0.8\parskip]

SY-KhözhaQ &
Yabu, Shiro.
1994.
Body Parts Questionnaire (Khözha).
\\[0.8\parskip]

T-KomRQ &
Toba, Sueyoshi and Allen Kom.
1991.
Body Parts Questionnaire (Kom Rem).
\\[0.8\parskip]

TBL &
Dai Qingxia \SC{戴庆厦}, et al., eds.
1992.
\SC{藏缅语族语言词汇} \textit{A Tibeto-Burman Lexicon.}
Beijing: Central Institute of Minorities. % (“TBL”)
\\[0.8\parskip]

THI1972 &
Thirumalai, M.S.
1972.
\textit{Thaadou Phonetic Reader.}
Phonetic Reader Series \#6.  Mysore: CIIL.
\\[0.8\parskip]

TK-Yakha &
Kohn, Tamara\@.
ca.~1990.
Body Parts Questionnaire (Yakha).
\\[0.8\parskip]

VN-AngQ &
Nienu, Vikuosa.
1990.
Body Parts Questionnaire (Angami Naga).
\\[0.8\parskip]

VN-ChkQ &
Nienu, Vikuosa.
1990.
Body Parts Questionnaire (Chokri).
\\[0.8\parskip]

VN-LothQ &
Nienu, Vikuosa.
1990.
Body Parts Questionnaire (Lotha).
\\[0.8\parskip]

WBB-Deuri &
Brown, W.B.
1895.
\textit{An Outline Grammar of the Deori Chutiya Language Spoken in Upper Assam, with an introduction, illustrative sentences, and short vocabulary.}
Shillong: Assam Secretariat Printing Office.
\\[0.8\parskip]

WHB-OC &
Baxter, William.
1992.
\textit{A Handbook of Old Chinese Phonology.}
Berlin, New York: Mouton de Gruyter.
\\[0.8\parskip]

WSC-SH &
Coblin, Weldon South.
1986.
\textit{A Sinologist’s Handlist of Sino-Tibetan Lexical Comparisons.}
Monumenta Serica Monograph Series, Vol.\ 18.  Nettetal: Steyler Verlag.
\\[0.8\parskip]

WTF-PNN &
French, Walter T.
1983.
\textit{Northern Naga: a Tibeto-Burman Mesolanguage.}
Ph.D. dissertation, City University of New York.
\\[0.8\parskip]

WW-Bant &
Rai, Novel Kishore, Tikka Ram Rai, and Werner Winter.
1984.
\textit{A Tentative Bantawa Dictionary.}
Unpublished ms.
\\[0.8\parskip]

WW-Cham &
Winter, Werner.
1985.
\textit{Materials Towards a Dictionary of Chamling: I.~Chamling-English; II.~English-Chamling.}
Based on data collected by Dhan Prasad Rai.  Preliminary Version.  Kiel: Linguistic Survey of Nepal.
\\[0.8\parskip]

YHJC-Sani &
Wu Zili \SC{武自立}, Ang Zhiling \SC{昂智灵}, Huang Jianmin \SC{黄健民}.
1984.
\SC{彝汉简明词典} \textit{Yí-Hàn jiǎnmíng cídiǎn [A Concise Yi-Chinese dictionary].}
Yunnan Nationalities Press.
\\[0.8\parskip]

YN-Man &
Nagano, Yasuhiko.
1984.
\textit{A Manang Glossary.}
Anthropological and Linguistic Studies of the Gandaki Area in Nepal II.  (\textit{Monumenta Serindica} \#12.)  Tokyo: ILCAA.
\\[0.8\parskip]

ZLS-Tib &
Zhang Liansheng \SC{张连生}.
1988.
\textit{A Handbook of Chinese, Tibetan and English Words.}
Unpublished ms.
\\[0.8\parskip]

ZMYYC &
Sun Hongkai \SC{孙宏开}, et al., eds.
1991.
\SC{藏缅语语音和词汇} \textit{Zàngmiǎnyǔ yǔyīn hé cíhuì [Tibeto-Burman Phonology and Lexicon].}
Beijing: Chinese Social Sciences Press.
\\[0.8\parskip]

ZYS-Bai &
Zhao Yansun \SC{赵衍荪}.
1990.
Body Parts Questionnaire (Bai).
\\[0.8\parskip]

\end{longtable}
}

\cleartooddpage[\thispagestyle{empty}]
